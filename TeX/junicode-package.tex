\documentclass{article}
\usepackage[medium,semiexpanded]{Junicode}
\usepackage{microtype}
\usepackage{multicol}
\usepackage{supertabular}
\linespread{1.1}
\title{Junicode}
\author{Peter S. Baker}
\date{}
\begin{document}
\maketitle

\section{Introduction}

This package supports the Junicode static fonts (version 2.204 or higher)
for XeLaTeX and LuaLaTeX. The current version of the Junicode font should
be installed in your system. If the font is included in your TeX installation,
it is an obsolete version: you should remove it if possible. This package loads
fontspec, so it is not necessary to load it separately, even if you are using
other fonts alongside Junicode.

\section{Loading Junicode}

Load Junicode in the usual way, with {\small\verb|\usepackage{Junicode}|}. Several options are available:

\begin{description}
\item[fonttype] The type of font to look for, CFF or TrueType. These font types differ in the
way they draw outlines, and their hinting technologies are very different. Values may be otf
(the default) or ttf, e.g.\\
\hspace*{1in}{\small\verb|\usepackage[fonttype=ttf]{Junicode}|}
\item[light] The weight of the type for the main text is Light instead of Regular.
\item[medium] The weight of the type for the main text is Medium, somewhat heavier than Regular.
\item[semibold] The weight of bold type is somewhat lighter than the usual bold. This may be a
good choice if you have selected the light option.
\item[condensed] The width of the type is narrow. Note that bold type cannot be condensed: when
this option is selected, any bold type in the text will have normal width.
\item[semicondensed] The width of the type is wider than condensed but narrower than the default.
Note that bold type cannot be semicondensed: when this option is selected, any bold type in the
text will have normal width.
\item[expanded] The width of the type is wide. Note that light type cannot be expanded: using
both the light and the expanded options will produce an error.
\item[semiexpanded] The width of the type is wider than the default but narrower than expanded.
Note that light type cannot be semiexpanded: using both the light and the semiexpanded options
will produce an error.
\end{description}

\section{Selecting Alternate Styles}

The Junicode font comes in thirty-eight styles: nineteen roman and nineteen italic. You can
switch to any of these styles with one of the following commands, which will be self-explanatory
if you keep these abbreviations in mind: Sm = Semi, Cond = Condensed, Exp = Expanded.

\begin{multicols}{3}
    \jCond\textbackslash jBold

    \textbackslash jBoldItalic
    
    \textbackslash jCond
    
    \textbackslash jCondItalic
    
    \textbackslash jCondLight
    
    \textbackslash jCondLightItalic
    
    \textbackslash jCondMedium
    
    \textbackslash jCondMediumItalic
    
    \textbackslash jExp
    
    \textbackslash jExpItalic
    
    \textbackslash jExpBold
    
    \textbackslash jExpBoldItalic
    
    \textbackslash jExpMedium
    
    \textbackslash jExpMediumItalic
    
    \textbackslash jExpSmbold
    
    \textbackslash jExpSmboldItalic
    
    \textbackslash jItalic
    
    \textbackslash jLight
    
    \textbackslash jLightItalic
    
    \textbackslash jMedium
    
    \textbackslash jMediumItalic
    
    \textbackslash jRegular
    
    \textbackslash jSmbold
    
    \textbackslash jSmboldItalic
    
    \textbackslash jSmCond
    
    \textbackslash jSmCondItalic
    
    \textbackslash jSmCondLight
    
    \textbackslash jSmCondLightItalic
    
    \textbackslash jSmCondMedium
    
    \textbackslash jSmCondMediumItalic
    
    \textbackslash jSmExp
    
    \textbackslash jSmExpItalic
    
    \textbackslash jSmExpBold
    
    \textbackslash jSmExpBoldItalic
    
    \textbackslash jSmExpMedium
    
    \textbackslash jSmExpMediumItalic
    
    \textbackslash jSmExpSmbold
    
    \textbackslash jSmExpSmboldItalic    
\end{multicols}

\section{Other Commands}

These commands do nothing more than wrap fontspec commands, which can still be
used with the Junicode package and are actually needed if you want to take full
advantage of the Junicode font’s features. Some of these commands are more mnemonic
than the corresponding fontspec commands, and others are more compact. Each command
also has a corresponding “text” command that works like 
{\small\verb|\textit{}|}—that is, it takes
as its sole argument the text to which the command will be applied. Each “text” command
consists of the main command with “text” prefixed—for example,
{\small\verb|\textInsularLetterForms{}|}
corresponding to {\small\verb|\InsularLetterForms|}.  For a fuller account of the OpenType features
applied by these commands, see Chapter 4, Feature Reference.

\begin{center}
\tablehead{\hline}
\tabletail{\hline}
\begin{supertabular}{| l | p{2.75in} |}
\textbackslash AltThornEth & Applies ss01, Alternate thorn and eth.\\\hline
\textbackslash InsularLetterForms & Applies ss02, Insular letter-forms.
This has an effect only with English and Irish text.\\\hline
\textbackslash IPAAlternates & Applies ss03, IPA alternates.\\\hline
\textbackslash HighOverline & Applies ss04, High Overline.\\\hline
\textbackslash MediumHighOverline & Applies ss05, Medium-high Overline.\\\hline
\textbackslash EnlargedMinuscules & Applies ss06, Enlarged minuscules.\\\hline
\textbackslash Underdotted & Applies ss07, Underdotted.\\\hline
\textbackslash ContextualLongS & Applies ss08, Contextual long s. This should be used only in English or
French text. With other languages it simply converts all instances of s to ſ.\\\hline
\textbackslash AlternateFigures & Applies ss09, Alternate Figures.\\\hline
\textbackslash EntitiesAndTags & Applies ss10, Entities and Tags.\\\hline
\textbackslash EarlyEnglishFuthorc & Applies ss12, Early English Futhorc.\\\hline
\textbackslash ElderFuthark & Applies ss13, Elder Futhark.\\\hline
\textbackslash YoungerFuthark & Applies ss14, Younger Futhark.\\\hline
\textbackslash LongBranchToShortTwig & Applies ss15, Long Branch to Short Twig.\\\hline
\textbackslash ContextualRRotunda & Applies ss16, Contextual r rotunda.\\\hline
\textbackslash RareDigraphs & Applies ss17, Rare Digraphs.\\\hline
\textbackslash OldStylePunctuation & Applies ss18, Old-style Punctuation.\\\hline
\textbackslash LatinToGothic & Applies ss19, Latin to Gothic.\\\hline
\textbackslash LowDiacritics & Applies ss20, Low Diacritics.\\\hline
\textbackslash jcv, \textbackslash textcv & Applies any Character Variant feature (see below).\\
\end{supertabular}
\end{center}

\noindent The syntax of \textbackslash jcv
is {\small\verb|\jcv[num]{num}|}, where the second (required) argument is the number of the Character Variant feature,
and the first (optional) argument is an index into the variants provided by that feature (starting with zero, the default).
\textbackslash textcv takes an additional required argument ({\small\verb|\textcv[num]{num}{text}|}—the text to which the
feature should be applied.

Character Variant features can also be selected with mnemonics, listed below. For example, a feature for
lowercase \textbf{a} can be expressed as {\small\verb|\textcv[2]{\jcva}{a}|}, yielding \textbf{\textcv[2]{\jcva}{a}}.

\begin{multicols}{3}
\jCond\textbackslash jcvA

\textbackslash jcva

\textbackslash jcvB

\textbackslash jcvb

\textbackslash jcvC

\textbackslash jcvc

\textbackslash jcvD

\textbackslash jcvd

\textbackslash jcvE

\textbackslash jcve

\textbackslash jcvF

\textbackslash jcvf

\textbackslash jcvG

\textbackslash jcvg

\textbackslash jcvH

\textbackslash jcvh

\textbackslash jcvI

\textbackslash jcvi

\textbackslash jcvJ

\textbackslash jcvj

\textbackslash jcvK

\textbackslash jcvk

\textbackslash jcvL

\textbackslash jcvl

\textbackslash jcvM

\textbackslash jcvm

\textbackslash jcvN

\textbackslash jcvn

\textbackslash jcvO

\textbackslash jcvo

\textbackslash jcvP

\textbackslash jcvp

\textbackslash jcvQ

\textbackslash jcvq

\textbackslash jcvR

\textbackslash jcvr

\textbackslash jcvS

\textbackslash jcvs

\textbackslash jcvT

\textbackslash jcvt

\textbackslash jcvU

\textbackslash jcvu

\textbackslash jcvV

\textbackslash jcvv

\textbackslash jcvW

\textbackslash jcvw

\textbackslash jcvX

\textbackslash jcvx

\textbackslash jcvY

\textbackslash jcvy

\textbackslash jcvZ

\textbackslash jcvz

\textbackslash jcvaa

\textbackslash jcvAE

\textbackslash jcvae

\textbackslash jcvAO

\textbackslash jcvao

\textbackslash jcvAogonek

\textbackslash jcvaogonek

\textbackslash jcvASCIItilde

\textbackslash jcvasterisk

\textbackslash jcvav

\textbackslash jcvbrevebelow

\textbackslash jcvcombiningdieresis

\textbackslash jcvcombiningdoublemacron

\textbackslash jcvcombininginsulard

\textbackslash jcvcombiningopena

\textbackslash jcvcombiningoverline

\textbackslash jcvcombiningrrotunda

\textbackslash jcvcombiningzigzag

\textbackslash jcvcomma

\textbackslash jcvcurrency

\textbackslash jcvdbar

\textbackslash jcvdcroat

\textbackslash jcvEng

\textbackslash jcvEogonek

\textbackslash jcvetabbrev

\textbackslash jcvexclam

\textbackslash jcvflorin

\textbackslash jcvGermanpenny

\textbackslash jcvglottal

\textbackslash jcvlb

\textbackslash jcvlhighstroke %somehow escaped the documentation

\textbackslash jcvmacron

\textbackslash jcvmiddot

\textbackslash jcvoPolish

\textbackslash jcvounce

\textbackslash jcvperiod

\textbackslash jcvpunctuselevatus

\textbackslash jcvquestion

\textbackslash jcvrum

\textbackslash jcvsemicolon

\textbackslash jcvslash

\textbackslash jcvspacingusabbrev

\textbackslash jcvspacingzigzag

\textbackslash jcvsterling

\textbackslash jcvthorncrossed

\textbackslash jcvTironianEt

\textbackslash jcvYogh
\end{multicols}

\end{document}