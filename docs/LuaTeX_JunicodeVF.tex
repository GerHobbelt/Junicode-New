\documentclass[11pt]{article}

\usepackage{fontspec}
\usepackage{realscripts}
\usepackage{relsize}
\usepackage{microtype}

\setmainfont{JunicodeTwoBetaVF-Roman.otf}[
  ItalicFont = JunicodeTwoBetaVF-Italic.otf,
  BoldFont = JunicodeTwoBetaVF-Roman.otf,
  BoldItalicFont =  JunicodeTwoBetaVF-Italic.otf,
  Numbers={Lowercase,Proportional},
  UprightFeatures={
    SizeFeatures={
      {Size={-8.5},      RawFeature={axis={wght=550,wdth=115}}},
      {Size={8.6-10.99}, RawFeature={axis={wght=475,wdth=110}}},
      {Size={11-21.59},  RawFeature={axis={wght=400,wdth=100}}},
      {Size={21.6-},     RawFeature={axis={wght=300,wdth=95}}}
    },
  },
  ItalicFeatures={
    SizeFeatures={
      {Size={-8.5},      RawFeature={axis={wght=550,wdth=112}}},
      {Size={8.6-10.99}, RawFeature={axis={wght=475,wdth=106}}},
      {Size={11-21.59},  RawFeature={axis={wght=400,wdth=100}}},
      {Size={21.6-},     RawFeature={axis={wght=300,wdth=95}}}
    },
  },
  BoldFeatures={
    SizeFeatures={
      {Size={-8.5},      RawFeature={axis={wght=700,wdth=118}}},
      {Size={8.6-10.99}, RawFeature={axis={wght=700,wdth=112}}},
      {Size={11-21.59},  RawFeature={axis={wght=650,wdth=100}}},
      {Size={21.6-},     RawFeature={axis={wght=600,wdth=95}}}
    },
  },
  BoldItalicFeatures={
    SizeFeatures={
      {Size={-8.5},      RawFeature={axis={wght=700,wdth=112}}},
      {Size={8.6-10.99}, RawFeature={axis={wght=700,wdth=106}}},
      {Size={11-21.59},  RawFeature={axis={wght=650,wdth=100}}},
      {Size={21.6-},     RawFeature={axis={wght=600,wdth=95}}}
    },
  },
  ]

  \setfontface\mediumupright{JunicodeTwoBetaVF-Roman.otf}[
    SizeFeatures={
      {Size={-8.5},      RawFeature={axis={wght=650,wdth=115}}},
      {Size={8.5-11}, RawFeature={axis={wght=575,wdth=110}}},
      {Size={11-22},  RawFeature={axis={wght=500,wdth=100}}},
      {Size={22-},     RawFeature={axis={wght=400,wdth=95}}}
    },
  ]
  \setfontface\light{JunicodeTwoBetaVF-Roman.otf}[
    SizeFeatures={
      {Size={-8.5},      RawFeature={axis={wght=400,wdth=115}}},
      {Size={8.5-11}, RawFeature={axis={wght=350,wdth=110}}},
      {Size={11-22},  RawFeature={axis={wght=300,wdth=100}}},
      {Size={22-},     RawFeature={axis={wght=300,wdth=95}}}
    },
  ]
  \setfontface\condensed{JunicodeTwoBetaVF-Roman.otf}[
    SizeFeatures={
      {Size={-8.5},      RawFeature={axis={wght=450,wdth=85}}},
      {Size={8.5-11}, RawFeature={axis={wght=400,wdth=80}}},
      {Size={11-22},  RawFeature={axis={wght=350,wdth=75}}},
      {Size={22-},     RawFeature={axis={wght=300,wdth=75}}}
    },
  ]
  \setfontface\expanded{JunicodeTwoBetaVF-Roman.otf}[
    SizeFeatures={
      {Size={-8.5},      RawFeature={axis={wght=575,wdth=125}}},
      {Size={8.5-11}, RawFeature={axis={wght=500,wdth=125}}},
      {Size={11-22},  RawFeature={axis={wght=425,wdth=120}}},
      {Size={22-},     RawFeature={axis={wght=325,wdth=120}}}
    },
  ]
  \setfontface\condensedbold{JunicodeTwoBetaVF-Roman.otf}[
    SizeFeatures={
      {Size={-8.5},      RawFeature={axis={wght=700,wdth=85}}},
      {Size={8.5-11},    RawFeature={axis={wght=700,wdth=80}}},
      {Size={11-22},     RawFeature={axis={wght=700,wdth=75}}},
      {Size={22-},       RawFeature={axis={wght=700,wdth=75}}}
    },
  ]
  \setfontface\expandedbold{JunicodeTwoBetaVF-Roman.otf}[
    SizeFeatures={
      {Size={-8.5},      RawFeature={axis={wght=700,wdth=125}}},
      {Size={8.5-11},    RawFeature={axis={wght=700,wdth=125}}},
      {Size={11-22},     RawFeature={axis={wght=700,wdth=120}}},
      {Size={22-},       RawFeature={axis={wght=700,wdth=120}}}
    },
  ]
  \setmonofont{SourceCodeVariable-Roman.otf}[
    Scale = MatchLowercase,
    Numbers = Lowercase,
    SizeFeatures={
      {Size={-8},        RawFeature={axis={wght=500}}},
      {Size={8-11},      RawFeature={axis={wght=450}}},
      {Size={11-},       RawFeature={axis={wght=400}}}
    }
  ]

  \newcommand{\cvd}[3][0]{{\addfontfeature{CharacterVariant=#2:#1}#3}}
  \newcommand{\sups}[1]{{\addfontfeature{VerticalPosition=Superior}#1}}
  \newcommand{\enla}[1]{{\addfontfeature{StylisticSet=6}#1}}
  \newcommand{\oldch}[1]{{\addfontfeature{CharacterVariant={38:2,69:1}}#1}}
  \newcommand{\ltech}{Lua\kern-1.5pt\TeX}

  \setlength{\footnotesep}{10pt}
  \setlength{\skip\footins}{18pt}
  \setlength{\parindent}{0pt}
  \renewcommand{\baselinestretch}{1.15}

  \tolerance=2000
  \frenchspacing


  \title{Junicode VF and Lua\kern-2pt\TeX}
  \author{Peter S. Baker}
  \date{}

  \newcommand{\textsample}{\oldch{\textit{Recti diligunt te. In
  canticis sponsa ad sponsum. Est rectum grammaticum. Rectum
  geomet\sups{i}cum. rectum theologicum. ⁊ sunt differencie totidem reguloꝝ.
  De recto theologico sermo nobis est.} Lauerd seið godes spuse to hire deoꝛeƿurðe
  spus. þeo richte luuieð þé. \enla{þ}eo beoð richte. þe liuieð efter riƿle. ⁊
  ȝe mine leoue sustren habbeð moni dei icraued me efter riƿle. Moni cunne riƿlen
  beoð. \enla{a}ch twa beoð bemong alle ꝥ ich ƿille speoken of þurch oƿer bone ⁊
  mid godes grace}.}

\begin{document}

\addfontfeature{Language=English,Contextuals=Alternate,Numbers={Lowercase,Proportional}}

\maketitle

\section*{Introduction}

\ltech{} now supports variable fonts. Support for TrueType-flavored
variable fonts is buggy (as of luaotfload version 3.18); until these bugs
are fixed (reportedly in the next release), users should prefer the Compact
File Format flavor of
Junicode VF (with extension \texttt{.otf} in the \slash fonts\slash variable-cff2 folder).\\

There are at least two reasons for \ltech{} users to prefer variable over
conventional fonts:
\begin{itemize}
\item With Junicode VF, you can precisely control the weight and width of letters
so as to achieve an even color.\footnote{\ “Color” is the overall ratio of black
and white in a block of text, ideally giving the impression of an even gray.}
With a conventional font, text set in a smaller size (like block quotations or
footnotes) may appear lighter than the main text. Users of Junicode VF can set small text
in a heavier weight (without going all the way to Bold) for evenness of color, and wider for better
legibility.\footnote{\ For example, these footnotes should harmonize well
with the main text: that is, the block of footnotes should be approximately the
same shade of gray as the main text.}
\item Second, if you prefer,
for example, weights that are a little different from the ordinary (e.g.
{\mediumupright “Medium”} or {\light “Light”}
instead of “Regular” for the main text),\footnote{\ Here's another footnote, just
to expand the footnote block. Lorem ipsum dolor sit amet, consectetur adipiscing elit,
sed do eiusmod tempor incididunt ut labore et dolore magna aliqua. \textit{Ut enim ad
minim veniam, quis nostrud exercitation ullamco laboris nisi ut aliquip ex ea
commodo consequat}. Duis aute irure dolor in reprehenderit in voluptate velit.}
%esse cillum dolore eu fugiat nulla pariatur.}
you can do that very easily. Likewise, you can choose a width anywhere from
{\condensed “Condensed”} to {\expanded “Expanded,”} adjusting for size as necessary.
In short, you have discretion to style Junicode VF in numerous ways, and one
document set in Junicode VF can look very different from another.
\end{itemize}

\section*{Axes}

Junicode VF has three axes, two of which are useful to \ltech{} users.
\begin{description}
\item[Weight] The tag is \texttt{wght}.\footnote{\ Data structures and features
of modern fonts are identified by four-letter tags. You will need these when
you set up your document.} This axis corresponds to the
distinction between Regular and Bold in a traditional font family; extended
font families may also include weights called Light, Medium, and Semibold,
to name a few. In Junicode VF, Weight is continuously variable from 300
({\light Light}) to 700 (\textbf{Bold}).
\item[Width] The tag is \texttt{wdth}. Some conventional fonts come in more
than one width. In Junicode VF, the width is continuously variable, from
{\condensed “Condensed”} (75) to {\expanded “Expanded”} (125). Note that the
weight of strokes remains the same as Width increases or decreases,%
\footnote{\ For example, the width of the three vertical strokes in the letter
\textbf{m} is the same whether it is set Condensed ({\condensedbold m})
or Expanded ({\expandedbold m}).} with the result that the ratio of black to
white decreases as width increases, and the text becomes a lighter gray. If you
set different widths together you may want to slightly decrease the weight of
Condensed text and increase the weight of Expanded text so as to compensate.
\item[Enlarge] The tag is \texttt{ENLA}. This is a custom axis, intended for a
particular purpose. In medieval manuscripts, sentences often begin with
“enlarged minuscules,” that is, letters that are lowercase in shape but
written larger than other lowercase letters. Junicode VF's Enlarge axis
provides a way to represent these “enlarged minuscules” in an accessible and
flexible way. Values run from 0 to 100, with 0 matching other lowercase letters
and 100 at the same height and weight as uppercase letters. Unfortunately,
luaotfload ignores this axis (perhaps it ignores \textit{all} custom axes), and so it
cannot be used in \ltech{} documents. Users should use Stylistic Set 6 (“Enlarged
Minuscules”) instead. The stylistic set does not provide the same flexibility
as the Enlarge axis, but it is an accessible solution and will suffice for most purposes.
\end{description}

These axes can be combined in any way you like, but some combinations
will work better than others. In particular, combining two extremes (say,
Condensed and Bold) may occasionally produce distorted letter-shapes.

\section*{Selecting Junicode VF with fontspec}

Fontspec provides a number of ways to select a font. The commands used
to set up the fonts for this document
are \verb|\setmainfont| and \verb|\setfontface|. First, to select Junicode
VF as the main font with automatic adjustment of weight and width to
provide for even text color:\footnote{\ This code is adapted from a concept
by Andrew Dunning; see https://github.com\slash psb1558\slash Junicode-font/discussions/92.}

\scriptsize\begin{verbatim}
\setmainfont{JunicodeTwoBetaVF-Roman}[
  Extension      = .otf,
  ItalicFont     = JunicodeTwoBetaVF-Italic,
  BoldFont       = JunicodeTwoBetaVF-Roman,
  BoldItalicFont = JunicodeTwoBetaVF-Italic,
  UprightFeatures={
    SizeFeatures={
      {Size={-8.5},      RawFeature={axis={wght=550,wdth=115}}},
      {Size={8.6-10.99}, RawFeature={axis={wght=475,wdth=110}}},
      {Size={11-21.59},  RawFeature={axis={wght=400,wdth=100}}},
      {Size={21.6-},     RawFeature={axis={wght=300,wdth=95}}}
    },
  },
  ItalicFeatures={
    SizeFeatures={
      {Size={-8.5},      RawFeature={axis={wght=550,wdth=112}}},
      {Size={8.6-10.99}, RawFeature={axis={wght=475,wdth=106}}},
      {Size={11-21.59},  RawFeature={axis={wght=400,wdth=100}}},
      {Size={21.6-},     RawFeature={axis={wght=300,wdth=95}}}
    },
  },
  BoldFeatures={
    SizeFeatures={
      {Size={-8.5},      RawFeature={axis={wght=700,wdth=118}}},
      {Size={8.6-10.99}, RawFeature={axis={wght=700,wdth=112}}},
      {Size={11-21.59},  RawFeature={axis={wght=650,wdth=100}}},
      {Size={21.6-},     RawFeature={axis={wght=600,wdth=95}}}
    },
  },
  BoldItalicFeatures={
    SizeFeatures={
      {Size={-8.5},      RawFeature={axis={wght=700,wdth=112}}},
      {Size={8.6-10.99}, RawFeature={axis={wght=700,wdth=106}}},
      {Size={11-21.59},  RawFeature={axis={wght=650,wdth=100}}},
      {Size={21.6-},     RawFeature={axis={wght=600,wdth=95}}}
    },
  },
]
\end{verbatim}

\normalsize Because fontspec does not (yet) support variable font axes, it is
necessary to use \texttt{RawFeature} to set these up. Axes can be omitted
from these declarations when you're using the font's default values
(\texttt{wght} 400, \texttt{wdth} 100, \texttt{ENLA} 0),
though it is perfectly all right to include these for clarity.\\

If you need other variants of Junicode VF, use \verb|\setfontface| to select
them. For example, this document has a few words in Junicode's
{\light Light} weight, so the preamble has this font definition:

\scriptsize\begin{verbatim}
\setfontface\light{JunicodeTwoBetaVF-Roman.otf}[
  SizeFeatures={
    {Size={-8.5},   RawFeature={axis={wght=400,wdth=115}}},
    {Size={8.5-11}, RawFeature={axis={wght=350,wdth=110}}},
    {Size={11-22},  RawFeature={axis={wght=300,wdth=100}}},
    {Size={22-},    RawFeature={axis={wght=300,wdth=95}}}
  },
]
\end{verbatim}

\normalsize Here the \verb|SizeFeatures| command looks exactly like the ones in
\verb|\setmainfont|, except for the different \texttt{wght} values, so the \verb|\light|
command can be created by copying and pasting and just a little editing.

\section*{Specimens}

\noindent\tiny\textbf{tiny}: \textsample\\

\noindent\scriptsize\textbf{scriptsize}: \textsample\\

\noindent\footnotesize\textbf{footnotesize}: \textsample\\

\noindent\small\textbf{small}: \textsample\\

\noindent\normalsize\textbf{normalsize}: \textsample\\

\noindent\large\textbf{large}: \textsample\\

\noindent\Large\textbf{Large}: \textsample\\

\noindent\LARGE\textbf{LARGE}: \textsample\\

\noindent\huge\textbf{huge}: \textsample\\

\Huge\textbf{Huge}: \textit{Recti diligunt te.} Lauerd seið godes spuse
to hire deoꝛeƿurðe spus.

\end{document}
