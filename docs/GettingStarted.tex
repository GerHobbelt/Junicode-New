
\chapter{Getting Started with Junicode}\hypertarget{GettingStarted}{}
%\fancyhead[CE]{\scshape\color{myRed} {\addfontfeatures{Numbers=OldStyle}\thepage}\hspace{10pt}getting started}

Junicode comes in two flavors—static and variable. Static fonts are the ones
users are most familiar with, each font file supplying a single set of outlines
that do not change except in size. By contrast, a single
\href{https://fonts.google.com/knowledge/introducing_type/introducing_variable_fonts}{variable font}
file stores a set of
outlines that can morph in various ways—for example, becoming bolder or lighter,
narrower or wider, and sometimes undergoing more complex transformations. The static
version of Junicode consists of thirty-eight font files,
each providing a distinct variation of the font’s style; the variable version consists
of only two (one each for roman and italic), but those two font files are capable of
much more than the static version's thirty-eight.

Because the static and variable versions of Junicode are differently named
(“Junicode” and “Junicode VF”), both can be installed on the same system. However,
you should choose one or the other for any particular project. Choose the static
version if the application you are using does not yet support variable fonts. Such
applications include Microsoft Word, Apple Pages, Quark Xpress, Google Docs,
Affinity Publisher, and most flavors of {\TeX} (except for {\LuaTeX}—see below). Another
reason to choose the static version is its familiarity: if you don't need the
advanced capabilities of the variable version, it is perfectly all right to stick
with what you know.

All major web browsers (including browsers for mobile devices) support variable fonts,
and there are good reasons to choose
the variable version of Junicode for any web project. The greatest reason to go
with the variable version is to speed the loading of web pages: users will never
have to download more than two font files (the size of which can be radically
reduced via subsetting, explained in Section 9 of this Manual). Additionally,
however, variable fonts can make a page of text more dynamic and visually
interesting. See Mozilla's
\href{https://developer.mozilla.org/en-US/docs/Web/CSS/CSS_fonts/Variable_fonts_guide}%
{Variable Fonts Guide} for more information about using variable fonts on the web.

A growing number of desktop applications support variable fonts. Use the variable version of
Junicode in Adobe InDesign (always with the World-Ready Paragraph Composer).\footnote{%
The choice of a Composer is well hidden in the “Justification” section of the
“Paragraph Style Options” dialog. Use of the default “Adobe Paragraph Composer”
with Junicode VF may cause InDesign to crash or otherwise misbehave. To prevent crashes
when using the variable version of Junicode,
it is also advisable to delete InDesign's preferences when launching the program after
a font upgrade. To do this, press Shift-Alt-Control (on Windows) or
Shift-Control-Option-Command (on the Mac) all together, \textit{immediately} after
clicking to launch InDesign.} LibreOffice
has supported variable fonts since version 7.5 (2023). {\LuaTeX} has excellent support
for variable fonts: make sure your {\TeX} installation is up to date (since in
recent years support for variable fonts has improved with every release), and always choose “harf”
mode in your font-selection code. For an example of font-selection code for a
variable font, see the file
\href{https://github.com/psb1558/Junicode-font/blob/master/docs/JunicodeManual.sty}%
{JunicodeManual.sty} (part of the source for this manual) in the “docs” directory of the GitHub Junicode site.
You are welcome to copy and modify this code. A number of graphical design apps 
also support variable fonts, including Adobe Illustrator, PhotoShop, Figma,
Sketch, and CorelDRAW.

The static version of Junicode has five weights and five widths, which are combined in many ways
for a total of nineteen styles in
both roman and italic.\footnote{Several of the twenty-five possible combinations
(e.g. {\jExpLight light expanded})
have been omitted as unlikely to be useful; however, these can be accessed via the variable font.}
It is not necessary to install all of these; in fact,
your life will be simplified (font menus easier to navigate) if you
make a selection. You will probably want the traditional Regular, Bold, Italic, and Bold
Italic fonts, but you should survey the styles displayed in the Specimen
section of this booklet, choose the ones that look best to you, and install
only those. A reasonable selection for many users will include the traditional four
styles for main text, several SemiExpanded styles for footnotes, and
SemiCondensed for titles.

Junicode’s static fonts come in two types, TrueType (files with a .ttf extension) and
CFF (files with an .otf extension). These are functionally identical, but they may look
subtly different on your computer’s screen because of the different technologies used to
render glyphs. Choose the one you find most appealing.

With around 5,000 characters, Junicode is a large font. Finding the things you
want in a collection that size can be a challenge, and then entering them in your
documents is another challenge. This document will help, but it
presupposes a certain amount of knowledge—for example, how to install a font in
Windows, Mac OS or Linux and how to install and use different kinds of software.

Medievalists will find the \href{https://bora.uib.no/bora-xmlui/handle/1956/10699}%
{\textit{MUFI Character Recommendation}}, version 4.0 (2015)
an essential supplement to this document. The \textit{Recommendation} lists
thousands of characters identified by the
Medieval Unicode Font Initiative as being of interest to medievalists. Junicode
contains all of these characters. There are two versions of the \textit{Recommendation}:
you will probably find the “Alphabetical Order” version most helpful.

From the \textit{MUFI Character Recommendation} and Chapter 10 (“Encoded Glyphs in
Junicode”) of this manual, you can find out
the \textbf{code points}\footnote{\ A Unicode code point is a numerical identifier for a character.
It is generally expressed as a
four-digit hexadecimal (or base-16) number with a prefix of ``U+''. The letter
capital ``A,'' for example, is \unic{U+0041} (65
in decimal notation), and lowercase ``ȝ'' (Middle English yogh) is \unic{U+021D}.}
of the characters you need. These code points can be used to enter
characters in your documents when they cannot be typed on the keyboard.

To enter any Unicode character in a Windows application, type its four-digit
code, followed by Alt-X. To do the same in the Mac OS, first install and switch
to the “Unicode Hex Input” keyboard, then type the code while holding down the Option
key. In most Linux distributions you can enter a code by typing Shift-Control-U,
then the code followed by Return or Enter.

\textbf{Combining marks} (diacritics and certain abbreviation signs) can pose special problems for
medievalists. Unicode contains a great
many \textbf{precomposed characters} consisting of a base letter plus one or more marks.
If these are all you need you're fortunate---especially if they can be
typed on an international keyboard (not all can).\footnote{\ Both
Windows and the Mac OS come with international keyboards that make it easy to
type special letters and diacritics. To find out how to enable these, search
online for “Mac OS International Keyboard” or “Windows International Keyboard.”}
But medieval manuscripts frequently contain
combinations of base + mark that are not used in modern written languages.
For these, you'll have to
enter bases and marks separately.

To position a mark correctly over a base character, first enter the base,
followed by the mark or marks.
The sequence \textbf{m} + \textbf{◌ᷙ} (\unic{U+1DD9})
will make \textex{mᷙ};
\mbox{\textbf{y} + \textbf{◌̄}} (\unic{U+0304}) + \textbf{◌̆} (\unic{U+0306}) will make \textex{ȳ̆};
\textbf{e} + \textbf{◌̣} (\unic{U+0323}) + \textbf{◌ᷠ} (\unic{U+1DE0}) will make \textex{ẹᷠ}.

More than sixteen hundred characters in Junicode can only be accessed via OpenType features—that is,
by way of the programming built into the font—and many others \textit{should} be
accessed that way for reasons explained in the Introduction
to the Feature Reference section of this document.

For example, some programs (including Microsoft Word) produce small caps by
scaling capitals down to approximately the height of lowercase letters.
These always look too thin and light.
But Junicode contains hundreds of \textsc{true small caps} designed to harmonize with
the surrounding text. These can only be accessed via the OpenType \textSourceText{smcp} feature,
which you can apply to a run of text much as you apply italic or bold styles:
select some text and then apply the feature.

Unfortunately, not every program supports OpenType features, and some that do
either support only a few or make them difficult to access. Programs
that support Junicode’s features fully include the free word processor
\href{https://www.libreoffice.org/}{LibreOffice Writer}, all major browsers
(Firefox, Chrome, Safari and Edge), and
the typesetting programs {\LuaLaTeX} and {\XeLaTeX}. Adobe InDesign supports
OpenType features only partially, but all features can be accessed via
Roland Dreger’s \href{https://github.com/RolandDreger/open-type-features}{open-type-features}
script, and InDesign provides access to
all of Junicode's characters via its “Glyphs” dialog.

Microsoft Word, unfortunately, provides only limited support for OpenType
features. It supports the \hyperlink{req}{Required Features} discussed below, and also
variant number forms and Stylistic Sets (though only one at a time). Many characters
(for example, \textsc{true small caps} and those accessible only via Character
Variant features) cannot be accessed at all. To activate Word's OpenType
support, you must open the “Font” dialog, click over to the “Advanced” tab,
and check the “Kerning” box. (Oddly, the “Kerning” box enables all other
OpenType features.) Then, in the same tab, select Standard Ligatures, Contextual
Alternates, and any other features you want.
OpenType features are best applied to character styles rather than
directly to text: this will
save you from having to perform this operation repeatedly.

It is also good to set the language properly for the text you're working on.
Programs like Word will automatically set the language to the default for your system. If you
change to a language other than your own for a passage (or even a single word),
you should set the language for that passage appropriately. This will unlock
a number of capabilities. For example, in Old and Middle English, Word and
other programs will use the English form of thorn and eth ({\addfontfeature{Language=English} þð}) instead of
the modern Icelandic ({\addfontfeature{Language=Icelandic} þð}), and in ancient
Greek you will be able to type accents after vowels instead of looking up
the codes for hundreds of polytonic vowel + accent combinations. But these and other capabilities
are only available when you set the language correctly.

In LibreOffice and InDesign you can set the language with a drop-down menu
in the “Character” dialog. In Word there is a separate “Language” dialog,
accessible from the “Tools” menu.

If you have questions about any aspect of Junicode,
post a query in the \href{https://github.com/psb1558/Junicode-font/discussions}%
{Junicode discussion forum}. If you notice a bug, or wish to
request an enhancement or other change, please open an
\href{https://github.com/psb1558/Junicode-font/issues}{issue} at the font's
\href{https://github.com/psb1558/Junicode-font}{development site}.
