%&program=xelatex
%&encoding=UTF-8 Unicode

\documentclass[12pt,a4paper,openany]{book}

\usepackage{fontspec}

\usepackage{microtype}

\setmainfont[Contextuals=Alternate]{Junicode}
%\setmainfont[Contextuals=Alternate]{Junicode.ttf}[
%  Path = /home/peter/.local/share/fonts/ ,
%  BoldFont = Junicode-Bold.ttf ,
%  ItalicFont = Junicode-Italic.ttf ,
%  BoldItalicFont = Junicode-BoldItalic.ttf ]
\newfontfamily\greekfont{FoulisGreek}

\newICUfeature{StyleSet}{insular}{+ss02}
\newICUfeature{StyleSet}{highline}{+ss04}
\newICUfeature{StyleSet}{medline}{+ss05}
\newICUfeature{StyleSet}{enlarged}{+ss06}
\newICUfeature{StyleSet}{underdot}{+ss07}
\newICUfeature{StyleSet}{altyogh}{+ss08}
\newICUfeature{StyleSet}{altshapes}{+ss09}
\newICUfeature{StyleSet}{longslash}{+ss10}
\newICUfeature{StyleSet}{altae}{+ss11}
\newICUfeature{StyleSet}{altogonek}{+ss15}
\newICUfeature{StyleSet}{oldpunct}{+ss18}
\newICUfeature{StyleSet}{gothic}{+ss19}
\newICUfeature{StyleSet}{gothtolat}{+ss20}
\newICUfeature{MirrorRunes}{on}{+rtlm}
\newICUfeature{IPAMode}{on}{+mgrk,-liga}
\newICUfeature{Fractions}{on}{+frac}
\newICUfeature{Superscripts}{on}{+sups}
\newICUfeature{Subscripts}{on}{+subs}
\newcommand{\salt}[1]{{\addfontfeatures{Alternate=0}{#1}}}
\newcommand{\saltb}[1]{{\addfontfeatures{Alternate=1}{#1}}}
\usepackage{color}
\definecolor{titlblue}{rgb}{0.34,0.33,0.63}
\definecolor{titlred}{rgb}{0.75,0.29,0.31}
\definecolor{titlbrown}{rgb}{0.41,0.34,0.30}
\definecolor{myRed}{rgb}{0.5,0,0}
\definecolor{myPink}{rgb}{1.0,0.7,0.7}
\definecolor{myBlue}{rgb}{0,0,0.5}
\definecolor{myLightBlue}{rgb}{0.7,0.7,1.0}
\definecolor{myGreen}{rgb}{0,0.5,0}
\definecolor{myMaroon}{rgb}{0.35,0,0.5}
\usepackage{fancyhdr}
\pagestyle{fancy}
\fancyfoot{}
\renewcommand{\headrulewidth}{0pt}
%\newcommand{\sampletext}{Successit autem uiro Domini Cudbercto in exercenda uita solitaria, quam in insula Farne ante episcopatus sui. 12345 \addfontfeatures{Numbers=OldStyle}12345}
%\newcommand{\sampletext}{Anno autem regni Aldfridi tertio, Caedualla, rex Occidentalium Saxonum, cum genti suae duobus annis strenuissime praeesset}
\newcommand{\sampletext}{Opima frugibus atque arboribus insula, et alendis apta pecoribus ac iumentis; uineas etiam qui\-bus\-dam in locis germinans}
\newcommand{\sctext}{Cum multa divinitus, pontifices, a
ma\-ioribus nos\-tris in\-venta atque in\-sti\-tuta sunt}

\linespread{1.1}
\frenchspacing
\setlength{\parskip}{0ex plus0ex minus0ex}
\tolerance=1000

\begin{document}
\begin{titlepage}
\huge\noindent
{\color{myRed}}\\[5cm]
\Huge \hfill {\color{myBlue}Junicode}\hfill \\[1cm]
\huge \hfill the font for medievalists\hfill \\[1cm]
 \Huge\hfill {\color{myRed}}\hfill \\[1cm]
 \huge\hfill {\itshape specimens and user’s guide}\hfill \\
\vfill
{\color{myRed}}
\end{titlepage}
\mainmatter
\fancyhead[CE]{\scshape\color{myRed} {\addfontfeatures{Numbers=OldStyle}\thepage}\hspace{10pt}junicode}
\fancyhead[CO]{\scshape\color{myRed} {junicode}\hspace{10pt}{\addfontfeatures{Numbers=OldStyle}\thepage}}
\chapter*{\color{myBlue}Junicode}
\large

\noindent The design of Junicode is based on scans of George Hickes,
          {\itshape Linguarum vett. septentrionalium thesaurus
            grammatico-criticus et archaeologicus} (Oxford, 1703–5). This massive two-volume folio is a fine
          example of the work of the Oxford University Press at this
          period: printed in multiple types (for every language had to
          have its proper type) and lavishly
          illustrated with engravings of manuscript pages, coins and
          artifacts.

The type used for Hickes’s {\itshape Thesaurus} resembles those assembled
by John Fell (1625–86) and bequeathed by him to the University of
Oxford. To my eye, however, it looks more like the “Pica Roman”
purchased by the University in 1692 than like any of Fell’s. For printing in Old English, this type was
supplemented by the “Pica Saxon” commissioned by the early
Anglo-Saxonist Franciscus Junius (1591–1677) and bequeathed by him to
the University. Specimens of both can be found in {\itshape A Specimen of the
Several Sorts of Letter Given to the University by Dr. John Fell,
Sometime Lord Bishop of Oxford. To Which Is Added the Letter Given by
Mr. F. Junius} (Oxford, 1693). Junius’s Pica Saxon
was mixed freely with Pica Roman in printing the {\itshape Thesaurus}.

The Foulis Greek font has a different origin from Junicode’s Latin
(though it harmonizes well), being based on type designed by Alexander
Wilson (1714–86) of Glasgow and used in numerous books published by
the Foulis Press, most notably the great Glasgow Homer.

\begin{center}
\Huge\color{myRed}
\end{center}

\chapter*{\color{myBlue}Specimens}

\noindent {\tiny \sampletext} {\small \sampletext} {\large \sampletext}
{\Large \sampletext} {\LARGE \sampletext} {\huge \sampletext}\\

{\itshape\noindent {\tiny \sampletext} {\small \sampletext} {\large \sampletext}
{\Large \sampletext} {\LARGE \sampletext} {\huge \sampletext}}\\

{\bfseries\noindent {\tiny \sampletext} {\small \sampletext} {\large \sampletext}
{\Large \sampletext} {\LARGE \sampletext} {\huge \sampletext}}\\

{\bfseries\itshape\noindent {\tiny \sampletext} {\small \sampletext} {\large \sampletext}
{\Large \sampletext} {\LARGE \sampletext} {\huge \sampletext}}\\

\noindent {\scshape {\tiny \sctext} {\small \sctext} {\large \sctext}
{\Large \sctext} {\LARGE \sctext}}\\

\noindent\textit{\scshape {\tiny \sctext} {\small \sctext} {\large \sctext}
  {\Large \sctext} {\LARGE \sctext}}\\

\noindent {\scshape\bfseries {\tiny \sctext} {\small \sctext} {\large \sctext}
{\Large \sctext} {\LARGE \sctext}}

\subsection*{Old and Middle English}

\noindent Wē æthrynon mid ūrum ārum þā ȳðan þæs dēopan wǣles; wē
ġesāwon ēac þā muntas ymbe þǣre sealtan sǣ strande, and wē mid
āðēnedum hræġle and ġesundfullum windum þǣr ġewīcodon on þām
ġemǣrum þǣre fæġerestan þēode. Þā ȳðan ġetācniað þisne dēopan
cræft, and þā muntas ġetācniað ēac þā miċelnyssa þisses cræftes.\\

\noindent S{\scshape iþen} þe sege and þe assaut watz sesed at Troye,\\
Þe borȝ brittened and brent to brondez and askez,\\
Þe tulk þat þe trammes of tresoun þer wroȝt\\
Watz tried for his tricherie, þe trewest on erthe:\\
Hit watz Ennias þe athel, and his highe kynde,\\
Þat siþen depreced prouinces, and patrounes bicome\\
Welneȝe of al þe wele in þe west iles.\\

\noindent{\small\itshape Apply the OpenType feature ss02 (Stylistic Set 2)
for insular letter-forms.}\\[1ex]
{\addfontfeature{StyleSet=insular,Ligatures=NoCommon,StyleSet=altogonek}
Her cynewulf benam sigebryht his rices \& weſtſeaxna wiotan for
un\-ryht\-um dędū buton hamtúnſcire \& he hæfde þa oþ he ofslog
þone aldormon þe hī lengeſt wunode \& hiene þa cynewulf on
andred adræfde \& ħ þær wunade oþ þæt hine án ſwán ofſtang
æt pryfetesflodan \& he wręc þone aldormon cumbran \& se cynewulf
oft miclum gefeohtum feaht uuiþ bretwalū.}



\subsection*{Old Icelandic}
{\small\itshape For Nordic shapes of þ and ð, specify the Icelandic
language, if your application has good language support; or apply the OpenType
ss01 (Stylistic Set 1) feature.}\\[1ex]
{\addfontfeature{Language=Icelandic}Um haustit sendi Mǫrðr Valgarðsson orð at Gunnarr myndi vera einn heimi, en
lið alt myndi vera niðri í eyjum at lúka heyverkum. Riðu þeir Gizurr Hvíti ok
Geirr Goði austr yfir ár, þegar þeir spurðu þat, ok austr yfir sanda til Hofs.
Þá sendu þeir orð Starkaði undir Þríhyrningi; ok fundusk þeir þar allir er at
Gunnari skyldu fara, ok réðu hversu at skyldi fara.}

\subsection*{Runic}
ᚠᛁᛋᚳ ᚠᛚᚩᛞᚢ ᚪᚻᚩᚠ ᚩᚾ ᚠᛖᚱᚷᛖᚾᛒᛖᚱᛁᚷ ᚹᚪᚱᚦ ᚷᚪ᛬ᛇᚱᛁᚳ ᚷᚱᚩᚱᚾ ᚦᚨᚱ ᚻᛖ ᚩᚾ ᚷᚱᛖᚢᛏ ᚷᛁᛇᚹᚩᛗ
ᚻᚱᚩᚾᚨᛇ ᛒᚪᚾ\\
\textit{ᚱᚩᛗᚹᚪᛚᚢᛇ ᚪᚾᛞ ᚱᛖᚢᛗᚹᚪᛚᚢᛇ} {\bfseries ᛏᚹᛟᚷᛖᚾ ᚷᛁᛒᚱᚩᚦᚫᚱ ᚪᚠᛟᛞᛞᚫ ᛞᛁᚫ ᚹᚣᛚᛁᚠ \textit{ᚩᚾ ᚱᚩᛗᚫ\linebreak[0]ᚳᚫᛇᛏᛁ᛬
ᚩᚦᛚᚫ ᚢᚾᚾᛖᚷ}}

\subsection*{German}

Ich ſag üch aber / minen fründen / Foͤꝛchtēd üch nit voꝛ denen die den
lyb toͤdend / vnd darnach nichts habennd das ſy mer thuͤgind. Ich wil
üch aber zeigē voꝛ welchem ir üch \saltb{f}oͤꝛchten ſollend. Foͤꝛchtend üch voꝛ
dem / der / nach dem er toͤdet hat / ouch macht hat zewerffen inn die
hell: ja ich ſag üch / voꝛ dem ſelben \saltb{f}oͤꝛchtēd üch. Koufft man nit
fünff Sparen vm̄ zween pfennig\\[1ex]
Die straße ist zu schmal für autos. Wohin fährt dieser Zug?\\
DIE STRAẞE IST ZU SCHMAL FÜR AUTOS.
{\itshape DIE STRAẞE IST ZU SCHMAL FÜR AUTOS.}
{\bfseries DIE STRAẞE IST ZU SCHMAL FÜR AUTOS.}
{\itshape\bfseries DIE STRAẞE IST ZU SCHMAL FÜR AUTOS.}\\[1ex]
{\scshape Die straße ist zu schmal für autos.
\bfseries Die straße ist zu schmal für autos.}
{\itshape Use c2sc for small cap Eszett:}
{\addfontfeature{Letters=UppercaseSmallCaps}DIE STRAẞE IST ZU SCHMAL FÜR AUTOS.
\bfseries DIE STRAẞE IST ZU SCHMAL FÜR AUTOS.}



\subsection*{Latin}

{\small\itshape Junicode contains the most common Latin abbreviations,
  making it suitable for diplomatic editions of Latin texts.}\\[1ex]
{\addfontfeatures{StyleSet=altogonek}Adiuuanos dſ̄ ſalutariſ noſter \&
 ꝓpt̄ głam nominiſ tui dnē liƀanoſ· \& ꝓpitiuſ eſto peccatiſ noſtriſ
 ꝓpter nomen tuum· Ne forte dicant ingentib: ubi eſt dſ̄ eorum \&
  innoteſcat innationib: corā oculiſ nr̄iſ· Poſuerunt moſticina
  ſeruorū ruorū eſcaſ uolatilib: cęli carneſ ſcōꝝ tuoꝝ beſtiiſ tenice·
  Facti ſumꝰ ob\kern+0.2ptꝓbrium uiciniſ nr̄iſ·}

\subsection*{Gothic}

jabai auk ƕas gasaiƕiþ þuk þana habandan kunþi in galiuge stada
anakumbjandan, niu miþwissei is siukis wis\-an\-dins timrjada du
galiugagudam gasaliþ matjan?  fraqistniþ auk sa unmahteiga ana
þeinamma witubnja broþar in þize Xristus gaswalt.  swaþ~þan
frawaurkjandans wiþra broþruns, slahandans ize gahugd siuka, du
Xristau fra\-waur\-keiþ.\\

{\noindent\small\itshape Use ss19 to produce Gothic letters
  automatically from transliterated text and ss20 to produce Latin
  letters from Gothic. Available in all four faces.}\\[1ex]
{\addfontfeature{StyleSet=gothic}jabai auk ƕas gasaiƕiþ þuk þana
  habandan kunþi in ga\-liuge stada anakumbjandan, niu miþwissei is
  siukis wis\-an\-dins timrjada du galiugagudam gasaliþ matjan?
  {\bfseries jabai auk ƕas gasaiƕiþ þuk þana habandan kunþi in
    ga\-liuge stada anakumbjandan, niu miþwissei is siukis
    wis\-an\-dins timrjada du galiugagudam gasaliþ matjan?}
  \textit{abgdeqzh \bfseries abgdeqzh}}

\subsection*{Sanskrit Transliteration}

\noindent mānaṁ dvividhaṁ viṣayadvai vidyātśaktyaśaktitaḥ \\
     arthakriyāyāṁ keśadirnārtho ’narthādhimokṣataḥ\\[1ex]
sadr̥śāsadr̥śatvācca viṣayāviṣayatvataḥ \\
     śabdasyānyanimittānāṁ bhāve dhīsadasattvataḥ

\subsection*{International Phonetic Alphabet}
%\fontspec[IPAMode=on]{Junicode}
hwɑn θɑt ɑːprɪl wiθ is ʃuːrəs soːtə θə drʊxt ɔf mɑrʧ hɑθ peːrsəd toː
θə roːte ɑnd bɑːðəd ɛvrɪ væɪn ɪn swɪʧ lɪkuːr ɔf hwɪʧ vɛrtɪu
ɛnʤɛndrəd ɪs θə fluːr hwɑn zɛfɪrʊs eːk wɪθ hɪs sweːtə bræːθ

\subsection*{Greek}

{\small\itshape The Greek typeface packaged with Junicode is
  Foulis Greek, named for the Foulis brothers, publishers of the famous
  Glasgow Homer (1756--8), which used the Greek Double Pica cut by
  Alexander Wilson. Those who want a more modern Greek face that
  harmonizes well with Junicode should consider GFS Didot
  Classic or GFS Porson.}\\[1ex]
{\greekfont βίβλος
γενέσεως ἰησοῦ χ\kern+1pt\salt{ρ}ιστοῦ υἱοῦ δαυὶδ
υἱοῦ ἀβραάμ.
ἀβραὰμ
ἐγέννησεν τὸν ἰσαάκ, ἰσαὰκ δὲ ἐγέννησεν
τὸν ἰακώβ, ἰακὼβ δὲ ἐγέννησεν τὸν
ἰούδαν καὶ τοὺς ἀδελφοὺς αὐτοῦ,
ἰούδας
δὲ ἐγέννησεν τὸν φάρες καὶ τὸν ζάρα
ἐκ τῆς θαμάρ, φάρες δὲ ἐγέννησεν τὸν
ἑσρώμ, ἑσρὼμ δὲ ἐγέννησεν τὸν ἀράμ,
ἀρὰμ
δὲ ἐγέννησεν τὸν ἀμιναδάβ, ἀμιναδὰβ
δὲ ἐγέννησεν τὸν ναασσών, ναασσὼν δὲ
ἐγέννησεν τὸν σαλμών,
σαλμὼν
δὲ ἐγέννησεν τὸν βόες ἐκ τῆς ῥαχά}\\[1ex]
%,
%βόες δὲ ἐγέννησεν}\\[1ex]
{\small\itshape Turn on Historic Ligatures and Stylistic Alternates for old-style ligatures
and alternative letter-shapes:}\\[1ex]
{\greekfont\addfontfeature{Ligatures=Historic}βί\salt{β}λος
γενέσεως ἰησοῦ χ\kern+1pt\salt{ρ}ισ\salt{τ}οῦ υἱοῦ δαυὶδ
υἱοῦ ἀ\salt{β}ραάμ.
ἀ\salt{β}ραὰμ
ἐγέννησεν τὸν ἰσαάκ, ἰσαὰκ δὲ ἐγέννησεν
τὸν ἰακώ\salt{β}, ἰακὼ\salt{β} δὲ ἐγέννησεν τὸν
ἰούδαν καὶ τοὺς ἀδελφοὺς αὐτοῦ,
ἰούδας
δὲ ἐγέννησεν τὸν \salt{φ}άρες καὶ τὸν ζάρα
ἐκ τῆς \salt{θ}αμάρ, φάρες δὲ ἐγέννησεν τὸν
ἑσρώμ, ἑσρὼμ δὲ ἐγέννησεν τὸν ἀράμ,
ἀρὰμ
δὲ ἐγέννησεν τὸν ἀμιναδά\salt{β}, ἀμιναδὰ\salt{β}
δὲ ἐγέννησεν τὸν ναασσών, ναασσὼν δὲ
ἐγέννησεν τὸν σαλμών,
σαλμὼν
δὲ ἐγέννησεν τὸν βόες ἐκ τῆς ῥαχά\salt{β},
βόες δὲ ἐγέννησεν}\\[1ex]

\subsection*{Lithuanian}

{\small\itshape Lithuanian poses several typographical challenges. An
  accented i retains its dot: i̇́; and certain characters with ogonek
  must avoid colliding with a following j:
  {\upshape\addfontfeatures{Contextuals=Alternate} ęj ųj}. Make sure
  Contextual Alternates (calt) is turned on; for i̇́, use i followed
  by combining dot accent (0307) and acute (0301).}\\[1ex]
Visa žemė turėjo vieną kalbą ir tuos pačius žodžius.  Kai žmonės
kėlėsi iš rytų, jie rado slėnį Šinaro krašte ir ten įsikūrė.  Vieni
kitiems sakė: Eime, pasidirbkime plytų ir jas išdekime. – Vietoj
akmens jie naudojo plytas, o vietoj kalkių – bitumą.  Eime, – jie
sakė, – pasistatykime miestą ir bokštą su dangų siekiančia viršūne ir
pasidarykime sau vardą, kad nebūtume išblaškyti po visą žemės veidą.

\subsection*{Polish}
{\small\itshape The default shape and position of ogonek in Junicode are suitable
for modern Polish. For the medieval Latin e-caudata, consider using
ss15.}\\[1ex]
Mieszkańcy całej ziemi mieli jedną mowę, czyli jednakowe słowa.  A
gdy wędrowali ze wschodu, napotkali równinę w kraju Szinear i tam
zamieszkali.  I mówili jeden do drugiego: Chodźcie, wyrabiajmy cegłę
i wypalmy ją w ogniu. A gdy już mieli cegłę zamiast kamieni i smołę
zamiast zaprawy murarskiej, rzekli: Chodźcie, zbudujemy sobie miasto
i wieżę, której wierzchołek będzie sięgał nieba, i w ten sposób
uczynimy sobie znak, abyśmy się nie rozproszyli po całej ziemi.

\subsection*{Czech}
{\small\itshape Special care has been taken with the
  handling
of Eastern European languages. The developer solicits suggestions for
further improvement.}\\[1ex]
Pojďme do Betléma a přesvědčme
se o tom, co nám anděl oznámil. Mojžíšův Zákon přikazoval, aby každá
žena čtyřicátý den po narození chlapce přinesla oběť do chrámu.
{\itshape Pojďme do Betléma a přesvědčme
se o tom, co nám anděl oznámil. Mojžíšův Zákon přikazoval, aby každá
žena čtyřicátý den po narození chlapce přinesla oběť do chrámu.}
{\bfseries Pojďme do Betléma a přesvědčme
se o tom, co nám anděl oznámil. Mojžíšův Zákon přikazoval, aby každá
žena čtyřicátý den po narození chlapce přinesla oběť do chrámu.}
{\scshape Pojďme do Betléma a přesvědčme
se o tom, co nám anděl oznámil. Mojžíšův Zákon přikazoval, aby každá
žena čtyřicátý den po narození chlapce přinesla oběť do chrámu.}

\subsection*{Fleurons}

{\small\itshape Junicode contains a number of fleurons (floral
  ornaments) copied from a 1785 Caslon specimen book. This document
  contains a number of examples. Fleurons may be found at these
  code-points: E270, E27D, E670, E67D, E68A, E736, E8B0, E8B1,
  EF90–EF9C, EF9F, F011, F014, F018, F019, F01B, F01D, F01E.}

\begin{center}
\huge    \\
 \\[0.7ex]
\\[0.7ex]
\\
 
\end{center}

\chapter*{\color{myBlue}OpenType Features}

{\itshape Following is a list of the OpenType features in
  Junicode. For instructions on applying OpenType features, consult
  the documentation for your preferred application. The first three of
  these (Standard Ligatures, Contextual Alternates, Kerning) should
  generally be on (they already are in most applications, but in
  Microsoft Word you must turn them on yourself).}

\subsection*{Standard Ligatures (liga)}

Like many old-style fonts, Junicode contains the most common f-ligatures
(first flight offer office afflict fjord) and some that are less common
(e.g. thrift fifty afraid für fördern).  It
also has long-s ligatures (e.g. aſſert ſtart ſlick omiſſion).

\subsection*{Contextual Alternates (calt)}

When this feature is on, Junicode
will avoid unsightly collisions between neighboring characters such as
f and vowels with diacritics, e.g. fêler fíf fŭl. If you find that f
collides with some other character, you can select the narrow
\saltb{f} via the OpenType {\scshape Stylistic Alternates} feature.

\subsection*{Kerning (kern)}

In most text-based applications, {\scshape Kerning} (which makes fine
adjustments to the spacing
between characters) is on by default, but in
Microsoft Word it must be turned on explicitly. Turn it off for
an antique look.

\subsection*{Stylistic Alternates (salt)}

This feature gives you direct access to a number of alternates that
are available via other features. Some of these (for example the
narrow \saltb{f}\kern+2pt) may be useful to avoid collisions that the font designer has
not anticipated. In Foulis Greek, a number of alternative letter shapes
can be accessed in this way: e.g. {\greekfont β\salt{β}γ\salt{γ}ρ\salt{ρ}τ\salt{τ}φ\salt{φ}.}

\subsection*{Discretionary Ligatures (dlig)}

This feature will give you fancy ligatures, e.g. %
{\addfontfeature{Ligatures=Discretionary} act star track bitten
  attract,} %
and also connected Roman numbers (%
{\addfontfeature{Ligatures=Discretionary} I II III IV V VI VII VIII IX X XI
  XII}---regular and italic faces).
Use it also for circled numbers and letters:
[1] {\addfontfeature{Ligatures=Discretionary}= [1]};
[A] {\addfontfeature{Ligatures=Discretionary}= [A]};
[a] {\addfontfeature{Ligatures=Discretionary}= [a]};
[[1]] {\addfontfeature{Ligatures=Discretionary}= [[1]]};
<1> {\addfontfeature{Ligatures=Discretionary}= <1>}
(regular and italic faces only).

\subsection*{Historic Ligatures (hlig)}

Nearly all of MUFI’s ``non-structural'' ligatures are
accessible via {\scshape Historic Ligatures}.
{\addfontfeature{Ligatures=Historic}Even if you are not a medievalist,
  you may still be amused by the strange effects you can achieve by
  turning on this feature: egg track fan sock aardvark.}
  This feature will also permit you to access a number
of historical ligatures in Foulis Greek, e.g.
{\greekfont\addfontfeatures{Ligatures=Historic}ἰφθίμους
  ἐτελείε\salt{τ}ο
  διαστήτην μάχεσθαι χραίσμῃ.}

\subsection*{Historic Forms (hist)}

This feature provides long ſ. In keeping with the usage of early printers, round s is preserved
at the ends of words: {\addfontfeature{Style=Historic}“When to the sessions of sweet silent thought.”}

\subsection*{Mark Positioning (mark and mkmk)}

Where no precomposed character is available, combining marks are
still correctly positioned, and marks can be “stacked” via {\scshape Mark
to Base} (mark) and {\scshape Mark to Mark} (mkmk): ŏ́ (o + U+306 + U+301);
ī̆ (i + U+304 + U+306).  The dot of an i or j followed by a diacritic
is removed: i̽. If your application supports these
features, they are almost certainly on by default.

\subsection*{Small Capitals (smcp and c2sc)}

Use {\scshape Small Caps} to change lower-case letters to small caps;
add {\scshape Caps to Small Caps} for text entirely in small
caps. {\scshape Junicode has true small caps rather than scaled
  capitals.} Special small cap versions of common combining diacritics
are available, and these should be positioned correctly relative to
the base characters: {\scshape äçé}. {\itshape Regular, Italic, and
  Bold faces.}

\subsection*{Old-Style Numbers (onum)}

You have a choice of either standard “lining” figures or old-style
figures, selected by {\scshape Old-Style Numbers}: 0123456789
{\addfontfeature{Numbers=OldStyle}0123456789.}

\subsection*{Slashed Zero (zero)}

Turn on this feature for a slash through the digit zero (both lining and old-style):
{\addfontfeature{Numbers=SlashedZero} 0 {\addfontfeature{Numbers=OldStyle} 0}}.

\subsection*{Superscripts and Subscripts (sups, subs)}

\noindent Superscript numbers are rendered with {\scshape Superscripts}:
{\addfontfeature{Superscripts=on} 0123456789}.  Subscript numbers
are rendered with {\scshape Subscripts}:
{\addfontfeature{Subscripts=on} 0123456789}. In the regular and
italic styles there is a complete alphabet of superscripts (e.g.
{\addfontfeature{Superscripts=on}abcxyz}).

\subsection*{Fractions (frac)}

A sequence of number + slash + number is rendered by a fraction if the
fraction has a Unicode encoding and this feature is on:
{\addfontfeature{Fractions=on} 1/2 1/4 2/3 3/4} (complete set of Unicode
fractions in regular and italic).

\subsection*{Letters with flourishes (swsh)}
For letters with flourishes (sometimes used for setting Middle English
texts), use {\scshape Swash}:
{\addfontfeature{Style=Swash}c d f g k n r}. Some capital swashes are also
available in the italic face, based on those in Hickes's \textit{Thesaurus}:
{\addfontfeature{Style=Swash}\textit{A D J Q Æ}}.

\subsection*{Mirrored runes (rtlm)}

In the regular and italic faces Junicode
contains mirrored versions of runes. To access these, use
{\scshape Right-to-Left Mirroring}: {\addfontfeatures{MirrorRunes=on}
  ᚾᚪᛒᛋᚫᚾᚩᚱᚻ.} This feature will not reverse the order of the runes,
but only the shapes of the characters.

\subsection*{Greek letters in IPA}

Earlier versions of Junicode contained an awkward workaround for the
problem of IPA characters based on Greek together in a font with a
complete Greek character set in a different style. Now that Junicode's
Greek has been moved to the Foulis Greek font, IPA characters based on
Greek have been moved to the Greek range, and no special coding is
needed to access them.

\subsection*{Nordic letter-shapes}

The default shapes of ð and þ in Junicode are English: this is unusual in
modern fonts. For the shapes used in Icelandic, specify the Icelandic
language, if your application has good language support, or select
{\scshape Stylistic Set 1}: {\addfontfeature{Language=Icelandic} ðþ}.

\subsection*{Insular letter-shapes (ss02)}

Use {\scshape Stylistic Set 2} for insular letter-forms:
{\fontspec[StyleSet=insular]{Junicode} abcdefg.}

\subsection*{Overlined characters (ss04, ss05)}

Use {\scshape Stylistic Set 4} for roman numbers with high overline
({\fontspec[StyleSet=highline]{Junicode} viii XCXV}) and {\scshape Stylistic Set 5}
for lower-case roman numbers with medium-high overline
({\fontspec[StyleSet=medline]{Junicode} viii dclx}). These Stylistic
Sets will work only with letters used in Roman numbers.

\subsection*{Enlarged minuscules (ss06)}

{\scshape Stylistic Set 6} produces enlarged minuscules, thus:
{\addfontfeature{StyleSet=enlarged} abcdefg.} Since the underlying
text remains unchanged, enlarged text can be searched like normal
text.

\subsection*{Deleted text (ss07)}

In medieval manuscripts, text is often deleted by placing a dot under each
letter. Both Unicode and MUFI define many characters with dots below:
{\addfontfeature{StyleSet=underdot} if possible, you should avoid
hard-coding these and instead use} {\scshape Stylistic Set 7}.

\subsection*{Alternate yogh (ss08)}

For Middle English, always use the yogh at U+021C and U+021D (Ȝȝ).
Unicode also has an alternative yogh, which in Junicode has a
flat top. If you prefer this, leave the underlying text the same and
specify {\scshape Stylistic Set 8}:
{\addfontfeature{StyleSet=altyogh} Ȝȝ}.


\subsection*{Retired letter-shapes (ss09)}

The design of a few Junicode characters has changed since the font was
introduced. The original designs, if you prefer them, will always be
available via {\scshape Stylistic Set 9}. Currently there are just a few
such alternates: {\fontspec[StyleSet=altshapes]{Junicode} ꝺ} for ꝺ,
{\addfontfeature{StyleSet=altshapes} T} for T,
{\scshape{\addfontfeature{StyleSet=altshapes} t} for t}.

\subsection*{Long slashes (ss10)}

Some users prefer slashes and backslashes that are longer than
Junicode's default. Use {\scshape Stylistic Set 10} to access these:
a/a, {\addfontfeatures{StyleSet=longslash}a/a}.

\subsection*{Alternative italic æ (ss11)}

In texts that contain both æ and œ, the two may be confused in the
italic face: \textit{æ œ}. In this case, use {\scshape Stylistic Set 11} to
substitute \textit{\addfontfeatures{StylisticSet=11}æ} for \textit{æ}.

\subsection*{E caudata (ss15)}

Medieval Latin texts often use an {\itshape e} with tail, called
{\itshape e caudata}, to represent Latin {\itshape ae} or {\itshape
  oe}. Polish, Lithuanian, and several other languages also use this
letter. While in modern editions of medieval texts the {\itshape
  cauda} (or in Polish, the {\itshape ogonek}) is often attached to
the very bottom of the letter, in modern Polish and Lithuanian
printing it is attached to the end of the bottom stroke: Polish ę,
medieval Latin {\addfontfeatures{StyleSet=altogonek}ę}. The modern
Polish version of the letter is acceptable for medieval Latin;
however, if you prefer a centered {\itshape cauda}, use
{\scshape Stylistic Set 15}.

\subsection*{Linguistic alternates (ss17)}

One character (ʔ, U+0294) has an alternate shape used in phonetic
transcription. Access this with {\scshape Stylistic Set 17}.

\subsection*{Old-Style Punctuation (ss18)}

{\addfontfeature{StyleSet=oldpunct}Old books generally set
extra space before the heavier punctuation marks (; : ! ?);
they also leave extra space inside quotation marks and
parentheses (e.g. “here”). For a similar effect, use {\scshape Stylistic Set 18}. Make sure
that {\scshape Contextual Alternates} are also on so that Junicode can correct
the spacing in certain environments.}

\subsection*{Latin-to-Gothic Transliteration (ss19)}

As transliteration of Latin to Gothic characters is straightforward,
it can easily be handled with OpenType features. Note that the Gothic
alphabet has no distinction between upper- and lower-case, so capitals
and lower-case letters are transliterated the same way:
{\addfontfeature{StyleSet=gothic} mahtedi sweþauh jah inu mans leik}.

\subsection*{Gothic-to-Latin Transliteration (ss20)}

The same as ss19, but in reverse. It produces all lower-case
letters. Thus 𐌲𐌰𐌳𐍉𐌱 𐌽𐌿 𐍅𐌰𐍃 𐌼𐌰𐌹𐍃 𐌸𐌰𐌽𐍃 𐍃𐍅𐌴𐍃𐍅𐌰𐌼𐌼𐌰
becomes ‘{\addfontfeature{StyleSet=gothtolat}𐌲𐌰𐌳𐍉𐌱 𐌽𐌿 𐍅𐌰𐍃 𐌼𐌰𐌹𐍃 𐌸𐌰𐌽𐍃 𐍃𐍅𐌴𐍃𐍅𐌰𐌼𐌼𐌰}’.

\begin{center}
\huge {\color{myRed}}
\end{center}

\chapter*{\color{myBlue}Other Features}

\subsection*{Treatment of Obsolete Characters}

A number of medieval characters originally assigned by MUFI to the
Unicode Private Use Area have been accepted into the Unicode
standard. For several years Junicode retained the obsolete
characters, adding a mark to warn document maintainers to reencode
their documents. Beginning with version 0.7.3 obsolete MUFI characters
were removed from the font.

\subsection*{Character Protrusion}

For XeLaTeX users who use the Microtype package for character
protrusion, a configuration file (mt-Junicode.cfg) is provided for
Junicode. Users of XeLaTeX will need Microtype version 2.5 or
higher. The configuration file is designed for XeLaTeX, but it can
easily be edited to work with LuaTeX.

\chapter*{\color{myBlue}Miscellanea}

\noindent The Junicode font is available at
https://github.com/psb1558/Junicode-New/tree/master/legacy. You can also find it in the
repositories of many Linux distributions and via CTAN. Visit the
Junicode Project Page at SourceForge to leave feature requests and bug
reports. Suggestions and Contributions are welcome: if you wish to
contribute to Junicode, leave a patch at the Project Page or contact
the developer.  Feature requests and bug reports can be left in the
same place.

Junicode comes in Regular, Italic, Bold and Bold Italic faces, but the
Regular and Italic faces have the fullest character set and are
richest in OpenType features. The font implements the recommendation
of the Medieval Unicode Font Initiative version 4.0. Download the
complete recommendation at http://www.mufi.info/.

Junicode is licensed under the SIL Open Font License: for the full
text, go to http://scripts.sil.org/OFL. Briefly: You may use
Junicode in any kind of publication, print or electronic, without fee
or restriction. You may modify the font for your own use. You may
distribute your modified version in accordance with the terms of the
license.\\[1ex]

\def\reflect#1{{\setbox0=\hbox{#1}\rlap{\kern0.5\wd0
  \special{x:gsave}\special{x:scale -1 1}}\box0 \special{x:grestore}}}
\def\XeTeX{\leavevmode
  \setbox0=\hbox{X\lower.5ex\hbox{\kern-.15em\reflect{E}}\kern-.1667em \TeX}%
  \dp0=0pt\ht0=0pt\box0 }

\noindent This document was set with {\XeTeX}.
\end{document}
