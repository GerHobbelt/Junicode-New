% This file was converted to LaTeX by Writer2LaTeX ver. 1.6.1
% see http://writer2latex.sourceforge.net for more info
\documentclass[letterpaper,12pt]{article}
\title{Junicode 2 Feature Reference}
\author{Peter S. Baker}
\usepackage{fancyhdr}
\usepackage[ascii]{inputenc}
\usepackage{amsmath}
\usepackage{amssymb,amsfonts,textcomp}
% \usepackage[T1]{fontenc}
\usepackage[quiet]{fontspec}
\setmainfont{Junicode Two Beta SemiExpanded}[Language=English,StylisticSet=10]
\usepackage[english]{babel}
\usepackage{xcolor,colortbl}
\definecolor{BrickRed}{RGB}{146,18,6}
%\definecolor{LightGray}{RGB}{205,205,205}
\definecolor{SlateGray}{RGB}{112,128,144}
\definecolor{GGOrange}{RGB}{240,74,6}
\definecolor{RViolet}{RGB}{70,18,87}
\usepackage{multicol}
\usepackage{array}
\usepackage{supertabular}
\usepackage{hhline}
\usepackage{hyperref}
\usepackage{metalogo}
\hypersetup{pdftex, colorlinks=true, linkcolor=blue, citecolor=blue, filecolor=blue, urlcolor=blue, pdftitle=, pdfauthor=, pdfsubject=, pdfkeywords=}
% footnotes configuration
\makeatletter
\renewcommand\thefootnote{\arabic{footnote}}
\makeatother
% \setlength{\skip\footins}{20pt}
% Text styles
\linespread{1.1}
\newcommand\textLetterExample[1]{\textrm{\textbf{\color{BrickRed}#1}}}
\newcommand\textUName[1]{\textsc{#1}}
\newcommand\textSourceText[1]{{\color{GGOrange}\texttt{#1}}}
\newcommand\textstyleEmphasis[1]{\textit{#1}}
\newcommand\textstyleEntityRef[1]{\textrm{#1}}
% \newcommand{\cv}[3]{{\addfontfeature{CharacterVariant=#1:#2}#3}}
\newcommand{\cvd}[3][0]{{\addfontfeature{CharacterVariant=#2:#1}#3}}
\newcommand{\hlig}[1]{{\addfontfeature{Ligatures=Historic}#1}}
\newcommand{\sups}[1]{{\addfontfeature{VerticalPosition = Superior}#1}}
\newcommand{\subs}[1]{{\addfontfeature{VerticalPosition = Inferior}#1}}
\newcommand{\oprop}[1]{{\addfontfeature{Numbers={Lowercase,Proportional}}#1}}
\newcommand{\lprop}[1]{{\addfontfeature{Numbers={Uppercase,Proportional}}#1}}
\newcommand{\otab}[1]{{\addfontfeature{Numbers={Lowercase,Monospaced}}#1}}
\newcommand{\ltab}[1]{{\addfontfeature{Numbers={Uppercase,Monospaced}}#1}}
\newcommand{\ornm}[2][0]{{\addfontfeature{Ornament=#1}#2}}
\newcommand{\revthorn}[1]{{\addfontfeature{StylisticSet=1}#1}}
\newfontface\narrow{Junicode Two Beta Condensed}
\newfontface\stditalic{Junicode Two Beta Italic}
\newopentypefeature{Style}{mirrored}{+rtlm}
%\newopentypefeature{Ligatures}{histon}{+hlig}
\newopentypefeature{Ligatures}{histoff}{-hlig}
% Outline numbering
\setcounter{secnumdepth}{0}
\makeatletter
\newcommand\arraybslash{\let\\\@arraycr}
\makeatother
% Page layout (geometry)
\setlength\voffset{-1in}
\setlength\hoffset{-0.75in}
\setlength\topmargin{1in}
\setlength\oddsidemargin{1in}
\setlength\textheight{8.000001in}
\setlength\textwidth{6in}
\setlength\footskip{0.0cm}
\setlength\headheight{0.4in}
\setlength\headsep{0.2in}
% Footnote rule
\setlength{\skip\footins}{14pt}
\renewcommand\footnoterule{\vspace*{-0.0071in}\setlength\leftskip{0pt}\setlength\rightskip{0pt plus 1fil}\noindent\textcolor{black}{\rule{0.25\columnwidth}{0.0071in}}\vspace*{0.0398in}}
% Pages styles
\pagestyle{fancy}
\footskip = 30pt
\headsep = 30pt
\renewcommand{\headrule}{}
\lhead{}
\chead{\textsc{Junicode 2 Feature Reference}}
\rhead{}
\lfoot{}
\cfoot{\thepage}
\rfoot{}
\setlength\tabcolsep{1mm}
\renewcommand\arraystretch{1.3}
% Headers
\usepackage{sectsty}
\subsectionfont{\color{BrickRed}}
\sectionfont{\color{SlateGray}}
% List styles
\newcommand\liststyleLi{%
\renewcommand\labelitemi{{\textbullet}}
\renewcommand\labelitemii{{\textbullet}}
\renewcommand\labelitemiii{{\textbullet}}
\renewcommand\labelitemiv{{\textbullet}}
}
\newcommand\liststyleLii{%
\renewcommand\labelitemi{{\textbullet}}
\renewcommand\labelitemii{{\textbullet}}
\renewcommand\labelitemiii{{\textbullet}}
\renewcommand\labelitemiv{{\textbullet}}
}
\newcounter{Feature}
\renewcommand\theFeature{\arabic{Feature}}
\tolerance=1500
\frenchspacing
%
%
%
\begin{document}
\maketitle
%{\mdseries\color[rgb]{0.47058824,0.011764706,0.4509804}
%\Large OpenType Features in Junicode 2}\\[1ex]
\thispagestyle{plain}

\hyperlink{intro}{Introduction}

\begin{itemize}
\setlength\itemsep{0em}
\item[A] \hyperlink{SectionA}{Case-Related Features}

\item[B] \hyperlink{SectionB}{Numbers and Sequencing}

\item[C] \hyperlink{SectionC}{Superscripts and Subscripts}

\item[D] \hyperlink{SectionD}{Ornaments}

\item[E] \hyperlink{SectionE}{Alphabetic Variants}

\item[F] \hyperlink{SectionF}{Punctuation}

\item[G] \hyperlink{SectionG}{Abbreviations}

\item[H] \hyperlink{SectionH}{Combining Marks}

\item[I] \hyperlink{SectionI}{Currency and Weights}

\item[J] \hyperlink{SectionJ}{Gothic}

\item[K] \hyperlink{SectionK}{Runic}

\item[L] \hyperlink{SectionL}{Ligatures and Digraphs}

\item[M] \hyperlink{req}{Required Features}

\item[N] \hyperlink{nonmufi}{Non-MUFI Code Points}
\end{itemize}

\hypertarget{intro}{}\section{Introduction}
The OpenType features of Junicode version 2 and its variable counterpart (hereafter referred to together as
``Junicode'') have two purposes. One is to provide convenient access to the rich character set of the Medieval Unicode
Font Initiative (MUFI) recommendation. The other is to enable best practices in the presentation of medieval text,
promoting accessibility in electronic texts from PDFs to e-books to web pages.
\thispagestyle{plain}

Each character in the MUFI recommendation has a code point\footnote{A Unicode code point is generally expressed as a
four-digit hexadecimal (or base-16) number with a prefix of ``U+''. The letter capital ``A,'' for example, is U+0041 (65
in decimal notation), and lowercase ``ȝ'' (Middle English yogh) is U+021D.} associated with it: either the one
assigned by Unicode or, where the character is not recognized by Unicode, in the Private Use Area (PUA) of the Basic
Multilingual Plane, a block of codes, running from U+E000 to U+F8FF, that are assigned no value by Unicode but instead
are available for font designers to use in any way they please.

The problem with PUA code points is precisely their lack of any value. Consider, as a point of comparison, the letter
\textLetterExample{a} (U+0061). Your computer, your phone, and probably a good many other devices around the house
store a good bit of information about this \textLetterExample{a}: that it’s a letter in the Latin script, that
it’s lowercase, and that the uppercase equivalent is \textLetterExample{A} (U+0041). All this information is
available to word processors, browsers, and other applications running on your computer.

Now suppose you're preparing an electronic text containing what MUFI calls \textUName{latin small letter neckless
a} (\textLetterExample{}). It is assigned to code point U+F215, which belongs to the PUA. Beyond that, your
computer knows nothing about it: not that it is a variant of \textLetterExample{a}, or that it is lowercase, or a letter in the Latin
alphabet, or even a character in a language system. A screen reader cannot read, or even spell out, a word with U+F215
in it; a search engine will not recognize the word as containing the letter \textLetterExample{a}.

Junicode offers the full range of MUFI characters---you can enter the PUA code points---but also a solution to the
problems posed by those code points. Think of an electronic text (a web page, perhaps, or a PDF) as having two layers:
an underlying text, stable and unchanging, and the displayed text, generated by software at the instant it is needed
and discarded when it is no longer on the screen. For greatest accessibility the underlying text should contain the
plain letter \textLetterExample{a} (U+0061) along with markup indicating how it should be displayed. To generate
the displayed text, a program called a ``layout engine'' will (simplifying a bit here) read the markup and apply the
OpenType feature \textSourceText{cv02[5]}\footnote{Many OpenType features produce different outcomes depending on
an index passed to an application’s layout engine along with the feature tag. Different applications have different
ways of entering this index: consult your application’s documentation. Here, the index is recorded in brackets after
the feature tag. Users of fontspec (with {\XeLaTeX} or {\LuaTeX}) should also be aware that fontspec indexes start at zero
while OpenType indexes start at one. Therefore all index numbers listed in this document must be reduced by one for
use with fontspec.\par } to the underlying \textLetterExample{a}, bypassing the PUA code point, with the result that
readers see \textLetterExample{\cvd[4]{2}{a}}{}---the ``neckless a.'' And yet the letter will still register as
\textLetterExample{a} with search engines, screen readers, and so on.

This is the Junicode model for text display, but it is not peculiar to Junicode: it is widely considered to be the best
practice for displaying text using current font technology.

The full range of OpenType features listed in this document is supported by all major web browsers, LibreOffice, XeTeX,
LuaTeX, and (presumably) other document processing applications. All characters listed here are available in Adobe
InDesign, though that program supports only a selection of OpenType features. Microsoft Word, unfortunately, supports
only Stylistic Sets, ligatures (all but the standard ones in peculiar and probably useless combinations), number
variants, and the \hyperlink{req}{Required Features}. In terms of
OpenType support, Word is the most primitive of the major text processing applications.

Many MUFI characters cannot be produced by using the OpenType variants of Junicode. These characters fall into three
categories:

\liststyleLi
\begin{itemize}
\item Those with Unicode (non-PUA) code points. MUFI has done valuable work obtaining Unicode code points for medieval characters.
All such characters (those with hexadecimal codes that \textstyleEmphasis{do not} begin with \textLetterExample{E}
or \textLetterExample{F}) are presumed safe to use in accessible and searchable text. However, some of these are
covered by Junicode OpenType features for particular reasons.
\item Precomposed characters---those consisting of base character + one or more diacritics. For greatest accessibility,
these should be entered not as PUA code points, but rather as sequences consisting of base character + one or more
diacritics. For example, instead of MUFI U+E498 \textUName{latin small letter e with dot below and acute}, use
\textLetterExample{e} + U+0323 \textUName{combining dot below} + U+0301 \textUName{combining acute accent}:
\textLetterExample{ẹ́} (when applying combining marks, start with any marks below the character and work
downwards, then continue with any marks above the character and work upwards. For example, to make
\textLetterExample{ǭ̣́}, place characters in this order: \textLetterExample{o},
\textUName{combining ogonek} U+0328, \textUName{combining dot below} U+0323, \textUName{combining
macron} U+0304, \textUName{combining acute} U+0301). Some MUFI characters have marks in unconventional locations,
e.g. \textLetterExample{ȯ́} \textUName{latin small letter o with dot above and acute}, where the
acute appears beside the dot instead of above. This and other characters like it should still be entered as a sequence
of base character + marks (here \textLetterExample{o}, \textUName{combining dot above} U+0307,
\textUName{combining acute} U+0301). Junicode will position the marks correctly.
\item Characters for which a base character (a Unicode character to which it can be linked) cannot be identified, or for
which there may be an inconsistency in the MUFI recommendation. These include:

\begin{itemize}
\item \textLetterExample{} U+E8AF. This is a ligature of long \textLetterExample{s} and \textLetterExample{l} with stroke,
but there are no base characters with this style of stroke.
\item \textLetterExample{} U+EFD8 and U+EFD9. MUFI lists these as ligatures (corresponding to the
historic ligatures \textLetterExample{\hlig{uuUU}}, but they cannot be treated as ligatures in the
font because a single diacritic is positioned over the glyphs as if they were digraphs like
\textLetterExample{ꜳꜲ}.
\item \textLetterExample{} U+EBE7 and U+EBE6, for the same reason.
\item \textLetterExample{} U+F159 \textUName{latin abbreviation sign small de}. Neither a variant of
\textLetterExample{d} nor an eth (\textLetterExample{ð}), this character may be a candidate for Unicode
encoding.
\end{itemize}
\item Characters for which OpenType programming is not yet available. These will be added as they are located and
studied. [Check: U+EBF1, and smcp version.]
\end{itemize}
These characters should be avoided, even if you are otherwise using MUFI’s PUA characters:

\liststyleLii
\begin{itemize}
\item U+F1C5 \textUName{combining curl high position}. Use U+1DCE \textUName{combining ogonek above}. The
positioning problem mentioned in the MUFI recommendation is solved in Junicode (and, to be fair, many other fonts with
OpenType programming).
\item U+F1CA \textUName{combining dot above high position}. Use U+0307 \textUName{combining dot above}. It
will be positioned correctly on any character.
\end{itemize}
\pagebreak
\hypertarget{SectionA}{}\section{A. Case-Related Features}
\subsection[1. smcp {}-- Small Capitals]{\stepcounter{Feature}{\theFeature}.
\textSourceText{smcp} -- Small Capitals}
Converts lowercase letters to small caps; also several symbols and combining marks. All lower- and uppercase pairs (with
exceptions noted below) have a small cap equivalent. Lowercase letters without matching caps may lack matching small
caps. fghij $\rightarrow $ \textsc{fghij}.

Note: Precomposed characters defined by MUFI in the Private Use Area have no small cap equivalents. Instead, compose
characters using combining diacritics, as outlined in the introduction. For example, \textSourceText{smcp} applied
to the sequence \textLetterExample{t} + \textUName{combining ogonek} (U+0328) + \textUName{combining
acute} (U+0301) will change \textLetterExample{t̨́} to \textLetterExample{\textsc{t̨́}}.

\subsection[2. c2sc {}-- Small Capitals from Capitals]{\stepcounter{Feature}{\theFeature}. \textSourceText{c2sc} --
Small Capitals from Capitals}
Use with \textSourceText{smcp} for all-small-cap text. ABCDE $\rightarrow $ {\addfontfeature{Letters = UppercaseSmallCaps}ABCDE}.

\subsection[3. pcap {}-- Petite Capitals]{\stepcounter{Feature}{\theFeature}. \textSourceText{pcap} -- Petite
Capitals}
Produces small caps in a smaller size than \textSourceText{smcp}. Use these when small caps have to be mixed with
lowercase letters. The whole of the basic Latin alphabet is covered, plus several other letters. klmno{\th}
$\rightarrow $ {\addfontfeature{Letters = PetiteCaps}klmno\th}.

\subsection[4. case {}-- Case{}-Sensitive Forms]{\stepcounter{Feature}{\theFeature}. \textSourceText{case} --
Case-Sensitive Forms}
Produces combining marks that harmonize with capital letters: {\addfontfeature{Letters=Uppercase}\v{R}, X̉}, etc. Use of this feature reduces the
likelihood that a combining mark will collide with a glyph in the line above.

\hypertarget{SectionB}{}\section{B. Numbers and Sequencing}
\subsection[5. nalt {}-- Alternate Annotation Forms]{\stepcounter{Feature}{\theFeature}.
\textSourceText{nalt} -- Alternate Annotation Forms}
Produces letters and numbers circled, in parenthesis, or followed by periods, as follows:

\textSourceText{nalt[1]}, circled letters or numbers: {\addfontfeature{Annotation=0}a b .~.~. z; 0 1 2 .~.~. 20}.

\textSourceText{nalt[2]}, letter or numbers in parentheses: {\addfontfeature{Annotation=1}a .~.~. z; 0 1 .~.~. 20}.

\textSourceText{nalt[3]}, double-circled numbers: {\addfontfeature{Annotation=2}0 1 .~.~. 10}.

\textSourceText{nalt[4]}, white numbers in black circles: {\addfontfeature{Annotation=3}0 1 2 3 . . . 20}.

\textSourceText{nalt[5]}, numbers followed by period: {\addfontfeature{Annotation=0}0 1 . . . 20}.

\noindent For enclosed figures 10 and higher, \textSourceText{rlig} (Required Ligatures) must also be enabled (as it should
be by default: see \hyperlink{req}{Required Features} below).

\subsection[6. tnum {}-- Tabular Figures]{\stepcounter{Feature}{\theFeature}. \textSourceText{tnum} -- Tabular
Figures}
Fixed-width figures: 0123456789 (default or with \textSourceText{lnum}), {\addfontfeature{Numbers=OldStyle}0123456789} (with
\textSourceText{onum}).

\subsection[7. onum {}-- Oldstyle Figures]{\stepcounter{Feature}{\theFeature}. \textSourceText{onum} -- Oldstyle
Figures}
Figures that harmonize with lowercase characters: {\addfontfeature{Numbers=OldStyle}0123456789} (default or with
\textSourceText{tnum}), {\addfontfeature{Numbers={Proportional,OldStyle}}0123456789}
(with \textSourceText{pnum}). When combined with \textSourceText{pnum}, this feature also affects subscripts
and superscripts.

\subsection[8. pnum {}-- Proportional Figures]{\stepcounter{Feature}{\theFeature}. \textSourceText{pnum} --
Proportional Figures}
Proportionally spaced figures: {\addfontfeature{Numbers=Proportional}0123456789} (default or with \textSourceText{lnum}),
{\addfontfeature{Numbers={Lowercase,Proportional}}0123456789} (with
\textSourceText{onum}). When combined with \textSourceText{onum}, this feature also affects subscripts and
superscripts. Most applications (including MS Word) with any support of OpenType features will support this feature and
\textSourceText{lnum} in such a way that you don't have to enter them manually.

\subsection[9. lnum {}-- Lining Figures]{\stepcounter{Feature}{\theFeature}. \textSourceText{lnum} -- Lining
Figures}
Figures in a uniform height, harmonizing with uppercase letters: 0123456789 (default or with
\textSourceText{tnum}), {\addfontfeature{Numbers=Proportional}0123456789} (with \textSourceText{pnum}).

\subsection[10. zero {}-- Slashed Zero]{\stepcounter{Feature}{\theFeature}. \textSourceText{zero} -- Slashed Zero}
Produces slashed zero in all number styles:
{\addfontfeature{Numbers=SlashedZero}0 \otab{0} \lprop{0} \oprop{0}. Includes superscripts and subscripts:
\sups{0~\oprop{0}}~\subs{0~\oprop{0}}}.

\hypertarget{SectionC}{}\section{C. Superscripts and Subscripts}
\subsection[11. sups {}-- Superscripts]{\stepcounter{Feature}{\theFeature}.
\textSourceText{sups} -- Superscripts}
Produces superscript numbers and letters. Only affects lining tabular and oldstyle proportional figures. All lowercase
letters of the basic Latin alphabet are covered, and most uppercase letters: \sups{0123 \oprop{4567} abcde ABDEG}. Wherever
superscripts are needed (e.g. for footnote numbers), use \textSourceText{sups} instead of the raised and scaled
characters generated by some programs. With sups: \sups{4567}. Scaled: \textsuperscript{4567}.

\subsection[12. subs {}-- Subscripts]{\stepcounter{Feature}{\theFeature}. \textSourceText{subs} -- Subscripts}
Produces subscript numbers. Only affects lining tabular (the default numbers) and oldstyle proportional figures (use
\textSourceText{pnum} and \textSourceText{onum}): \subs{8901 \oprop{2345}}.

\hypertarget{SectionD}{}\section{D. Ornaments}
\subsection[13. ornm {}-- Ornaments]{\stepcounter{Feature}{\theFeature}.
\textSourceText{ornm} -- Ornaments}
Produces ornaments (fleurons) in either of two ways: as an indexed variant of the bullet character (U+2022) or as
variants of a-z, A-C (all fleurons are available by either method):

As a variant of {\textbullet}: 1=\ornm{\textbullet}, 2=\ornm[1]{\textbullet}, 3=\ornm[2]{\textbullet}, 4=\ornm[3]{\textbullet}, etc., up to 29.

As a variant of a-z, A-C: e=\ornm{e}, f=\ornm{f}, g=\ornm{g}, h=\ornm{h}, etc.

\noindent The method with letters of the alphabet is easier, but the method with bullets will produce a more satisfactory result
when text is displayed in an environment where Junicode is not available or \textSourceText{ornm} is not
implemented.

\hypertarget{SectionE}{}\section{E. Alphabetic Variants}
\subsection[14. cv01{}--cv52 {}-- Basic Latin Variants]{\stepcounter{Feature}{\theFeature}.
\textSourceText{cv01-cv52} -- Basic Latin Variants}
These features also affect small cap (\textSourceText{smcp}) and underdotted (\textSourceText{ss07}) forms,
where available. Where Junicode has no variant for a Basic Latin letter, the expected \textSourceText{cvNN}
feature is skipped, being reserved for future development.

%\definecolor{LightBlueGray}{RGB}{212,218,218}
\begin{center}
%\tablefirsthead{}
\tablefirsthead{\hline
%\rowcolor{LightBlueGray}
\centering{\bfseries Variant of} &\
\centering{\bfseries cvNN} &
\centering\arraybslash{\bfseries Variants}\\}
%\tablehead{}
\tablehead{\hline
%\rowcolor{LightBlueGray}
\centering{\bfseries Variant of} &
\centering{\bfseries cvNN} &
\centering\arraybslash{\bfseries Variants}\\}
\tabletail{\hline}
\tablelasttail{}
\begin{supertabular}{|m{0.79135984in}|m{0.79135984in}|m{2.9212599in}|}
\hline
%\centering{\bfseries Variant of} &
%\centering{\bfseries cvNN} &
%\centering\arraybslash{\bfseries Variants}\\\hline
\color{black}\centering{A} &
\centering{cv01} &
1=\cvd{1}{A}, 2=\cvd[1]{1}{A}, 3=\cvd[2]{1}{A}\\\hline
\centering{a} &
\centering{cv02} &
{1=\cvd{2}{a}, 2=\cvd[1]{2}{a}, 3=\cvd[2]{2}{a}, 4=\cvd[3]{2}{a}, 5=\cvd[4]{2}{a}}\\\hline
\centering{B} &
\centering{cv03} &
{No variants available}\\\hline
\centering{b} &
\centering{cv04} &
{No variants available}\\\hline
\centering{C} &
\centering{cv05} &
{1=\cvd{5}{C}}\\\hline
\centering{c} &
\centering{cv06} &
{1=\cvd{6}{c}}\\\hline
\centering{D} &
\centering{cv07} &
{1=\cvd{7}{D}}\\\hline
\centering{d} &
\centering{cv08} &
{1=\cvd{8}{d}, 2=\cvd[1]{8}{d}, 3=\cvd[2]{8}{d} (for 1, see also ss02)}\\\hline
\centering{E} &
\centering{cv09} &
{1=\cvd{9}{E}, 2=\cvd[1]{9}{E}}\\\hline
\centering{e} &
\centering{cv10} &
{1=\cvd{10}{e}, 2=\cvd[1]{10}{e}, 3=\cvd[2]{10}{e}}\\\hline
\centering{F} &
\centering{cv11} &
{1=\cvd{11}{F}}\\\hline
\centering{f} &
\centering{cv12} &
{1=\cvd{12}{f}, 2=\cvd[1]{12}{f}, 3=\cvd[2]{12}{f}, 4=\cvd[3]{12}{f}, 5=\cvd[4]{12}{f}, 6=\cvd[5]{12}{f}}\\\hline
\centering{G} &
\centering{cv13} &
{1=\cvd{13}{G}, 2=\cvd[1]{13}{G}, 3=\cvd[2]{13}{G}}\\\hline
\centering{g} &
\centering{cv14} &
{1=\cvd{14}{g}, 2=\cvd[1]{14}{g}, 3=\cvd[2]{14}{g}, 4=\cvd[3]{14}{g}, 5=\cvd[4]{14}{g}, 6=\cvd[5]{14}{g}, 7=\cvd[6]{14}{g}}\\\hline
\centering{H} &
\centering{cv15} &
{1=\cvd{15}{H}}\\\hline
\centering{h} &
\centering{cv16} &
{1=\cvd{16}{h}, 2=\cvd[1]{16}{h}}\\\hline
\centering{I} &
\centering{cv17} &
{1=\cvd{17}{I}, 2=\cvd[1]{17}{I}}\\\hline
\centering{i} &
\centering{cv18} &
{1=\cvd{18}{i}, 2=\cvd[1]{18}{i}, 3=\cvd[2]{18}{i}}\\\hline
\centering{J} &
\centering{cv19} &
{1=\cvd{19}{J}}\\\hline
\centering{j} &
\centering{cv20} &
{1=\cvd{20}{j}, 2=\cvd[1]{20}{j}, 3=\cvd[2]{20}{j}}\\\hline
\centering{K} &
\centering{cv21} &
{No variants available}\\\hline
\centering{k} &
\centering{cv22} &
{1=\cvd{22}{k}, 2=\cvd[1]{22}{k}, 3=\cvd[2]{22}{k}, 4=\cvd[3]{22}{k}}\\\hline
\centering{L} &
\centering{cv23} &
{No variants available}\\\hline
\centering{l} &
\centering{cv24} &
{1=\cvd{24}{l}}\\\hline
\centering{M} &
\centering{cv25} &
{1=\cvd{25}{M}, 2=\cvd[1]{25}{M}, 3=\cvd[2]{25}{M}}\\\hline
\centering{m} &
\centering{cv26} &
{1=\cvd{26}{m}, 2=\cvd[1]{26}{m}, 3=\cvd[2]{26}{m}}\\\hline
\centering{N} &
\centering{cv27} &
{1=\cvd{27}{N}}\\\hline
\centering{n} &
\centering{cv28} &
{1=\cvd{28}{n}, 2=\cvd[1]{28}{n}, 3=\cvd[2]{28}{n}, 4=\cvd[3]{28}{n}}\\\hline
\centering{O} &
\centering{cv29} &
{1=\cvd{29}{O}}\\\hline
\centering{o} &
\centering{cv30} &
{1=\cvd{30}{o}}\\\hline
\centering{P} &
\centering{cv31} &
{1=\cvd{31}{P}}\\\hline
\centering{p} &
\centering{cv32} &
{No variants available}\\\hline
\centering{Q} &
\centering{cv33} &
{1=\cvd{33}{Q}}\\\hline
\centering{q} &
\centering{cv34} &
{1=\cvd{34}{q}}\\\hline
\centering{R} &
\centering{cv35} &
{1=\cvd{35}{R}}\\\hline
\centering{r} &
\centering{cv36} &
{1=\cvd{36}{r}, 2=\cvd[1]{36}{r}}\\\hline
\centering{S} &
\centering{cv37} &
{1=\cvd{37}{S}, 2=\cvd[1]{37}{S}}\\\hline
\centering{s} &
\centering{cv38} &
{1=\cvd{38}{s}, 2=\cvd[1]{38}{s}, 3=\cvd[2]{38}{s}, 4=\cvd[3]{38}{s},
            5=\cvd[4]{38}{s}, 6=\cvd[5]{38}{s}}\\\hline
\centering{T} &
\centering{cv39} &
{1=\cvd{39}{T}}\\\hline
\centering{t} &
\centering{cv40} &
{1=\cvd{40}{t}, 2=\cvd[1]{40}{t}}\\\hline
\centering{U} &
\centering{cv41} &
{No variants available}\\\hline
\centering{u} &
\centering{cv42} &
{No variants available}\\\hline
\centering{V} &
\centering{cv43} &
{No variants available}\\\hline
\centering{v} &
\centering{cv44} &
{1=\cvd{44}{v}, 2=\cvd[1]{44}{v}, 3=\cvd[2]{44}{v}, 4=\cvd[3]{44}{v}}\\\hline
\centering{W} &
\centering{cv45} &
{No variants available}\\\hline
\centering{w} &
\centering{cv46} &
{No variants available}\\\hline
\centering{X} &
\centering{cv47} &
{No variants available}\\\hline
\centering{x} &
\centering{cv48} &
{1=\cvd{48}{x}, 2=\cvd[1]{48}{x}, 3=\cvd[2]{48}{x}, 4=\cvd[3]{48}{x}}\\\hline
\centering{Y} &
\centering{cv49} &
{1=\cvd{49}{Y}}\\\hline
\centering{y} &
\centering{cv50} &
{1=\cvd{50}{y}, 2=\cvd[1]{50}{y}}\\\hline
\centering{Z} &
\centering{cv51} &
{1=\cvd{51}{Z}}\\\hline
\centering{z} &
\centering{cv52} &
{1=\cvd{52}{z}, 2=\cvd[1]{52}{z}}\\\hline
\end{supertabular}
\end{center}
\hypertarget{OtherLatin}{}\subsection[15. cv53{}-cv67 {}-- Other Latin Letters]{\stepcounter{Feature}{\theFeature}.
\textSourceText{cv53-cv67} -- Other Latin Letters}
Some features affect both upper- and lowercase forms. \textSourceText{cv62} also affects
combining \textLetterExample{e} with ogonek, accessible via \textSourceText{\hyperlink{ss10}{ss10}} with the
entity reference \textSourceText{\&\_eogo;}. In this range, \textSourceText{cvNN} features are not reserved
for future development, since Junicode already uses or reserves all of the available \textSourceText{cvNN}
features.

\begin{center}
\tablefirsthead{\hline
%\rowcolor{LightBlueGray}
\centering{\bfseries Variant of} &
\centering{\bfseries cvNN} &
\centering\arraybslash{\bfseries Variants}\\}
\tablehead{\hline
%\rowcolor{LightBlueGray}
\centering{\bfseries Variant of} &
\centering{\bfseries cvNN} &
\centering\arraybslash{\bfseries Variants}\\}
\tabletail{\hline}
\tablelasttail{}
\begin{supertabular}{|m{1.8913599in}|m{0.79135984in}|m{1.8212599in}|}
\hline
\centering \k{A} (U+0104) &
\centering cv53 &
{1=\cvd{53}{Ą}, 2=\cvd[1]{53}{Ą}, 3=\cvd[2]{53}{Ą}}\\\hline
\centering \k{a} (U+0105) &
\centering cv54 &
{1=\cvd{54}{ą}, 2=\cvd[1]{54}{ą}}\\\hline
\centering ꜳ (U+A733) &
\centering cv55 &
{1=\cvd{55}{ꜳ}, 2=\cvd[1]{55}{ꜳ}}\\\hline
\centering  {\AE} (U+00C6) &
\centering cv56 &
{1=\cvd{56}{\AE}}\\\hline
\centering {\ae} (U+00E6) &
\centering cv57 &
{1=\cvd{57}{\ae}, 2=\cvd[1]{57}{\ae}, 3=\cvd[2]{57}{\ae}}\\\hline
\centering Ꜵ (U+A734) &
\centering cv58 &
{1=\cvd{58}{Ꜵ}}\\\hline
\centering ꜵ (U+A735) &
\centering cv59 &
{1=\cvd{59}{ꜵ}, 2=\cvd[1]{59}{ꜵ}, 3=\cvd[2]{59}{ꜵ}}\\\hline
\centering ꜹ (U+A739) &
\centering cv60 &
{1=\cvd{60}{ꜹ}}\\\hline
\centering {\dj} (U+0111) &
\centering cv61 &
{1=\cvd{61}{\dj}}\\\hline
\centering {\narrow Ę, ę ... (U+0118, U+0119)} &
\centering cv62 &
{1=\cvd{62}{Ę, ę ...}; 2=\cvd[1]{62}{Ę, ę ...}}\\\hline
\centering {\narrow Ȝ, ȝ (U+021C, U+021D)} &
\centering cv63 &
{1=\cvd{63}{Ȝ, ȝ}}\\\hline
\centering ꝉ (U+A749) &
\centering cv64 &
{1=\cvd{64}{ꝉ}}\\\hline
\centering {\o} (U+00F8) &
\centering cv65 &
{1=\cvd{65}{\o}, 2=\cvd[1]{65}{\o}, 3=\cvd[2]{65}{\o}, 4=\cvd[3]{65}{\o}}\\\hline
\centering ꝥ, \revthorn{ꝥ} (U+A765) &
\centering cv66 &
{1=\cvd{66}{ꝥ, \revthorn{ꝥ}}}\\\hline
\centering ? &
\centering cv67 & Reserved for future use\\\hline
\end{supertabular}
\end{center}
\subsection[16. ss01 {}-- Alternate thorn and eth]{\stepcounter{Feature}{\theFeature}. \textSourceText{ss01} --
Alternate thorn and eth}
Produces Nordic thorn and eth (\revthorn{{\th}{\dh}{\TH}}) when the language is English, and English thorn and eth
({\th}{\dh}{\TH}) with any other language, reversing the font’s usual usage. This also affects small caps, crossed
thorn (ꝥ \revthorn{ꝥ}---see also
\hyperlink{OtherLatin}{\textSourceText{cv67}}), combining mark eth
(U+1DD9, ◌ᷙ \revthorn{◌ᷙ}), and enlarged thorn and eth (see \textSourceText{\hyperlink{ss06}{ss06}}).
This feature depends on \textSourceText{\hyperlink{req}{loca}} (Localized Forms), which in most applications will
always be enabled.

\subsection[17. ss02 {}-- Insular Letter{}-Forms]{\stepcounter{Feature}{\theFeature}. \textSourceText{ss02} --
Insular Letter-Forms}
Produces insular letter-forms, e.g. {\addfontfeature{StylisticSet=2}dfgrsw}. Does not affect capitals (except W), as these do not not commonly have
insular shapes in early manuscripts. For these, enter the Unicode code points or use the Character Variant
(\textSourceText{cvNN}) features.

\subsection[18. ss04 {}-- High Overline]{\stepcounter{Feature}{\theFeature}. \textSourceText{ss04} -- High
Overline}
Produces a high overline over letters used as roman numbers: {\addfontfeature{StylisticSet=4}cdijlmvx CDIJLMVXↃ}.

\subsection[19. ss05 {}-- Medium{}-High Overline]{\stepcounter{Feature}{\theFeature}. \textSourceText{ss05} --
Medium-High Overline}
Produces a medium-high overline over (or through the ascenders of) letters used as roman numbers, and some others as
well: {\addfontfeature{StylisticSet=5,Style=Historic}bcdhijklmſvx{\th}}.

\hypertarget{ss06}{}\subsection[20. ss06 {}-- Enlarged Minuscules]{\stepcounter{Feature}{\theFeature}. \textSourceText{ss06} --
Enlarged Minuscules}
Lowercase letters that match the height of normal ones, but with a higher x-height, e.g.
{\addfontfeature{StylisticSet=6}abcdefg}.
Covers the whole of the basic Latin alphabet and several other letters: consult the MUFI recommendation for details,
and if you are using the variable version of the font (JunicodeVF), consider using the
\href{https://psb1558.github.io/Junicode-New/EnlargedAxis.html}{Enlarged axis} instead, for reasons of flexibility and
accessibility.

\subsection[21. ss07 {}-- Underdotted Text]{\stepcounter{Feature}{\theFeature}. \textSourceText{ss07} --
Underdotted Text}
Produces underdotted text (indicating deletion in medieval manuscripts) for many
letters (including
the whole of the basic Latin alphabet and a number of other letters), e.g.
{\addfontfeature{StylisticSet=7}abcdefg HIJKLM}. This also affects small
caps, e.g. \textsc{abcdef} $\rightarrow $ {\addfontfeature{StylisticSet=7}\textsc{abcdef}}.
For letters without corresponding underdotted forms (e.g. U+A751, ꝑ),
use U+0323, combining dot below (\hspace{0.2em}ꝑ̣).

\subsection[22. ss08 {}-- Contextual Long s]{\stepcounter{Feature}{\theFeature}. \textSourceText{ss08} --
Contextual Long s}
In English and French text only, varies \textLetterExample{s} and \textLetterExample{ſ} according to rules
followed by many early printers: {\addfontfeature{StylisticSet=8}sports, essence, stormy, disheveled, transfusions, slyness, cliffside}. For this
feature to work properly, \textSourceText{calt} ``Contextual Alternates'' must also be enabled (as it should be by
default: see \hyperlink{req}{Required Features} below).

\hypertarget{ss16}{}\subsection[23. ss16 {}-- Contextual r Rotunda]{\stepcounter{Feature}{\theFeature}. \textSourceText{ss16} --
Contextual r Rotunda}
Converts \textLetterExample{r} to \textLetterExample{ꝛ} (lowercase only) following the
most common rules of medieval manuscripts: {\addfontfeature{StylisticSet=16}priest, firmer, frost, ornament}. For this feature to work properly,
\textSourceText{calt} ``Contextual Alternates'' must also be enabled (as it should be by default: see
\hyperlink{req}{Required Features} below).

\subsection[24. cv68 {}-- Variant of ʔ (U+0294, glottal stop)]{\stepcounter{Feature}{\theFeature}.
\textSourceText{cv68} -- Variant of ʔ (U+0294, glottal stop)}
1=\cvd{68}{ʔ}.

\hypertarget{RevertSmallCapA}{}\subsection[25. cv99 {}-- revert small cap a to lowercase a]{\stepcounter{Feature}{\theFeature}.
\textSourceText{cv}\textSourceText{99} -- revert small cap \textsc{a} to lowercase a}
1=a. This features reverts small cap
\textLetterExample{\textsc{a}} to \textLetterExample{a}, enabling it to
ligature with small cap \textLetterExample{\textsc{n}} or
\textLetterExample{\textsc{r}} via \textSourceText{hlig}: \textsc{\cvd{99}{\hlig{an, ar}}}.
Be sure to apply \textSourceText{smcp}, \textSourceText{cv99} and
\textSourceText{hlig} to both components of the ligature.

\hypertarget{SectionF}{}\section{F. Punctuation}
MUFI encodes nearly twenty marks of punctuation in the PUA. In Junicode these can be accessed in
either of two ways: all are indexed variants of \textLetterExample{.} (period), and all are associated with the Unicode marks of
punctuation they most resemble (but it should not be inferred that the medieval marks are semantically identical with
the Unicode marks, or that there is an etymological relationship between them). The first method will be easier for
most to use, but the second is more likely to yield acceptable fallbacks in environments where Junicode is not
available.

Marks with Unicode encoding are not included here, as they can safely be entered directly. In MUFI 4.0 several marks
have PUA encodings, but have since that release been assigned Unicode code points: \textit{paragraphus} (⹍
U+2E4D), medieval comma (⹌~U+2E4C), \textit{punctus elevatus} (⹎ U+2E4E), \textit{virgula suspensiva}
(⹊ U+2E4A), triple dagger (⹋ U+2E4B).

\subsection[26. ss18 {}-- Old{}-Style Punctuation Spacing]{\stepcounter{Feature}{\theFeature}.
\textSourceText{ss18} -- Old-Style Punctuation Spacing}
Colons, semicolons, parentheses, quotation marks and several other glyphs are spaced as in early printed books.

\subsection[27. cv69 {}-- Variants of ⁊⹒ (U+204A / U+2E52, Tironian
nota)]{\stepcounter{Feature}{\theFeature}. \textSourceText{cv69} -- Variants of ⁊⹒
(U+204A / U+2E52, Tironian nota)}
1=\cvd{69}{⁊⹒}, 2=\cvd[1]{69}{⁊⹒}.

\subsection[28. cv70 {}-- Variants of . (period)]{\stepcounter{Feature}{\theFeature}. \textSourceText{cv70} --
  Variants of . (period)}
1=\cvd{70}{.}, 2=\cvd[1]{70}{.}, 3=\cvd[2]{70}{.}, 4=\cvd[3]{70}{.}, 5=\cvd[4]{70}{.}, 6=\cvd[5]{70}{.},
7=\cvd[6]{70}{.}, 8=\cvd[7]{70}{.}, 9=\cvd[8]{70}{.}, 10=\cvd[9]{70}{.}, 11=\cvd[10]{70}{.}, 12=\cvd[11]{70}{.},
13=\cvd[12]{70}{.}, 14=\cvd[13]{70}{.}, 15=\cvd[14]{70}{.}, 16=\cvd[15]{70}{.}, 17=\cvd[16]{70}{.},
18=\cvd[17]{70}{.}, 19=\cvd[18]{70}{.}, 20=\cvd[19]{70}{.}. This
feature provides access to all non-Unicode MUFI punctuation marks. Some of them are available via other features (see
below).

\subsection[29. cv71 {}-- Variant of {\textperiodcentered} (U+00B7, middle dot)]{\stepcounter{Feature}{\theFeature}.
\textSourceText{cv71} -- Variant of {\textperiodcentered} (U+00B7, middle dot)}
1=\cvd{71}{\textperiodcentered} (\textit{distinctio}).

\subsection[30. cv72 {}-- Variants of , (comma)]{\stepcounter{Feature}{\theFeature}. \textSourceText{cv72} --
Variants of , (comma)}
1=\cvd{72}{,}, 2=\cvd[1]{72}{,}.

\subsection[31. cv73 {}-- Variants of ; (semicolon)]{\stepcounter{Feature}{\theFeature}. \textSourceText{cv73} --
Variants of ; (semicolon)}
1=\cvd{73}{;} (\textit{punctus versus}), 2=\cvd[1]{73}{;}, 3=\cvd[2]{73}{;}, 4=\cvd[3]{73}{;}, 5=\cvd[4]{73}{;}.

\subsection[32. cv74 {}-- Variants of ⹎ (U+2E4E, punctus elevatus)]{\stepcounter{Feature}{\theFeature}.
\textSourceText{cv74} -- Variants of ⹎ (U+2E4E, \textit{punctus elevatus})}
1=\cvd{74}{⹎}, 2=\cvd[1]{74}{⹎}, 3=\cvd[2]{74}{⹎}, 4=\cvd[3]{74}{⹎} (\textit{punctus flexus}).

\subsection[33. cv75 {}-- Variant of ! (exclamation mark)]{\stepcounter{Feature}{\theFeature}.
\textSourceText{cv7}\textSourceText{5} -- Variant of ! (exclamation mark)}
1=\cvd{75}{!} (\textit{punctus exclamativus}).

\subsection[34. cv76 {}-- Variants of ? (question mark)]{\stepcounter{Feature}{\theFeature}. \textSourceText{cv76}
-- Variants of ? (question mark)}
1=\cvd{76}{?}, 2=\cvd[1]{76}{?}, 3=\cvd[2]{76}{?}. Shapes of the \textit{punctus interrogativus}.

\subsection[35. cv77 {}-- Variant of \~{} (ASCII tilde)]{\stepcounter{Feature}{\theFeature}. \textSourceText{cv77}
-- Variant of \~{} (ASCII tilde)}
1=\cvd{77}{\~{}} (same as MUFI U+F1F9, ``wavy line'').

\subsection[36. cv78 {}-- Variant of * (asterisk)]{\stepcounter{Feature}{\theFeature}. \textSourceText{cv78} --
Variant of * (asterisk)}
1=\cvd{78}{*}. MUFI defines U+F1EC as a \textit{signe de renvoi}. Manuscripts employ a number of shapes (of which this is one) for
this purpose. Junicode defines it as a variant of the asterisk---the most common modern \textit{signe de renvoi}.

\subsection[37. cv79 {}-- Variants of / (slash)]{\stepcounter{Feature}{\theFeature}.
\textSourceText{cv7}\textSourceText{9} -- Variants of / (slash)}
1=\cvd{79}{/}, 2=\cvd[1]{79}{/}. The first of these is Unicode, U+2E4E.

\hypertarget{SectionG}{}\section{G. Abbreviations}
\subsection[38. cv80 {}-- Variant of ꝝ (U+A75D, rum
abbreviation)]{\stepcounter{Feature}{\theFeature}. \textSourceText{cv80} -- Variant of ꝝ (U+A75D, rum
abbreviation)}
1=\cvd{80}{ꝝ}.

\subsection[39. cv81 {}-- Variants of ◌͛ (U+035B, combining zigzag
above)]{\stepcounter{Feature}{\theFeature}. \textSourceText{cv81} -- Variants of ◌͛ (U+035B, combining
zigzag above)}
1=\cvd{81}{◌͛}, 2=\cvd[1]{81}{◌͛}, 3=\cvd[2]{81}{◌͛}. Positioning of the zigzag can differ from that of other combining
marks, e.g. b͛, f͛, d͛. If \textSourceText{calt} ``Contextual Alternates'' is enabled (as it is by
default in most apps), variant forms of \textSourceText{cv81[2]} will be used with several letters, e.g. \cvd[1]{81}{d͛,
  f͛, k͛}. Enable \textSourceText{case} for forms that harmonize with capitals
({\addfontfeature{Letters=Uppercase}\cvd[1]{81}{A͛ B͛ C͛ D͛}}), \textSourceText{smcp} for forms that harmonize with small caps
  (\textsc{\cvd[1]{81}{e͛ f͛ g͛ h͛}}).

\subsection[40. cv82 {}-- Variants of spacing ꝰ (U+A770)]{\stepcounter{Feature}{\theFeature}.
\textSourceText{cv82} -- Variants of spacing ꝰ (U+A770)}
1=\cvd{82}{ꝰ}, 2=\cvd[1]{82}{ꝰ}. \textSourceText{cv82[1]} produces the baseline \textit{{}-us} abbreviation (same as MUFI
U+F1A6). MUFI also has an uppercase baseline \textit{{}-us} abbreviation (U+F1A5), but as there is no uppercase version
of U+A770 to pair it with, it is indexed separately here.

\subsection[41. cv83 {}-- Variants of ꝫ (U+A76B, {}``et{}'' abbreviation)]{\stepcounter{Feature}{\theFeature}.
\textSourceText{cv83} -- Variants of ꝫ (U+A76B, ``et'' abbreviation)}
1=\cvd{83}{ꝫ}, 2=\cvd[1]{83}{ꝫ}. \textSourceText{[1]} is identical in shape to a
semicolon, but as it is semantically the same as U+A76B, it is preferable to use that
character with this feature. \textSourceText{[2]} produces a subscript version of
the character, a common variant in printed books.

\hypertarget{SectionH}{}\section{H. Combining Marks}
\hypertarget{cv84}{}\subsection[42. cv84 {}-- MUFI combining marks (variants of ◌̄
U+0304)]{\stepcounter{Feature}{\theFeature}. \textSourceText{cv84} -- MUFI combining marks (variants of U+0304)}
MUFI encodes a number of combining marks in the PUA (with code points between E000 and F8FF), but when these characters
are entered directly, they can interfere with searching and accessibility, and some important applications fail to
position them correctly over their base characters. To avoid these problems, enter U+0304 (◌̄, \textUName{combining
macron}) and apply \textSourceText{cv84}, with the appropriate index, to both mark and base character. This
collection of marks does not include any Unicode-encoded marks (from the ``Combining Diacritical Marks'' ranges), as
these can safely be entered directly. It does include three marks (\textSourceText{cv84[36]},
\textSourceText{[37]} and \textSourceText{[38]}) that lack MUFI code points but are used to form MUFI
characters.

This feature may often appear to have no effect. When this happens it is because
an application replaced a sequence like \textLetterExample{a U+0304} with a precomposed character
like \textLetterExample{ā} (U+0101) before Junicode's OpenType programming had a chance to work.
This process is called normalization, and it usually has the effect of simplifying
text processing tasks, but can sometimes prevent the proper functioning of OpenType
features. To disable it, place the character U+034F \textUName{combining
grapheme joiner} (don't waste any time puzzling over the name) between the base
character and the combining mark (or the first combining mark). For example, to produce
the combination \textLetterExample{\cvd[1]{84}{u͏̄}}, enter \textLetterExample{u U+034F U+0304}.
(Without U+034F, you would get \textLetterExample{\cvd[1]{84}{ū}}).

These marks can sometimes be produced by other \textSourceText{cvNN} features, which may be preferable to
\textSourceText{cv84} as providing more suitable fallbacks for applications that do not support Character Variant
(\textSourceText{cvNN}) features.

For some marks with PUA code points, users may find it easier to use \hyperlink{ss10}{entities} than this feature.

These marks are not affected by most other features. This is to preserve flexibility, given the rule that the feature
that produces them must be applied to both the mark and the base character. For example, if \textSourceText{smcp}
``Small Caps'' changed \textSourceText{cv84[11]} \cvd[11]{84}{◌͏̄} to \textSourceText{[12]} \cvd[12]{84}{◌͏̄}, it
would be impossible to produce the sequence
\textLetterExample{\cvd[11]{84}{\textsc{na͏̄a}}} with the diacritic properly positioned.

\begin{multicols}{4}
\color{BrickRed}1=\cvd{84}{◌͏̄}

2=\cvd[1]{84}{◌͏̄}

3=\cvd[2]{84}{◌͏̄}

4=\cvd[3]{84}{◌͏̄}

5=\cvd[4]{84}{◌͏̄}

6=\cvd[5]{84}{◌͏̄}

7=\cvd[6]{84}{◌͏̄}

8=\cvd[7]{84}{◌͏̄}

9=\cvd[8]{84}{◌͏̄}

10=\cvd[9]{84}{◌͏̄}

11=\cvd[10]{84}{◌͏̄}

12=\cvd[11]{84}{◌͏̄}

13=\cvd[12]{84}{◌͏̄}

14=\cvd[13]{84}{◌͏̄}

15=\cvd[14]{84}{◌͏̄}

16=\cvd[15]{84}{◌͏̄}

17=\cvd[16]{84}{◌͏̄}

18=\cvd[17]{84}{◌͏̄}

19=\cvd[18]{84}{◌͏̄}

20=\cvd[19]{84}{◌͏̄}

21=\cvd[20]{84}{◌͏̄}

22=\cvd[21]{84}{◌͏̄}

23=\cvd[22]{84}{◌͏̄}

24=\cvd[23]{84}{◌͏̄}

25=\cvd[24]{84}{◌͏̄}

26=\cvd[25]{84}{◌͏̄}

27=\cvd[26]{84}{◌͏̄}

28=\cvd[27]{84}{◌͏̄}

29=\cvd[28]{84}{◌͏̄}

30=\cvd[29]{84}{◌͏̄}

31=\cvd[30]{84}{◌͏̄}

32=\cvd[31]{84}{◌͏̄}

33=\cvd[32]{84}{◌͏̄}

34=\cvd[33]{84}{◌͏̄}

35=\cvd[34]{84}{◌͏̄}

36=\cvd[35]{84}{◌͏̄}

37=\cvd[36]{84}{◌͏̄}

37=\cvd[37]{84}{◌͏̄}
\end{multicols}

\hypertarget{ss10}{}\subsection[43. ss10 {}-- Entity References for Combining Marks]{\stepcounter{Feature}{\theFeature}.
\textSourceText{ss10} -- Entity References for Combining Marks}
Instead of \textSourceText{\hyperlink{cv84}{cv84}} for representing non-Unicode combining marks, some users may
wish to use XML/HTML-style entities. When \textSourceText{ss10} is turned on (preferably for the entire
text), these entities appear as combining marks and are correctly positioned over base characters.
For example, the letter \textLetterExample{e} followed by
\textLetterExample{\&{\textcompwordmark}\_eogo;} will yield \textLetterExample{e\&\_eogo;}. An advantage of entities is that
they are (unlike the PUA code points or the indexes of \textSourceText{cv84}) mnemonic and thus easy to use.
A disadvantage is that
searches cannot ignore combining marks entered by this method as they can using the \textSourceText{cv84} method.
(Every method of entering non-Unicode combining marks has disadvantages: users should weigh these, choose a method,
and stick with it.)

If you use any of these entities in a work intended for print publication, you should call your publisher’s
attention to them, since they will likely have their own method of representing them.

\begin{multicols}{3}
\color{RViolet}
\&{\textcompwordmark}\_ansc; $\rightarrow $
\textstyleEntityRef{◌\&\_ansc;}

\&{\textcompwordmark}\_an; $\rightarrow $
\textstyleEntityRef{◌\&\_an;}

\&{\textcompwordmark}\_ar; $\rightarrow $
\textstyleEntityRef{◌\&\_ar;}

\&{\textcompwordmark}\_arsc; $\rightarrow $
\textstyleEntityRef{◌\&\_arsc;}

\&{\textcompwordmark}\_bsc; $\rightarrow $
\textstyleEntityRef{◌\&\_bsc;}

\&{\textcompwordmark}\_dsc; $\rightarrow $
\textstyleEntityRef{◌\&\_dsc;}

\&{\textcompwordmark}\_eogo; $\rightarrow $
\textstyleEntityRef{◌\&\_eogo;}

\&{\textcompwordmark}\_emac; $\rightarrow $
\textstyleEntityRef{◌\&\_emac;}

\&\_{\textcompwordmark}idotl; $\rightarrow $
\textstyleEntityRef{◌\&\_idotl;}

\&\_{\textcompwordmark}j; $\rightarrow $
\textstyleEntityRef{◌\&\_j;}

\&\_{\textcompwordmark}jdotl; $\rightarrow $
\textstyleEntityRef{◌\&\_jdotl;}

\&\_{\textcompwordmark}ksc; $\rightarrow $
\textstyleEntityRef{◌\&\_ksc;}

{\narrow\&\_{\textcompwordmark}munc; $\rightarrow $
\textstyleEntityRef{◌\&\_munc;}}

\&\_{\textcompwordmark}oogo; $\rightarrow $
\textstyleEntityRef{◌\&\_oogo;}

{\narrow\&\_{\textcompwordmark}oslash; $\rightarrow $
\textstyleEntityRef{◌\&\_oslash;}}

\&\_{\textcompwordmark}omac; $\rightarrow $
\textstyleEntityRef{◌\&\_omac;}

\&\_{\textcompwordmark}orr; $\rightarrow $
\textstyleEntityRef{◌\&\_orr;}

\&\_{\textcompwordmark}oru; $\rightarrow $
\textstyleEntityRef{◌\&\_oru;}

\&\_{\textcompwordmark}q; $\rightarrow $
\textstyleEntityRef{◌\&\_q;}

\&\_{\textcompwordmark}ru; $\rightarrow $
\textstyleEntityRef{◌\&\_ru;}

\&\_{\textcompwordmark}tsc; $\rightarrow $
\textstyleEntityRef{◌\&\_tsc;}

\&\_{\textcompwordmark}y; $\rightarrow $
\textstyleEntityRef{◌\&\_y;}

{\narrow\&\_{\textcompwordmark}thorn; $\rightarrow $
\textstyleEntityRef{◌\&\_thorn;}}
\end{multicols}

\subsection[44. ss20 {}-- Low Diacritics]{\stepcounter{Feature}{\theFeature}. ss20 -- Low Diacritics}
The MUFI recommendation includes a number of precomposed characters with base letters b, h, k, {\th}, ꝺ and {\dh}
and combining marks ◌ͣ (U+0363), ◌ͤ (U+0364), \cvd[17]{84}{◌͏̄}
(U+0304\slash\textSourceText{cv84[18]}), ◌ͦ (U+0366), ◌ͬ (U+036C), ◌ᷢ (U+1DE2),
◌ͭ (U+036D), ◌ͮ (U+036E), ◌ᷦ (U+1DE6) and \cvd[21]{84}{◌͏̄}
(U+0304/\textSourceText{cv84[21]}). Instead of being positioned above ascender height as usual (e.g.
\textLetterExample{hͣ}), the MUFI glyphs have the marks positioned above the x-height
(e.g. \textLetterExample{\addfontfeature{StylisticSet=20}hͣ}).
Using the MUFI code points for these precomposed glyphs can interfere with searching
and drastically reduce accessibility. Users of Junicode should instead use a sequence of base character + combining
mark, and apply \textSourceText{ss20} to the two glyphs. A variant shape of eth (\textLetterExample{{\dh}})
that accommodates the combining mark will be substituted for the normal base character (but this is not necessary for
the other base characters). Examples:
{\addfontfeature{StylisticSet=20}bͦ, ꝺᷦ, \cvd[17]{84}{h̄}, kͤ, {\th}ͭ, {\dh}ᷢ}.

\textSourceText{ss20} affects only the diacritics and base characters listed here; other combinations (e.g.
\textLetterExample{mͤ}, \textLetterExample{\'{h}}) are not affected. It will therefore probably be safe
to apply this feature to the whole text if it is needed anywhere.

\subsection[45. cv85 {}-- Variant of ◌ᷓ (U+1DD3, combining open a)]{\stepcounter{Feature}{\theFeature}.
\textSourceText{cv85} -- Variant of ◌ᷓ (U+1DD3, combining open a)}
1=\cvd{85}{◌ᷓ}.

\subsection[46. cv86 {}-- Variant of ◌ᷘ U(U+1DD8, combining insular d)]{%
\stepcounter{Feature}{\theFeature}. \textSourceText{cv86} -- Variant of ◌ᷘ (U+1DD8, combining insular
d)}
1=\cvd{86}{◌ᷘ}.

\subsection[47. cv87 {}-- Variant of ◌ᷣ (U+1DE3, combining r rotunda)]{\stepcounter{Feature}{\theFeature}.
\textSourceText{cv87} -- Variant of ◌ᷣ (U+1DE3, combining r rotunda)}
1=\cvd{87}{◌ᷣ}.

\subsection[48. cv88 {}-- Variant of combining dieresis (U+0308)]{\stepcounter{Feature}{\theFeature}.
\textSourceText{cv8}\textSourceText{8} -- Variant of combining dieresis (U+0308)}
1=\cvd{88}{◌̈}. This also affects precomposed characters on which this variant dieresis may occur, e.g.
\textLetterExample{\"a}.

\subsection[49. cv89 {}-- Variant of ◌̅ (U+0305, combining
overline)]{\stepcounter{Feature}{\theFeature}. \textSourceText{cv89} -- Variant of ◌̅ (U+0305,
combining overline)}
1=\cvd{89}{◌̅}.

%\subsection[50. cv90 {}-- Variants of ◌̄  (U+0304, combining macron)]{\stepcounter{Feature}{\theFeature}.
%\textSourceText{cv}\textSourceText{90} -- Variants of ◌̄ (U+0304, combining macron)}
%1=\cvd{90}{◌̄}, 2=\cvd[1]{90}{◌̄}. These are used for abbreviations in manuscripts and
%early printed books.
%
\subsection[51. cv91 {}-- Variants of short horizontal stroke (U+0335)]{\stepcounter{Feature}{\theFeature}.
\textSourceText{cv91} -- Variants of short horizontal stroke (U+0335)}
1=\cvd{91}{◌̵}, 2=\cvd[1]{91}{◌̵}, 3=\cvd[2]{91}{◌̵}. This character can be used with letters with ascenders or
descenders, e.g. \textLetterExample{\cvd{91}{d̵ b̵ {\th}̵ p̵}}. \textSourceText{cv91[1]} widens the
stroke, and \textSourceText{cv91[2]} and \textSourceText{[3]} offset the stroke to the right or left. Via
\textSourceText{calt} ``Contextual Alternates,'' this offset is performed automatically for many characters
with ascenders and descenders, and so it should rarely be necessary to use an index with \textSourceText{cv91}.

\subsection[52. cv92 {}-- Variant of breve below (U+032E)]{\stepcounter{Feature}{\theFeature}. cv92 -- Variant of breve
below (U+032E)}
1=\cvd{92}{◌◌̮◌}. Position the mark after the middle of three glyphs, and apply \textSourceText{cv92}
to both the mark and (at least) the middle glyph. This mark is not available via \textSourceText{cv84}.

\hypertarget{SectionI}{}\section{I. Currency and Weights}
\subsection[53. cv93 {}-- Variants of {\textcurrency} (U+0044, generic currency
sign)]{\stepcounter{Feature}{\theFeature}. \textSourceText{cv93} -- Variants of {\textcurrency} (U+0044, generic
currency sign)}

\begin{multicols}{4}
\color{RViolet}1=\cvd{93}{\textcurrency}

2=\cvd[1]{93}{\textcurrency}

3=\cvd[2]{93}{\textcurrency}

4=\cvd[3]{93}{\textcurrency}

5=\cvd[4]{93}{\textcurrency}

6=\cvd[5]{93}{\textcurrency}

7=\cvd[6]{93}{\textcurrency}

8=\cvd[7]{93}{\textcurrency}

9=\cvd[8]{93}{\textcurrency}

10=\cvd[9]{93}{\textcurrency}

11=\cvd[10]{93}{\textcurrency}

12=\cvd[11]{93}{\textcurrency}

13=\cvd[12]{93}{\textcurrency}

14=\cvd[13]{93}{\textcurrency}

15=\cvd[14]{93}{\textcurrency}

16=\cvd[15]{93}{\textcurrency}

17=\cvd[16]{93}{\textcurrency}

18=\cvd[17]{93}{\textcurrency}

19=\cvd[18]{93}{\textcurrency}

20=\cvd[19]{93}{\textcurrency}

21=\cvd[20]{93}{\textcurrency}

22=\cvd[21]{93}{\textcurrency}

23=\cvd[22]{93}{\textcurrency}

24=\cvd[23]{93}{\textcurrency}

25=\cvd[24]{93}{\textcurrency}

26=\cvd[25]{93}{\textcurrency}

27=\cvd[26]{93}{\textcurrency}
\end{multicols}

\noindent All of MUFI’s currency and weight symbols (those that do
not have Unicode code points) are gathered here, but some are also variants of other currency signs (see below).

\subsection[54. cv94 {}-- Variant of ℔ (U+2114)]{\stepcounter{Feature}{\theFeature}.
\textSourceText{cv9}\textSourceText{4} -- Variant of ℔ (U+2114)}
1=\cvd{94}{℔}. Same as MUFI U+F2EB (French Libra sign).

\subsection[55. cv95 {}-- Variants of {\pounds} (U+00A3, British pound sign)]{\stepcounter{Feature}{\theFeature}.
\textSourceText{cv95} -- Variants of {\pounds} (U+00A3, British pound sign)}
1=\cvd{95}{\pounds}, 2=\cvd[1]{95}{\pounds}, 3=\cvd[2]{95}{\pounds}, 4=\cvd[3]{95}{\pounds},
5=\cvd[4]{95}{\pounds}, 6=\cvd[5]{95}{\pounds}. Same as MUFI U+F2EA, F2EB, F2EC, F2ED,
F2EE, F2EF, pound signs from various locales.

\subsection[56. cv96 {}-- Variant of ₰ (U+20B0, German penny sign)]{\stepcounter{Feature}{\theFeature}.
\textSourceText{cv96} -- Variant of ₰ (U+20B0, German penny sign)}
1=\cvd{96}{₰}. Same as MUFI U+F2F5.

\subsection[57. cv97 {}-- Variant of ƒ (U+0192, florin)]{\stepcounter{Feature}{\theFeature}.
\textSourceText{cv97} -- Variant of ƒ (U+0192, florin)}
1=\cvd{97}{ƒ}. Same as MUFI U+F2E8.

\subsection[58. cv98 {}-- Variant of ℥ (U+2125, Ounce sign)]{\stepcounter{Feature}{\theFeature}.
\textSourceText{cv98} -- Variant of ℥ (U+2125, Ounce sign)}
1=\cvd{98}{℥}. Same as MUFI U+F2FD, Script ounce sign.

\hypertarget{SectionJ}{}\section{J. Gothic}
\subsection[59. ss19 {}-- Latin to Gothic Transliteration]{\stepcounter{Feature}{\theFeature}.
\textSourceText{ss19} -- Latin to Gothic Transliteration}
Produces Gothic letters from Latin: \revthorn{Warþ þan in dagans jainans} $\rightarrow $
{\addfontfeature{StylisticSet=19}Warþ þan in dagans
jainans}. In web pages, the letters will be searchable as their Latin equivalents.

\hypertarget{SectionK}{}\section[K. Runic]{K. Runic}
\subsection[60. ss12 {}-- Early English Futhorc]{\stepcounter{Feature}{\theFeature}.
\textSourceText{ss12} -- Early English Futhorc}
Changes Latin letters to their equivalents in the early English futhorc. Because of the variability of the runic
alphabet, this method of transliteration may not produce the result you want. In that case, it may be necessary to
manually edit the result. fisc flodu ahof $\rightarrow $ {\addfontfeature{StylisticSet=12}fisc flodu ahof}.

\subsection[61. ss13 {}-- Elder Futhark]{\stepcounter{Feature}{\theFeature}. \textSourceText{ss13} -- Elder
Futhark}
Changes Latin letters to their equivalents in the Elder Futhark. Because of the variability of the runic alphabet, this
method of transliteration may not produce the result you want. In that case, it may be necessary to manually edit the
result. ABCDEFG $\rightarrow $ {\addfontfeature{StylisticSet=13}ABCDEFG}.

\subsection[62. ss14 {}-- Younger Futhark]{\stepcounter{Feature}{\theFeature}. \textSourceText{ss14} -- Younger
Futhark}
Changes Latin letters to their equivalents in the Younger Futhark. Because of the variability of the runic alphabet,
this method of transliteration may not produce the result you want. In that case, it may be necessary to manually edit
the result. ABCDEFG $\rightarrow $ {\addfontfeature{StylisticSet=14}ABCDEFG}.

\subsection[63. ss15 {}-- Long Branch to Short Twig]{\stepcounter{Feature}{\theFeature}. \textSourceText{ss15} --
Long Branch to Short Twig}
In combination with \textSourceText{ss14}, converts long branch (the default for the Younger Futhark) to short twig runes:
{\addfontfeature{StylisticSet=14}{ABCDEFG $\rightarrow $
\addfontfeature{StylisticSet=15}ABCDEFG}}.

\subsection[64. rtlm {}-- Right to Left Mirrored Forms]{\stepcounter{Feature}{\theFeature}. \textSourceText{rtlm}
-- Right to Left Mirrored Forms}
Produces mirrored runes, e.g. {\addfontfeature{StylisticSet=12}ABCDEFG $\rightarrow $ \addfontfeature{Style=mirrored}GFEDCBA}.
This feature cannot change the direction of text.

\hypertarget{SectionL}{}\section[L. Ligatures and Digraphs]{L. Ligatures and Digraphs}
\subsection[65. hlig {}-- Historic Ligatures]{\stepcounter{Feature}{\theFeature}.
\textSourceText{hlig} -- Historic Ligatures}

Produces ligatures for combinations that should not ordinarily be rendered as
ligatures in modern text.\footnote{Some
fonts define \textSourceText{hlig} differently, as including all ligatures in which at least one
element is an archaic character, e.g.
those involving long s (\textrm{ſ\hspace{0.2em}}). In Junicode, however, a
historic ligature is defined not by the characters it is composed of, but
rather by the join between them. If two characters (though modern) should not be joined except
in certain historic contexts, they form a historic ligature. If they should be
joined in all contexts (even if archaic), the ligature is not historic
and should be defined in \textSourceText{liga}.} Most of these are from the MUFI recommendation,
ranges B.1.1(b) and B.1.4. This feature does
not produce digraphs (which have a phonetic value), for which see
\textSourceText{\hyperlink{ss17}{ss17}}. The ligatures:
\addfontfeatures{Ligatures=Historic}

\begin{multicols}{5}
{\color[rgb]{0.38039216,0.09019608,0.16078432}
a{\textcompwordmark}f$\rightarrow $af}

{\color[rgb]{0.38039216,0.09019608,0.16078432}
a{\textcompwordmark}ꝼ$\rightarrow $aꝼ}

{\color[rgb]{0.38039216,0.09019608,0.16078432}
a{\textcompwordmark}g$\rightarrow $ag}

{\color[rgb]{0.38039216,0.09019608,0.16078432}
a{\textcompwordmark}l$\rightarrow $al}

{\color[rgb]{0.38039216,0.09019608,0.16078432}
a{\textcompwordmark}n$\rightarrow $an}

{\color[rgb]{0.38039216,0.09019608,0.16078432}
a{\textcompwordmark}\textsc{n}$\rightarrow $\textsc{\cvd{99}{an}}}

{\color[rgb]{0.38039216,0.09019608,0.16078432}
a{\textcompwordmark}p$\rightarrow $ap}

{\color[rgb]{0.38039216,0.09019608,0.16078432}
a{\textcompwordmark}r$\rightarrow $ar}

{\color[rgb]{0.38039216,0.09019608,0.16078432}
a{\textcompwordmark}\textsc{r}$\rightarrow $\textsc{\cvd{99}{ar}}}

{\color[rgb]{0.38039216,0.09019608,0.16078432}
a{\textcompwordmark}{\th}$\rightarrow $a{\th}}

{\color[rgb]{0.38039216,0.09019608,0.16078432}
b{\textcompwordmark}b$\rightarrow $bb}

{\color[rgb]{0.38039216,0.09019608,0.16078432}
b{\textcompwordmark}g$\rightarrow $bg}

{\color[rgb]{0.38039216,0.09019608,0.16078432}
c{\textcompwordmark}h$\rightarrow $ch}

{\color[rgb]{0.38039216,0.09019608,0.16078432}
c{\textcompwordmark}k$\rightarrow $ck}

{\color[rgb]{0.38039216,0.09019608,0.16078432}
ꝺ{\textcompwordmark}ꝺ$\rightarrow $ꝺꝺ}

{\color[rgb]{0.38039216,0.09019608,0.16078432}
e{\textcompwordmark}y$\rightarrow $ey}

{\color[rgb]{0.38039216,0.09019608,0.16078432}
f{\textcompwordmark}ä$\rightarrow $fä}

{\color[rgb]{0.38039216,0.09019608,0.16078432}
g{\textcompwordmark}d$\rightarrow $gd}

{\color[rgb]{0.38039216,0.09019608,0.16078432}
g{\textcompwordmark}\revthorn{ð}$\rightarrow $\revthorn{gð}}

{\color[rgb]{0.38039216,0.09019608,0.16078432}
g{\textcompwordmark}ꝺ$\rightarrow $gꝺ}

{\color[rgb]{0.38039216,0.09019608,0.16078432}
g{\textcompwordmark}g$\rightarrow $gg}

{\color[rgb]{0.38039216,0.09019608,0.16078432}
\cvd[2]{14}{ɡ{\textcompwordmark}ɡ}$\rightarrow $ɡɡ}

{\color[rgb]{0.38039216,0.09019608,0.16078432}
g{\textcompwordmark}o$\rightarrow $go}

{\color[rgb]{0.38039216,0.09019608,0.16078432}
g{\textcompwordmark}p$\rightarrow $gp}

{\color[rgb]{0.38039216,0.09019608,0.16078432}
g{\textcompwordmark}r$\rightarrow $gr}

{\color[rgb]{0.38039216,0.09019608,0.16078432}
H{\textcompwordmark}r$\rightarrow $Hr}

{\color[rgb]{0.38039216,0.09019608,0.16078432}
h{\textcompwordmark}r$\rightarrow $hr}

{\color[rgb]{0.38039216,0.09019608,0.16078432}
h{\textcompwordmark}ſ$\rightarrow $hſ}

{\color[rgb]{0.38039216,0.09019608,0.16078432}
h{\textcompwordmark}ẝ$\rightarrow $hẝ}

{\color[rgb]{0.38039216,0.09019608,0.16078432}
k{\textcompwordmark}r$\rightarrow $kr}

{\color[rgb]{0.38039216,0.09019608,0.16078432}
k{\textcompwordmark}ſ$\rightarrow $kſ}

{\color[rgb]{0.38039216,0.09019608,0.16078432}
k{\textcompwordmark}ẝ$\rightarrow $kẝ}

{\color[rgb]{0.38039216,0.09019608,0.16078432}
l{\textcompwordmark}l$\rightarrow $ll}

{\color[rgb]{0.38039216,0.09019608,0.16078432}
\textsc{n}{\textcompwordmark}ſ$\rightarrow $\textsc{nſ}}

{\color[rgb]{0.38039216,0.09019608,0.16078432}
o{\textcompwordmark}c$\rightarrow $oc}

{\color[rgb]{0.38039216,0.09019608,0.16078432}
O{\textcompwordmark}Ꝛ$\rightarrow $OꝚ}

{\color[rgb]{0.38039216,0.09019608,0.16078432}
o{\textcompwordmark}ꝛ$\rightarrow $oꝛ}

{\color[rgb]{0.38039216,0.09019608,0.16078432}
O{\textcompwordmark}Ꝝ$\rightarrow $OꝜ}

{\color[rgb]{0.38039216,0.09019608,0.16078432}
o{\textcompwordmark}ꝝ$\rightarrow $oꝝ}

{\color[rgb]{0.38039216,0.09019608,0.16078432}
P{\textcompwordmark}P$\rightarrow $PP}

{\color[rgb]{0.38039216,0.09019608,0.16078432}
p{\textcompwordmark}p$\rightarrow $pp}

{\color[rgb]{0.38039216,0.09019608,0.16078432}
ꝓ{\textcompwordmark}p$\rightarrow $ꝓp}

{\color[rgb]{0.38039216,0.09019608,0.16078432}
P{\textcompwordmark}s$\rightarrow $Ps}

{\color[rgb]{0.38039216,0.09019608,0.16078432}
p{\textcompwordmark}s$\rightarrow $ps}

{\color[rgb]{0.38039216,0.09019608,0.16078432}
P{\textcompwordmark}si$\rightarrow $Psi}

{\color[rgb]{0.38039216,0.09019608,0.16078432}
psi$\rightarrow $psi}

{\color[rgb]{0.38039216,0.09019608,0.16078432}
q{\textcompwordmark}ꝩ$\rightarrow $qꝩ}

{\color[rgb]{0.38039216,0.09019608,0.16078432}
q{\textcompwordmark}ꝫ/q\cvd[1]{83}{ꝫ}$\rightarrow $qꝫ/\cvd[1]{83}{qꝫ}}

{\color[rgb]{0.38039216,0.09019608,0.16078432}
Q{\textcompwordmark}Ꝛ$\rightarrow $QꝚ}

{\color[rgb]{0.38039216,0.09019608,0.16078432}
q{\textcompwordmark}ꝛ$\rightarrow $qꝛ}

{\color[rgb]{0.38039216,0.09019608,0.16078432}
ſ{\textcompwordmark}\"a$\rightarrow $ſ\"a}

{\color[rgb]{0.38039216,0.09019608,0.16078432}
ſ{\textcompwordmark}c{\textcompwordmark}h$\rightarrow $ſch}

{\color[rgb]{0.38039216,0.09019608,0.16078432}
ſ{\textcompwordmark}t{\textcompwordmark}r$\rightarrow $ſtr}

{\color[rgb]{0.38039216,0.09019608,0.16078432}
ſ{\textcompwordmark}ꝩ$\rightarrow $ſꝩ}

{\color[rgb]{0.38039216,0.09019608,0.16078432}
ꞇ{\textcompwordmark}ꞇ$\rightarrow $ꞇꞇ}

{\color[rgb]{0.38039216,0.09019608,0.16078432}
U{\textcompwordmark}E$\rightarrow $UE}

{\color[rgb]{0.38039216,0.09019608,0.16078432}
u{\textcompwordmark}e$\rightarrow $ue}

{\color[rgb]{0.38039216,0.09019608,0.16078432}
U{\textcompwordmark}U$\rightarrow $UU}

{\color[rgb]{0.38039216,0.09019608,0.16078432}
u{\textcompwordmark}u$\rightarrow $uu}

{\color[rgb]{0.38039216,0.09019608,0.16078432}
ƿ{\textcompwordmark}ƿ$\rightarrow $ƿƿ}

{\color[rgb]{0.38039216,0.09019608,0.16078432}
\revthorn{{\th}{\textcompwordmark}r$\rightarrow ${\th}r}}

{\color[rgb]{0.38039216,0.09019608,0.16078432}
\revthorn{{\th}{\textcompwordmark}s$\rightarrow ${\th}s}}

{\color[rgb]{0.38039216,0.09019608,0.16078432}
\revthorn{{\th}{\textcompwordmark}ẝ$\rightarrow ${\th}ẝ}}

{\color[rgb]{0.38039216,0.09019608,0.16078432}
{\th}\textcompwordmark{\th}$\rightarrow ${\th}{\th}}
\end{multicols}

\noindent\addfontfeatures{Ligatures=histoff}
Notes: For \textLetterExample{\textsc{\cvd{99}{\hlig{an}}}} and
\textLetterExample{\textsc{\cvd{99}{\hlig{ar}}}} see
\textSourceText{\hyperlink{RevertSmallCapA}{cv99}} above. For the ligature \textLetterExample{\textsc{\hlig{nſ}}} to
work properly, U+017F \textLetterExample{ſ} must be entered directly, not by applying an OpenType feature to
\textLetterExample{s}.

\subsection[66. dlig {}-- Discretionary Ligatures]{\stepcounter{Feature}{\theFeature}. \textSourceText{dlig} --
Discretionary Ligatures}
Produces lesser-used ligatures, but also roman numbers, e.g.
{\addfontfeature{Ligatures=Rare}ii, II, xi, XI}. The lesser-used ligatures:
\textLetterExample{\textcolor[rgb]{0.5529412,0.15686275,0.11764706}{\addfontfeature{Ligatures=Rare}ct, ſp, str, st, tr, tt, ty}}.

\hypertarget{ss17}{}\subsection[67. ss17 {}-- Rare Digraphs]{\stepcounter{Feature}{\theFeature}. \textSourceText{ss17} -- Rare
Digraphs}
By ``digraph'' we mean conjoined letters that represent a phonetic value: the most common examples
for western languages are \textLetterExample{{\ae}} and \textLetterExample{{\oe}} (though these, because they
are so common, are not included in this feature). Use of this feature in web pages enables easier searches: for
example, producing \textLetterExample{\addfontfeature{StylisticSet=17}{\th}au} from
\textLetterExample{{\th}au} allows the word to be
searched as ``{\th}au.'' The digraphs covered by this feature are \textcolor[rgb]{0.5529412,0.15686275,0.11764706}{%
\addfontfeature{StylisticSet=17}aa, ao, au, av, ay, oo, vy,} plus capital and small cap equivalents and digraph + diacritic combinations anticipated in the
MUFI recommendation. To produce such a digraph + diacritic combination, either type a letter + diacritic combination as
the second element of the digraph or type the diacritic after the second element. For example,
\textLetterExample{a} + \textLetterExample{\'u} yields \textLetterExample{\addfontfeature{StylisticSet=17}a\'u}, and so does
\textLetterExample{a} + \textLetterExample{u} + U+0301 (combining acute accent). To produce a digraph +
diacritic combination not covered by MUFI (e.g. \textLetterExample{ꜵ̀}), you may have to place U+034F
\textUName{combining grapheme joiner} (see \hyperlink{cv84}{cv84} above) between the second element of the digraph and the combining mark.
Without U+034F: \textLetterExample{\addfontfeature{StylisticSet=17}aō}.
With U+034F: \textLetterExample{\addfontfeature{StylisticSet=17}ao͏̄}.

\hypertarget{req}{}\section{M. Required Features}
Required features, which provide some of the font’s most basic functionality---ligatures, support for
other features, kerning, and more---include \textSourceText{ccmp} (Glyph Composition/Decomposition),
\textSourceText{calt} (Contextual Alternates), \textSourceText{liga} (Standard Ligatures),
\textSourceText{loca} (Localized Forms), \textSourceText{rlig} (Required Ligatures),
\textSourceText{kern} (Horizontal Kerning), and \textSourceText{mark}/\textSourceText{mkmk} (Mark
Positioning). In MS Word these features have to be explicitly enabled on the Advanced tab of the Font dialog (Ctrl-D or
Cmd-D: enable Kerning, Standard Ligatures, and Contextual Alternates, and the others will be enabled automatically),
but in most other applications they are enabled by default.


\hypertarget{nonmufi}{}\section{N. Non-MUFI Code Points}
Characters in Junicode that do not have Unicode code points should be accessed via OpenType
features whenever possible. MUFI/PUA code points should be used only in applications that do not support OpenType, or
that support it only partially (for example, MS Word). For certain characters that lack either Unicode or MUFI code
points, code points in the Supplementary Private Use Area-A (plane 15) are available.

\begin{multicols}{4}
{\color[rgb]{0.13333334,0.29411766,0.07058824}
U+F0000 󰀀}

{\color[rgb]{0.13333334,0.29411766,0.07058824}
U+F0001 󰀁}

{\color[rgb]{0.13333334,0.29411766,0.07058824}
U+F0002 󰀂}

{\color[rgb]{0.13333334,0.29411766,0.07058824}
U+F0003 󰀃}

{\color[rgb]{0.13333334,0.29411766,0.07058824}
U+F0004 󰀄}

{\color[rgb]{0.13333334,0.29411766,0.07058824}
U+F0005 󰀅}

{\color[rgb]{0.13333334,0.29411766,0.07058824}
U+F0006 󰀆}

{\color[rgb]{0.13333334,0.29411766,0.07058824}
U+F0007 󰀇}

{\color[rgb]{0.13333334,0.29411766,0.07058824}
U+F0008 󰀈}

{\color[rgb]{0.13333334,0.29411766,0.07058824}
U+F0009 󰀉}

{\color[rgb]{0.13333334,0.29411766,0.07058824}
U+F000A 󰀊}

{\color[rgb]{0.13333334,0.29411766,0.07058824}
U+F000B 󰀋}

{\color[rgb]{0.13333334,0.29411766,0.07058824}
U+F000C 󰀌}

{\color[rgb]{0.13333334,0.29411766,0.07058824}
U+F000D 󰀍}

{\color[rgb]{0.13333334,0.29411766,0.07058824}
U+F000E 󰀎}

{\color[rgb]{0.13333334,0.29411766,0.07058824}
U+F000F 󰀏}

{\color[rgb]{0.13333334,0.29411766,0.07058824}
U+F0010 󰀐}

{\color[rgb]{0.13333334,0.29411766,0.07058824}
U+F0011 󰀑}

{\color[rgb]{0.13333334,0.29411766,0.07058824}
U+F0012 󰀒}

{\color[rgb]{0.13333334,0.29411766,0.07058824}
U+F0013 󰀓}

{\color[rgb]{0.13333334,0.29411766,0.07058824}
U+F0014 󰀔}

{\color[rgb]{0.13333334,0.29411766,0.07058824}
U+F0015 󰀕}

{\color[rgb]{0.13333334,0.29411766,0.07058824}
U+F0016 󰀖}

{\color[rgb]{0.13333334,0.29411766,0.07058824}
U+F0017 󰀗}

{\color[rgb]{0.13333334,0.29411766,0.07058824}
U+F0018 󰀘}

{\color[rgb]{0.13333334,0.29411766,0.07058824}
U+F0019 󰀙}

{\color[rgb]{0.13333334,0.29411766,0.07058824}
U+F001A 󰀚}

{\color[rgb]{0.13333334,0.29411766,0.07058824}
U+F001B 󰀛}

{\color[rgb]{0.13333334,0.29411766,0.07058824}
U+F001C 󰀜}

{\color[rgb]{0.13333334,0.29411766,0.07058824}
U+F001D 󰀝}

{\color[rgb]{0.13333334,0.29411766,0.07058824}
U+F001E 󰀞}

{\color[rgb]{0.13333334,0.29411766,0.07058824}
U+F001F 󰀟}
\end{multicols}

\vspace*{\fill}
\begin{center}
{\stditalic{This document was set in 12pt Junicode SemiExpanded\\
using the {\XeLaTeX} typesetting system with fontspec for font management.\\
The source for the document, Feature\_Reference.tex, is available at}}\\
{\color{BrickRed}https://github.com/psb1558/Junicode-font.}
\end{center}
\thispagestyle{plain}
\end{document}
