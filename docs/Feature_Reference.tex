% This file was converted to LaTeX by Writer2LaTeX ver. 1.6.1
% see http://writer2latex.sourceforge.net for more info
%
% It was afterwards edited by P. Baker to remove unused packages,
% use fontspec and other packages, and keep the document up to
% date. The LibreOffice version of this document is no longer
% maintained and should not be used.
%
\documentclass[12pt,letterpaper,openany]{book}
\usepackage{microtype}
\usepackage{fancyhdr}
\usepackage[english,greek,lithuanian,polish,latin]{babel}
\usepackage[quiet]{fontspec}
\setmainfont{Junicode Two Beta}[
  UprightFont = {*-Regular},
  ItalicFont = {*-Italic},
  BoldFont = {*-Semibold},
  BoldItalicFont = {*-Bold Italic},
  Numbers={Lowercase,Proportional},
  StylisticSet=10,
  UprightFeatures={
    SizeFeatures={
      {Size={-8.5},      Font={*-SemiExpanded Medium}},
      {Size={8.6-10.99}, Font=*-Medium},
      {Size={11-21.59},  Font=*-Regular},
      {Size={21.6-},     Font=*-Light}
    },
  },
  ItalicFeatures={
    SizeFeatures={
      {Size={-7.5},      Font={*-SemiExpanded Medium Italic}},
      {Size={7.6-10.99}, Font={*-Medium Italic}},
      {Size={11-21.59},  Font=*-Italic},
      {Size={21.6-},     Font={*-Light Italic}}
    },
  },
  BoldFeatures={
    SizeFeatures={
      {Size={-8.5},      Font={*-SemiExpanded Bold}},
      {Size={8.6-10.99}, Font=*-Bold},
      {Size={11-21.59},  Font=*-Semibold},
      {Size={21.6-},     Font=*-Medium}
    },
  },
  BoldItalicFeatures={
    SizeFeatures={
      {Size={-7.5},      Font={*-SemiExpanded Bold Italic}},
      {Size={7.6-10.99}, Font={*-Bold Italic}},
      {Size={11-21.59},  Font={*-Semibold Italic}},
      {Size={21.6-},     Font={*-Medium Italic}}
    },
  },
]
%\setmainfont{JunicodeTwoBetaVF-Roman.ttf}[
%  ItalicFont = JunicodeTwoBetaVF-Italic.ttf,
%  BoldFont = JunicodeTwoBetaVF-Roman.ttf,
%  BoldItalicFont =  JunicodeTwoBetaVF-Italic.ttf,
%  Contextuals=Alternate,
%  StylisticSet=10,
%  Numbers={Lowercase,Proportional},
%  UprightFeatures={
%    SizeFeatures={
%      {Size={-8.5},      RawFeature={axis={wght=550,wdth=120}}},
%      {Size={8.6-10.99}, RawFeature={axis={wght=475,wdth=115}}},
%      {Size={11-21.59},  RawFeature={axis={wght=400,wdth=112.5}}},
%      {Size={21.6-},     RawFeature={axis={wght=351,wdth=100}}}
%    },
%  },
%  ItalicFeatures={
%    SizeFeatures={
%      {Size={-8.5},      RawFeature={axis={wght=550,wdth=118}}},
%      {Size={8.6-10.99}, RawFeature={axis={wght=475,wdth=114}}},
%      {Size={11-21.59},  RawFeature={axis={wght=400,wdth=111}}},
%      {Size={21.6-},     RawFeature={axis={wght=352,wdth=98}}}
%    },
%  },
%  BoldFeatures={
%    SizeFeatures={
%      {Size={-8.5},      RawFeature={axis={wght=700,wdth=120}}},
%      {Size={8.6-10.99}, RawFeature={axis={wght=700,wdth=115}}},
%      {Size={11-21.59},  RawFeature={axis={wght=650,wdth=112.5}}},
%      {Size={21.6-},     RawFeature={axis={wght=600,wdth=100}}}
%    },
%  },
%  BoldItalicFeatures={
%    SizeFeatures={
%      {Size={-8.5},      RawFeature={axis={wght=700,wdth=118}}},
%      {Size={8.6-10.99}, RawFeature={axis={wght=700,wdth=114}}},
%      {Size={11-21.59},  RawFeature={axis={wght=650,wdth=111}}},
%      {Size={21.6-},     RawFeature={axis={wght=600,wdth=98}}}
%    },
%  },
%]
\newfontface\regular{Junicode Two Beta}
%\setfontface\regular{JunicodeTwoBetaVF-Roman.ttf}[
%\setfontface\regular{Junicode Two Beta VF}[
%  SizeFeatures={
%    {Size={-8.5},      RawFeature={axis={wght=400,wdth=100}}},
%    {Size={8.5-11},    RawFeature={axis={wght=400,wdth=100}}},
%    {Size={11-22},     RawFeature={axis={wght=400,wdth=100}}},
%    {Size={22-},       RawFeature={axis={wght=400,wdth=100}}}
%  },
%]
\newfontface\narrow{Junicode Two Beta Condensed}
%\setfontface\narrow{JunicodeTwoBetaVF-Roman.ttf}[
%  SizeFeatures={
%    {Size={-8.5},      RawFeature={axis={wght=550,wdth=75}}},
%    {Size={8.5-11},    RawFeature={axis={wght=475,wdth=75}}},
%    {Size={11-22},     RawFeature={axis={wght=400,wdth=75}}},
%    {Size={22-},       RawFeature={axis={wght=353,wdth=75}}}
%  },
%]
\newfontface\seminarrow{Junicode Two Beta SemiCondensed}
%\setfontface\seminarrow{JunicodeTwoBetaVF-Roman.ttf}[
%  SizeFeatures={
%    {Size={-8.5},      RawFeature={axis={wght=550,wdth=87.5}}},
%    {Size={8.5-11},    RawFeature={axis={wght=475,wdth=87.5}}},
%    {Size={11-22},     RawFeature={axis={wght=400,wdth=87.5}}},
%    {Size={22-},       RawFeature={axis={wght=354,wdth=87.5}}}
%  },
%]
\newfontface\seminarrowlight{Junicode Two Beta SemiCondensed Light}
%\setfontface\seminarrowlight{JunicodeTwoBetaVF-Roman.ttf}[
%  SizeFeatures={
%    {Size={-8.5},      RawFeature={axis={wght=355,wdth=87.5}}},
%    {Size={8.5-11},    RawFeature={axis={wght=325,wdth=87.5}}},
%    {Size={11-22},     RawFeature={axis={wght=301,wdth=87.5}}},
%    {Size={22-},       RawFeature={axis={wght=301,wdth=87.5}}}
%  },
%]
\newfontface\semiwide{Junicode Two Beta SemiExpanded}
%\setfontface\semiwide{JunicodeTwoBetaVF-Roman.ttf}[
%  SizeFeatures={
%    {Size={-8.5},      RawFeature={axis={wght=550,wdth=112.5}}},
%    {Size={8.5-11},    RawFeature={axis={wght=475,wdth=112.5}}},
%    {Size={11-22},     RawFeature={axis={wght=400,wdth=112.5}}},
%    {Size={22-},       RawFeature={axis={wght=356,wdth=112.5}}}
%  },
%]
\newfontface\wide{Junicode Two Beta Expanded}
%\setfontface\wide{JunicodeTwoBetaVF-Roman.ttf}[
%  SizeFeatures={
%    {Size={-8.5},      RawFeature={axis={wght=550,wdth=125}}},
%    {Size={8.5-11},    RawFeature={axis={wght=475,wdth=125}}},
%    {Size={11-22},     RawFeature={axis={wght=400,wdth=125}}},
%    {Size={22-},       RawFeature={axis={wght=357,wdth=125}}}
%  },
%]
\newfontface\light{Junicode Two Beta Light}
%\setfontface\light{JunicodeTwoBetaVF-Roman.ttf}[
%  SizeFeatures={
%    {Size={-8.5},      RawFeature={axis={wght=358,wdth=120}}},
%    {Size={8.5-11},    RawFeature={axis={wght=325,wdth=115}}},
%    {Size={11-22},     RawFeature={axis={wght=301,wdth=112.5}}},
%    {Size={22-},       RawFeature={axis={wght=301,wdth=100}}}
%  },
%]
\newfontface\medium{Junicode Two Beta SemiExpanded Medium}
%\setfontface\medium{JunicodeTwoBetaVF-Roman.ttf}[
%  SizeFeatures={
%    {Size={-8.5},      RawFeature={axis={wght=550,wdth=120}}},
%    {Size={8.5-11},    RawFeature={axis={wght=525,wdth=115}}},
%    {Size={11-22},     RawFeature={axis={wght=500,wdth=112.5}}},
%    {Size={22-},       RawFeature={axis={wght=500,wdth=100}}}
%  },
%]
\newfontface\semicondsemibold{Junicode Two Beta SemiCondensed Semibold}
%\setfontface\semicondsemibold{JunicodeTwoBetaVF-Roman.ttf}[
%  SizeFeatures={
%    {Size={-8.5},      RawFeature={axis={wght=650,wdth=87.5}}},
%    {Size={8.5-11},    RawFeature={axis={wght=625,wdth=87.5}}},
%    {Size={11-22},     RawFeature={axis={wght=600,wdth=87.5}}},
%    {Size={22-},       RawFeature={axis={wght=600,wdth=87.5}}}
%  },
%]
\newfontface\condmed{Junicode Two Beta Condensed Medium}
%\setfontface\condmed{JunicodeTwoBetaVF-Roman.ttf}[
%  SizeFeatures={
%    {Size={-8.5},      RawFeature={axis={wght=550,wdth=75}}},
%    {Size={8.5-11},    RawFeature={axis={wght=525,wdth=75}}},
%    {Size={11-22},     RawFeature={axis={wght=500,wdth=75}}},
%    {Size={22-},       RawFeature={axis={wght=500,wdth=75}}}
%  },
%]
\newfontface\stditalic{Junicode Two Beta Italic}
%\setfontface\stditalic{JunicodeTwoBetaVF-Italic .ttf}[
%  SizeFeatures={
%    {Size={-8.5},      RawFeature={axis={wght=400,wdth=100}}},
%    {Size={8.5-11},    RawFeature={axis={wght=400,wdth=100}}},
%    {Size={11-22},     RawFeature={axis={wght=400,wdth=100}}},
%    {Size={22-},       RawFeature={axis={wght=400,wdth=100}}}
%  },
%]
\newfontface\semiconditalic{Junicode Two Beta SemiCondensed Italic}
%\setfontface\semiconditalic{JunicodeTwoBetaVF-Italic.ttf}[
%  SizeFeatures={
%    {Size={-8.5},      RawFeature={axis={wght=550,wdth=100}}},
%    {Size={8.5-11},    RawFeature={axis={wght=475,wdth=100}}},
%    {Size={11-22},     RawFeature={axis={wght=400,wdth=100}}},
%    {Size={22-},       RawFeature={axis={wght=301,wdth=100}}}
%  },
%]
\setmonofont{SourceCodePro-Regular.ttf}[Scale=MatchLowercase,Numbers=Lowercase]
%\setmonofont{SourceCodeVariable-Roman.otf}[
%  Scale = MatchLowercase,
%  Numbers = Lowercase,
%  SizeFeatures={
%    {Size={-8},        RawFeature={axis={wght=500}}},
%    {Size={8-11},      RawFeature={axis={wght=450}}},
%    {Size={11-},       RawFeature={axis={wght=400}}}
%  }
%]

\usepackage{xcolor,colortbl}
\definecolor{BrickRed}{RGB}{146,18,6}
\definecolor{SlateGray}{RGB}{112,128,144}
\definecolor{GGOrange}{RGB}{240,74,6}
\definecolor{RViolet}{RGB}{70,18,87}
\definecolor{myRed}{rgb}{0.5,0,0}
\definecolor{myBlue}{rgb}{0,0,0.5}
\usepackage{multicol}
\usepackage{array}
\usepackage{supertabular}
\usepackage{hhline}
\usepackage{metalogo}
\usepackage{hyperref}
\hypersetup{pdftex, colorlinks=true, linkcolor=blue, citecolor=blue, filecolor=blue,%
  urlcolor=blue, pdftitle=, pdfauthor=, pdfsubject=, pdfkeywords=}
% Footnotes configuration
\makeatletter
\renewcommand\thefootnote{\arabic{footnote}}
\makeatother
% Text styles
\linespread{1.1}
\newcommand\textLetterExample[1]{\textrm{\textbf{\color{BrickRed}#1}}}
\newcommand\textUName[1]{\textsc{#1}}
\newcommand\textSourceText[1]{{\color{GGOrange}\texttt{#1}}}
\newcommand\cvc[1]{{\color{magenta}#1}}
\newcommand\textstyleEmphasis[1]{\textit{#1}}
\newcommand\textstyleEntityRef[1]{\textrm{#1}}
\newcommand{\cvd}[3][0]{{\addfontfeature{CharacterVariant=#2:#1}#3}}
\newcommand{\hlig}[1]{{\addfontfeature{Ligatures=Historic}#1}}
\newcommand{\sups}[1]{{\addfontfeature{VerticalPosition = Superior}#1}}
\newcommand{\subs}[1]{{\addfontfeature{VerticalPosition = Inferior}#1}}
\newcommand{\oprop}[1]{{\addfontfeature{Numbers={Lowercase,Proportional}}#1}}
\newcommand{\lprop}[1]{{\addfontfeature{Numbers={Uppercase,Proportional}}#1}}
\newcommand{\otab}[1]{{\addfontfeature{Numbers={Lowercase,Monospaced}}#1}}
\newcommand{\ltab}[1]{{\addfontfeature{Numbers={Uppercase,Monospaced}}#1}}
\newcommand{\ornm}[2][0]{{\addfontfeature{Ornament=#1}#2}}
\newcommand{\revthorn}[1]{{\addfontfeature{StylisticSet=1}#1}}
\newcommand{\grk}{\addfontfeature{Script=Greek,Language=Greek}}
\newcommand{\eng}{\addfontfeature{Language=English}}
\newcommand{\icel}{\addfontfeature{Language=Icelandic}}
\newcommand{\unic}[1]{{\addfontfeature{Numbers={Uppercase,Monospaced}}#1}}
%\newcommand{\colongs}{\addfontfeature{Language=English,RawFeature={+ss08;+calt}}}
\newcommand{\colongs}{\addfontfeature{Language=English,StylisticSet=8}}
\newcommand{\ltech}{Lua\kern-1.5pt\TeX}
\newopentypefeature{MyStyle}{insular}{+ss02}
\newopentypefeature{MyStyle}{altogonek}{+cv62}
\newopentypefeature{MyStyle}{mirrored}{+rtlm}
\newopentypefeature{MyStyle}{gothic}{+ss19}
\newopentypefeature{MyStyle}{contextualr}{+ss16}
\newopentypefeature{MyStyle}{contextuals}{+ss08}
\newopentypefeature{MyStyle}{IPA}{+ss03}
\newopentypefeature{MyStyle}{thornswap}{+ss01}
%\newopentypefeature{Ligatures}{histon}{+hlig}
\newopentypefeature{Ligatures}{histoff}{-hlig}
% Outline numbering
\setcounter{secnumdepth}{0}
\makeatletter
\newcommand\arraybslash{\let\\\@arraycr}
\makeatother
% Page layout (geometry)
\setlength\voffset{-1in}
\setlength\hoffset{-0.75in}
\setlength\topmargin{1in}
\setlength\oddsidemargin{1in}
\setlength\textheight{8.000001in}
\setlength\textwidth{6in}
\setlength\footskip{0.0cm}
\setlength\headheight{0.4in}
\setlength\headsep{0.2in}
% Footnote rule
\setlength{\skip\footins}{14pt}
\renewcommand\footnoterule{\vspace*{-0.0071in}\setlength\leftskip{0pt}\setlength\rightskip{0pt plus 1fil}\noindent\textcolor{black}{\rule{0.25\columnwidth}{0.0071in}}\vspace*{0.0398in}}
% Pages styles
\pagestyle{fancy}
\footskip = 30pt
\headsep = 30pt
\renewcommand{\headrule}{}
\fancyhead[L]{}
\fancyhead[C]{}
\fancyhead[R]{}
\fancyfoot[L]{}
\fancyfoot[C]{}
\fancyfoot[R]{}
\setlength\tabcolsep{1mm}
\renewcommand\arraystretch{1.3}
% Headers
\usepackage{sectsty}
\subsectionfont{\color{BrickRed}}
\sectionfont{\color{SlateGray}}
% List styles
\newcommand\liststyleLi{%
\renewcommand\labelitemi{{\textbullet}}
\renewcommand\labelitemii{{\textbullet}}
\renewcommand\labelitemiii{{\textbullet}}
\renewcommand\labelitemiv{{\textbullet}}
}
\newcommand\liststyleLii{%
\renewcommand\labelitemi{{\textbullet}}
\renewcommand\labelitemii{{\textbullet}}
\renewcommand\labelitemiii{{\textbullet}}
\renewcommand\labelitemiv{{\textbullet}}
}
\newcounter{Feature}
\renewcommand\theFeature{\arabic{Feature}}
\tolerance=1500
\widowpenalty=500
\clubpenalty=500
\frenchspacing
\raggedbottom
%
%
%
\begin{document}
\addfontfeature{Script=Latin,Language=English,Contextuals=Alternate}
\begin{titlepage}
\huge\noindent
{\color{myRed}}\\[5cm]
\Huge \centering {\color{myBlue}Junicode} \\[1cm]
\huge \centering the font for medievalists \\[1cm]
 \Huge\centering {\color{myRed}} \\[1cm]
 \huge\centering {\stditalic specimens and user manual} \\[1ex]
 \Large\centering{\regular for version 2}\\
\vfill
{\color{myRed}}
\end{titlepage}
\mainmatter

\chapter*{\color{RViolet}Contents}
\thispagestyle{plain}

\noindent\hyperlink{aboutj}{\bfseries\large About Junicode}\\[-0.25em]

\noindent\hyperlink{specimens}{\bfseries\large Specimens}\\[-0.25em]

\noindent\hyperlink{GettingStarted}{\bfseries\large Getting Started With Junicode}\\[-0.25em]

\noindent\hyperlink{FeatureReference}{\bfseries\large Feature Reference}\\[-0.25em]

\hyperlink{intro}{Introduction}

\begin{itemize}
\setlength\itemsep{-0.25em}
\item[A] \hyperlink{SectionA}{Case-Related Features}

\item[B] \hyperlink{SectionB}{Numbers and Sequencing}

\item[C] \hyperlink{SectionC}{Superscripts and Subscripts}

\item[D] \hyperlink{SectionD}{Ornaments}

\item[E] \hyperlink{SectionE}{Alphabetic Variants}

\item[F] \hyperlink{SectionF}{Greek}

\item[G] \hyperlink{SectionG}{Punctuation}

\item[H] \hyperlink{SectionH}{Abbreviations}

\item[I] \hyperlink{SectionI}{Combining Marks}

\item[J] \hyperlink{SectionJ}{Currency and Weights}

\item[K] \hyperlink{SectionK}{Gothic}

\item[L] \hyperlink{SectionL}{Runic}

\item[M] \hyperlink{SectionM}{Ligatures and Digraphs}

\item[N] \hyperlink{req}{Required Features}

\item[O] \hyperlink{nonmufi}{Non-MUFI Code Points}
\end{itemize}
%\selectlanguage{english}

\cleardoublepage
\hypertarget{aboutj}{}\chapter*{\color{RViolet}About Junicode}

{\large%
\noindent Junicode is modeled on the Pica Roman type
purchased by Oxford University in 1692 and
used to set the bulk of the Latin text of George Hickes,
{\itshape Linguarum vett. septentrionalium thesaurus
grammatico-criticus et archaeologicus} (Oxford, 1703–5). This massive two-volume folio
is not only a major work of scholarship on the languages and literatures
of northern Europe in the Middle Ages, but also a fine
example of the work of the Oxford Press at this
period: printed in multiple types (for every language had to
have its proper type) and lavishly
illustrated with engravings of manuscript pages, coins and
artifacts.

Junicode also includes two other typefaces from the \textit{Thesaurus}:
Pica Saxon, used to set passages in the Old English language,
and also a typeface reproducing the Gothic alphabet (“Gothic” here
being not the late medieval style, but rather
the earliest extensively attested Germanic language).
These were commissioned by the literary scholar
Francis\-cus Junius (1591–1677) and bequeathed by him to
the University. Examples of all these typefaces can be found
in {\itshape A Specimen of the
Several Sorts of Letter Given to the University by Dr. John Fell,
Sometime Lord Bishop of Oxford. To Which Is Added the Letter Given by
Mr. F. Junius} (Oxford, 1693).

Junicode has two distinct Greek faces. The first, newly designed to harmonize with the roman face, is
up\-right and modern. The other, accompanying the italic face, is based on type designed by Alexander
Wilson (1714–86) of Glasgow and used in numerous books published by
the Foulis Press, most notably the great Glasgow Homer of 1756–58.

The Junicode project began around 1998, when the developer began to revise his
older (early 1990s) “Junius” fonts for medievalists to take account of the Unicode
standard, then relatively new. The font’s name, a contraction of
“Junius Unicode,” was supposed to be a stopgap, serving until a more suitable name
could be found, but the name “Junicode” is now so well known that it can’t be
changed.\footnote{\ An effort to change the name to “JuniusX” produced
only confusion. If you find a font by the name JuniusX on a free font site,
that is nothing more than an early version of Junicode 2.}
The project has been active for its entire history, responding to frequent
requests from users and changes in font technology; the developer is now pushing
towards the release of Junicode version 2, an extended font family of five weights
and five widths, with both static and variable versions, which aims to
promote best practices in the presentation of medieval texts, especially in
the area of accessibility. This aspect of the font is explored in the
Introduction to the Feature Reference.

\textit{Users should be aware that Junicode 2 is a beta version in the roman face,
complete but unstable and subject to change. In the italic it is an alpha, about
ninety-five percent complete (as of March 2022) and more unstable than the
roman. Until the official release of version 2 (probably in late 2022), versions of Junicode
numbered \textsc{1.00N} should be used for projects that require
stability. The latest stable version can be downloaded} \href{https://github.com/psb1558/Junicode-font/releases}{here}.

}
\pagestyle{fancy}
\fancyhead[CE]{\scshape\color{myRed} {\addfontfeatures{Numbers=OldStyle}\thepage}\hspace{10pt}junicode}
\fancyhead[CO]{\scshape\color{myRed} {junicode}\hspace{10pt}{\addfontfeatures{Numbers=OldStyle}\thepage}}


\hypertarget{specimens}{}\chapter*{\color{RViolet}Specimens}

\subsection*{Old and Middle English}

{\noindent\regular\addfontfeature{Language=English}Wē æthrynon mid ūrum ārum þā ȳðan þæs dēopan wǣles; wē
ġesāwon ēac þā muntas ymbe þǣre sealtan sǣ strande, and wē mid
āðēnedum hræġle and ġesundfullum windum þǣr ġewīcodon on þām
ġemǣrum þǣre fæġerestan þēode. Þā ȳðan ġetācniað þisne dēopan
cræft, and þā muntas ġetācniað ēac þā miċelnyssa þisses cræftes. (Regular)}\\

\noindent{\semiwide\addfontfeature{Language=English} S{\scshape iþen} þe sege and þe assaut watz sesed at Troye,\\
Þe borȝ brittened and brent to brondez and askez,\\
Þe tulk þat þe trammes of tresoun þer wroȝt\\
Watz tried for his tricherie, þe trewest on erthe:\\
Hit watz Ennias þe athel, and his highe kynde,\\
Þat siþen depreced prouinces, and patrounes bicome\\
Welneȝe of al þe wele in þe west iles. (SemiExpanded)}\\

\noindent{\small\semiconditalic Apply the OpenType feature ss02 (Stylistic Set 2)
for insular letter-forms.}\\[1ex]
{\seminarrow\addfontfeature{StylisticSet=2,Ligatures=NoCommon,MyStyle=altogonek}
Her cynewulf benam sigebryht his rices \& westseaxna wiotan for
un\-ryht\-um dędū buton hamtúnscire \& he hæfde þa oþ he ofslog
þone aldormon þe hī lengest wunode \& hiene þa cynewulf on
andred adræfde \& ħ þær wunade oþ þæt hine án swán ofstang
æt pryfetesflodan \& he wręc þone aldormon cumbran \& se cynewulf
oft miclum gefeohtum feaht uuiþ bretwalū.} (SemiCondensed)

\subsection*{Old Icelandic}
{\small\semiconditalic\addfontfeature{Language=English} For Nordic shapes of þ and ð in an
English context, specify the appropriate language (e.g. Icelandic or Norwegian);
or apply the OpenType ss01 (Stylistic Set 1) feature.}\\[1ex]
{\icel\medium Um haustit sendi Mǫrðr Valgarðsson orð at Gunnarr myndi vera einn heimi, en
lið alt myndi vera niðri í eyjum at lúka heyverkum. Riðu þeir Gizurr Hvíti ok
Geirr Goði austr yfir ár, þegar þeir spurðu þat, ok austr yfir sanda til Hofs.
Þá sendu þeir orð Starkaði undir Þríhyrningi; ok fundusk þeir þar allir er at
Gunnari skyldu fara, ok réðu hversu at skyldi fara.} (SemiExpanded Medium)

\subsection*{Runic}
{\small\semiconditalic\addfontfeature{Language=English} Junicode has features
for automated transliteration of Latin letters into various runic systems.}\\[1ex]
{\wide ᚠᛁᛋᚳ ᚠᛚᚩᛞᚢ ᚪᚻᚩᚠ ᚩᚾ ᚠᛖᚱᚷᛖᚾᛒᛖᚱᛁᚷ ᚹᚪᚱᚦ ᚷᚪ᛬ᛇᚱᛁᚳ ᚷᚱᚩᚱᚾ ᚦᚨᚱ ᚻᛖ ᚩᚾ ᚷᚱᛖᚢᛏ ᚷᛁᛇᚹᚩᛗ
ᚻᚱᚩᚾᚨᛇ ᛒᚪᚾ\\
ᚱᚩᛗᚹᚪᛚᚢᛇ ᚪᚾᛞ ᚱᛖᚢᛗᚹᚪᛚᚢᛇ ᛏᚹᛟᚷᛖᚾ ᚷᛁᛒᚱᚩᚦᚫᚱ ᚪᚠᛟᛞᛞᚫ ᛞᛁᚫ ᚹᚣᛚᛁᚠ ᚩᚾ ᚱᚩᛗᚫ\linebreak[0]ᚳᚫᛇᛏᛁ᛬
ᚩᚦᛚᚫ ᚢᚾᚾᛖᚷ} (Expanded)

\subsection*{German}

{\narrow\addfontfeature{Language=English} Ich ſag üch aber / minen fründen / Foͤꝛchtēd üch nit voꝛ denen die den
lyb toͤdend / vnd darnach nichts habennd das ſy mer thuͤgind. Ich wil
üch aber zeigē voꝛ welchem ir üch \cvd[4]{12}{foͤꝛchten} ſollend. Foͤꝛchtend üch voꝛ
dem / der / nach dem er toͤdet hat / ouch macht hat zewerffen inn die
hell: ja ich ſag üch / voꝛ dem ſelben \cvd[4]{12}{foͤꝛchtēd} üch. Koufft man nit
fünff Sparen vm̄ zween pfennig} (Condensed)

\subsection*{Latin}

{\small\semiconditalic Junicode contains the most common Latin abbreviations,
  making it suitable for diplomatic editions of Latin texts.}\\[1ex]
{\addfontfeatures{Language=Latin,MyStyle=altogonek,MyStyle=contextualr}\light Adiuuanos dſ̄ ſalutariſ noſter \&
 ꝓpt̄ głam nominiſ tui dnē liƀanoſ· \& ꝓpitiuſ eſto peccatiſ noſtriſ
 ꝓpter nomen tuum· Ne forte dicant ingentib: ubi eſt dſ̄ eorum \&
  innoteſcat innationib: corā oculiſ nr̄iſ· Poſuerunt moſticina
  ſeruorū ruorū eſcaſ uolatilib: cęli carneſ ſcōꝝ tuoꝝ beſtiiſ tenice·
  Facti ſumꝰ ob\kern+0.2ptꝓbrium uiciniſ nr̄iſ·} (Light)

\subsection*{Gothic}

{\seminarrowlight jabai auk ƕas gasaiƕiþ þuk þana habandan kunþi in galiuge stada
anakumbjandan, niu miþwissei is siukis wis\-an\-dins timrjada du
galiugagudam gasaliþ matjan?  fraqistniþ auk sa unmahteiga ana
þeinamma witubnja broþar in þize Xristus gaswalt.  swaþ~þan
frawaurkjandans wiþra broþruns, slahandans ize gahugd siuka, du
Xristau fra\-waur\-keiþ.} (SemiCondensed Light)\\

{\noindent\small\semiconditalic Use ss19 to produce Gothic letters
  automatically from transliterated text.}\\[1ex]
{\addfontfeature{MyStyle=gothic}\bfseries jabai auk ƕas gasaiƕiþ þuk þana
  habandan kunþi in ga\-liuge stada anakumbjandan, niu miþwissei is
  siukis wis\-an\-dins timrjada du galiugagudam gasaliþ matjan?
  jabai auk ƕas gasaiƕiþ þuk þana habandan kunþi in
    ga\-liuge stada anakumbjandan, niu miþwissei is siukis
    wis\-an\-dins timrjada du galiugagudam gasaliþ matjan?} (SemiExpanded Bold)

  \subsection*{Sanskrit Transliteration}

\noindent{\semicondsemibold mānaṁ dvividhaṁ viṣayadvai vidyātśaktyaśaktitaḥ \\
     arthakriyāyāṁ keśadirnārtho ’narthādhimokṣataḥ\\[1ex]
sadr̥śāsadr̥śatvācca viṣayāviṣayatvataḥ \\
     śabdasyānyanimittānāṁ bhāve dhīsadasattvataḥ} (SemiCondensed Semibold)

\subsection*{International Phonetic Alphabet}
{\addfontfeature{MyStyle=IPA,MyStyle=thornswap}
{\regular hwɑn θɑt ɑːprɪl wiθ is ʃuːrəs soːtə θə drʊxt ɔf mɑrʧ hɑθ peːrsəd toː
θə roːte ɑnd bɑːðəd ɛvrɪ væɪn ɪn swɪʧ lɪkuːr ɔf hwɪʧ vɛrtɪu
ɛnʤɛndrəd ɪs θə fluːr hwɑn zɛfɪrʊs eːk wɪθ hɪs sweːtə bræːθ}} (Regular)

\subsection*{Greek}
{\regular\grk βίβλος
γενέσεως ἰησοῦ χριστοῦ υἱοῦ δαυὶδ
υἱοῦ ἀβραάμ.
ἀβραὰμ
ἐγέννησεν τὸν ἰσαάκ, ἰσαὰκ δὲ ἐγέννησεν
τὸν ἰακώβ, ἰακὼβ δὲ ἐγέννησεν τὸν
ἰούδαν καὶ τοὺς ἀδελφοὺς αὐτοῦ,
ἰούδας
δὲ ἐγέννησεν τὸν φάρες καὶ τὸν ζάρα
ἐκ τῆς θαμάρ, φάρες δὲ ἐγέννησεν τὸν
ἑσρώμ, ἑσρὼμ δὲ ἐγέννησεν τὸν ἀράμ,
ἀρὰμ
δὲ ἐγέννησεν τὸν ἀμιναδάβ, ἀμιναδὰβ
δὲ ἐγέννησεν τὸν ναασσών, ναασσὼν δὲ
ἐγέννησεν τὸν σαλμών,
σαλμὼν
δὲ ἐγέννησεν τὸν βόες ἐκ τῆς ῥαχά} (Regular)\\

\noindent\textit{\grk βίβλος
γενέσεως ἰησοῦ χριστοῦ υἱοῦ δαυὶδ
υἱοῦ ἀβραάμ.
ἀβραὰμ
ἐγέννησεν τὸν ἰσαάκ, ἰσαὰκ δὲ ἐγέννησεν
τὸν ἰακώβ, ἰακὼβ δὲ ἐγέννησεν τὸν
ἰούδαν καὶ τοὺς ἀδελφοὺς αὐτοῦ,
ἰούδας
δὲ ἐγέννησεν τὸν φάρες καὶ τὸν ζάρα
ἐκ τῆς θαμάρ, φάρες δὲ ἐγέννησεν τὸν
ἑσρώμ, ἑσρὼμ δὲ ἐγέννησεν τὸν ἀράμ,
ἀρὰμ
δὲ ἐγέννησεν τὸν ἀμιναδάβ, ἀμιναδὰβ
δὲ ἐγέννησεν τὸν ναασσών, ναασσὼν δὲ
ἐγέννησεν τὸν σαλμών,
σαλμὼν
δὲ ἐγέννησεν τὸν βόες ἐκ τῆς ῥαχά} (Italic)

\subsection*{Lithuanian}

{\small\semiconditalic Lithuanian poses several typographical challenges. Make sure
  Contextual Alternates (calt) is turned on; for i̇́, use i followed
  by combining dot accent (\unic{U+0307}) and acute (\unic{U+0301}).}\\[1ex]
{\wide\addfontfeature{Language=Lithuanian} Visa žemė turėjo vieną kalbą ir tuos pačius žodžius.  Kai žmonės
kėlėsi iš rytų, jie rado slėnį Šinaro krašte ir ten įsikūrė.  Vieni
kitiems sakė: Eime, pasidirbkime plytų ir jas išdekime. – Vietoj
akmens jie naudojo plytas, o vietoj kalkių – bitumą.  Eime, – jie
sakė, – pasistatykime miestą ir bokštą su dangų siekiančia viršūne ir
pasidarykime sau vardą, kad nebūtume išblaškyti po visą žemės veidą.} (Expanded)

\subsection*{Polish}
{\small\semiconditalic The default shape and position of ogonek in Junicode are suitable
for modern Polish. For the medieval Latin e-caudata, consider using
cv62.}\\[1ex]
{\condmed\addfontfeature{Language=Polish} Mieszkańcy całej ziemi mieli jedną mowę, czyli jednakowe słowa.  A
gdy wędrowali ze wschodu, napotkali równinę w kraju Szinear i tam
zamieszkali.  I mówili jeden do drugiego: Chodźcie, wyrabiajmy cegłę
i wypalmy ją w ogniu. A gdy już mieli cegłę zamiast kamieni i smołę
zamiast zaprawy murarskiej, rzekli: Chodźcie, zbudujemy sobie miasto
i wieżę, której wierzchołek będzie sięgał nieba, i w ten sposób
uczynimy sobie znak, abyśmy się nie rozproszyli po całej ziemi.} (Condensed Medium)

\subsection*{Fleurons}

{\small\semiconditalic Junicode contains a number of fleurons (floral
  ornaments) copied from a 1785 Caslon specimen book. Access
  these via the OpenType feature \hyperlink{SectionD}{ornm}. Fleurons have only one weight and
  width, and they are the same in roman and italic.}

\begin{center}
\huge    \\
 \\[0.7ex]
\\[0.7ex]
\\
 
\end{center}

\chapter*{\color{RViolet}Getting Started with Junicode}\hypertarget{GettingStarted}{}

When installing Junicode on your system, you must choose the kind of font that
will work best for you you
and which faces you are most likely to use. You can choose between TrueType
and Compact Font Format (or CFF), and between static and variable.\footnote{\ TrueType is a font format developed by
Apple Computer in the 1980s; it is still the most commonly used format. Compact Font Format
(or CFF, often inaccurately called “OpenType”) was developed by Adobe Systems.
Both TrueType and CFF fonts are capable of fully supporting the OpenType
standard, developed in the late 1990s by Microsoft and Apple, which dramatically expanded the
capabilities of computer fonts and continues to evolve. Static fonts are the ones
users are most familiar with, each font file having a single set of outlines
scalable to any size. By contrast, a single variable font file stores a set of
outlines that can morph in various ways—for example, becoming bolder or lighter,
narrower or wider.} Most users will prefer TrueType fonts, as
these will look best on computer screens, and static fonts, as these will work
best with programs like Microsoft Word. You can find static TrueType fonts in
the fonts/ttf folder.

Use the CFF version (fonts/otf) if you run into problems with TrueType (for example,
a few Mac printer drivers mess up the placement of diacritics when rendering
these fonts). Use the TrueType variable version if you're using Junicode in a
web page, and the CFF variable version if you are using
{\ltech} (see the document LuaTeX\_JunicodeVF.pdf for instructions and pointers).

Junicode has five weights and five widths, which are combined in many ways
for a total of twenty-two styles in
both roman and italic. It is not necessary to install all of these; in fact,
your life will be simplified (font menus easier to navigate) if you
make a selection. You will probably want the traditional Regular, Bold, Italic, and Bold
Italic fonts, but you should survey the styles displayed in the Specimen
section of this booklet, choose the ones that look best to you, and install
only those. A reasonable selection for many users will include the traditional four
styles for main text, several SemiExpanded styles for footnotes, and
SemiCondensed for titles.

With more than 4,700 characters, Junicode is a large font. Finding the things you
want in a collection that size can be a challenge, and then entering them in your
documents is another challenge. This document will help, but it
presupposes a certain amount of knowledge—for example, how to install a font in
Windows, Mac OS or Linux and how to install and use different kinds of software.

Medievalists will find the \href{https://bora.uib.no/bora-xmlui/handle/1956/10699}%
{\textit{MUFI Character Recommendation}}, version 4.0 (2015)
an essential supplement to this document. The \textit{Recommendation} lists
thousands of characters identified by the
Medieval Unicode Font Initiative as being of interest to medievalists. Junicode
contains all of these characters. There are two versions of the \textit{Recommendation}:
you will probably find the “Alphabetical Order” version most helpful.

From the \textit{MUFI Character Recommendation} or, alternatively, the Junicode
\textit{List of Encoded Characters} (packaged with the font) you can find out
the \textbf{code points}\footnote{\ A Unicode code point is a numerical identifier for a character.
It is generally expressed as a
four-digit hexadecimal (or base-16) number with a prefix of ``U+''. The letter
capital ``A,'' for example, is \unic{U+0041} (65
in decimal notation), and lowercase ``ȝ'' (Middle English yogh) is \unic{U+021D}.}
of the characters you need. These code points can be used to enter
characters in your documents when they cannot be typed on the keyboard.

To enter any Unicode character in a Windows application, type its four-digit
code, followed by Alt-X. To do the same in the Mac OS, first install and switch
to the “Unicode Hex Input” keyboard, then type the code while holding down the Option
key. In most Linux distributions you can enter a code by typing Shift-Control-U,
then the code followed by Return or Enter.

\textbf{Combining marks} (diacritics and certain abbreviation signs) can pose special problems for
medievalists. Unicode contains a great
many \textbf{precomposed characters} consisting of a base letter plus one or more marks.
If these are all you need you're fortunate---especially if they can be
typed on an international keyboard (not all can).\footnote{\ Both
Windows and the Mac OS come with international keyboards that make it easy to
type special letters and diacritics. To find out how to enable these, search
online for “Mac OS International Keyboard” or “Windows International Keyboard.”}
But medieval manuscripts frequently contain
combinations of base + mark that are not used in modern written languages.
For these, you'll have to
enter bases and marks separately.

To position a mark correctly over a base character, first enter the base,
followed by the mark or marks.
The sequence \textbf{m} + \textbf{◌ᷙ} (\unic{U+1DD9})
will make \textbf{mᷙ}; \textbf{y} + \textbf{◌̄} (\unic{U+0304}) + \textbf{◌̆} (\unic{U+0306}) will make \textbf{ȳ̆};
\textbf{e} + \textbf{◌̣} (\unic{U+0323}) + \textbf{◌ᷠ} (\unic{U+1DE0}) will make \textbf{ẹᷠ}.

More than sixteen hundred characters in Junicode can only be accessed via OpenType features—that is,
by way of the programming built into the font—and many others \textit{should} be
accessed that way for reasons explained in the Introduction
to the Feature Reference section of this document.

For example, some programs (including Microsoft Word) produce small caps by
scaling capitals down to approximately the height of lowercase letters.
These always look too thin and light.
But Junicode contains hundreds of \textsc{true small caps} designed to harmonize with
the surrounding text. These can only be accessed via the OpenType \textSourceText{smcp} feature,
which you can apply to a run of text much as you apply italic or bold styles:
select some text and then apply the feature.

Unfortunately, not every program supports OpenType features, and some that do
either support only a few or make them difficult to access. Programs
that support Junicode’s features fully include the free word processor
\href{https://www.libreoffice.org/}{LibreOffice Writer}, all major browsers
(Firefox, Chrome, Safari and Edge), and
the typesetting programs {\LuaLaTeX} and {\XeLaTeX}. Adobe InDesign supports
OpenType features only partially, but provides access to all of Junicode's characters
through its own interface.

Microsoft Word, unfortunately, provides only limited support for OpenType
features. It supports the \hyperlink{req}{Required Features} discussed below, and also
variant number forms and Stylistic Sets (though only one at a time). Many characters
(for example, \textsc{true small caps} and those accessible only via Character
Variant features) cannot be accessed at all. To activate Word's OpenType
support, you must open the “Font” dialog, click over to the “Advanced” tab,
and check the “Kerning” box. (Oddly, the “Kerning” box enables all other
OpenType features.) Then, in the same tab, select Standard Ligatures, Contextual
Alternates, and any other features you want.
OpenType features are best applied to character styles rather than
directly to text: this will
save you from having to perform this operation repeatedly.

It is also good to set the language properly for the text you're working on.
Programs like Word will automatically set the language to the default for your system. If you
change to a language other than your own for a passage (or even a single word),
you should set the language for that passage appropriately. This will unlock
a number of capabilities. For example, in Old and Middle English, Word and
other programs will use the English form of thorn and eth ({\eng þð}) instead of
the modern Icelandic ({\icel þð}), and in ancient
Greek you will be able to type accents after vowels instead of looking up
the codes for hundreds of polytonic vowel + accent combinations. But these and other capabilities
are only available when you set the language correctly.

In LibreOffice and InDesign you can set the language with a drop-down menu
in the “Character” dialog. In Word there is a separate “Language” dialog,
accessible from the “Tools” menu.

If you have questions about any aspect of Junicode,
post a query in the \href{https://github.com/psb1558/Junicode-font/discussions}%
{Junicode discussion forum}. If you notice a bug, please open an
\href{https://github.com/psb1558/Junicode-font/issues}{issue} at the font's
\href{https://github.com/psb1558/Junicode-font}{development site}. If you need
help with programming, subsetting or other tasks, contact the developer
directly.\footnote{\ b dot tarde at gmail dot com.}

\chapter*{\color{RViolet}Feature Reference}\hypertarget{FeatureReference}{}

\hypertarget{intro}{}\section{Introduction}
The OpenType features of Junicode version 2 and its variable counterpart (hereafter referred to together as
``Junicode'') have two purposes. One is to provide convenient access to the rich character set of the Medieval Unicode
Font Initiative (MUFI) recommendation. The other is to enable best practices in the presentation of medieval text,
promoting accessibility in electronic texts from PDFs to e-books to web pages.
%\thispagestyle{plain}

Each character in the MUFI recommendation has a code point associated with it: either the one
assigned by Unicode or, where the character is not recognized by Unicode, in the Private Use Area (PUA) of the Basic
Multilingual Plane, a block of codes, running from \unic{U+E000} to \unic{U+F8FF}, that are assigned no value by Unicode but instead
are available for font designers to use in any way they please.

The problem with PUA code points is precisely their lack of any value. Consider, as a point of comparison, the letter
\textLetterExample{a} (\unic{U+0061}). Your computer, your phone, and probably a good many other devices around the house
store a good bit of information about this \textLetterExample{a}: that it’s a letter in the Latin script, that
it’s lowercase, and that the uppercase equivalent is \textLetterExample{A} (\unic{U+0041}). All this information is
available to word processors, browsers, and other applications running on your computer.

Now suppose you're preparing an electronic text containing what MUFI calls \textUName{latin small letter neckless
a} (\textLetterExample{}). It is assigned to code point \unic{U+F215}, which belongs to the PUA. Beyond that, your
computer knows nothing about it: not that it is a variant of \textLetterExample{a}, or that it is lowercase, or a letter in the Latin
alphabet, or even a character in a language system. A screen reader cannot read, or even spell out, a word with \unic{U+F215}
in it; a search engine will not recognize the word as containing the letter \textLetterExample{a}.

Junicode offers the full range of MUFI characters---you can enter the PUA code points---but also a solution to the
problems posed by those code points. Think of an electronic text (a web page, perhaps, or a PDF) as having two layers:
an underlying text, stable and unchanging, and the displayed text, generated by software at the instant it is needed
and discarded when it is no longer on the screen. For greatest accessibility the underlying text should contain the
plain letter \textLetterExample{a} (\unic{U+0061}) along with markup indicating how it should be displayed. To generate
the displayed text, a program called a ``layout engine'' will (simplifying a bit here) read the markup and apply the
OpenType feature \textSourceText{cv02[5]}\footnote{\ Many OpenType features produce different outcomes depending on
an index passed to an application’s layout engine along with the feature tag. Different applications have different
ways of entering this index: consult your application’s documentation. Here, the index is recorded in brackets after
the feature tag. Users of fontspec (with {\XeLaTeX} or {\LuaTeX}) should also be aware that fontspec indexes start at zero
while OpenType indexes start at one. Therefore all index numbers listed in this document must be reduced by one for
use with fontspec.\par } to the underlying \textLetterExample{a}, bypassing the PUA code point, with the result that
readers see \textLetterExample{\cvd[4]{2}{a}}{}---the ``neckless a.'' And yet the letter will still register as
\textLetterExample{a} with search engines, screen readers, and so on.

This is the Junicode model for text display, but it is not peculiar to Junicode: it is widely considered to be the best
practice for displaying text using current font technology.

The full range of OpenType features listed in this document is supported by all major web browsers, LibreOffice, XeTeX,
LuaTeX, and (presumably) other document processing applications. All characters listed here are available in Adobe
InDesign, though that program supports only a selection of OpenType features. Microsoft Word, unfortunately, supports
only Stylistic Sets, ligatures (all but the standard ones in peculiar and probably useless combinations), number
variants, and the \hyperlink{req}{Required Features}. In terms of
OpenType support, Word is the most primitive of the major text processing applications.

Many MUFI characters cannot be produced by using the OpenType variants of Junicode. These characters fall into three
categories:

\liststyleLi
\begin{itemize}
\item Those with Unicode (non-PUA) code points. MUFI has done valuable work obtaining Unicode code points for medieval characters.
All such characters (those with hexadecimal codes that \textstyleEmphasis{do not} begin with \textLetterExample{E}
or \textLetterExample{F}) are presumed safe to use in accessible and searchable text. However, some of these are
covered by Junicode OpenType features for particular reasons.
\item Precomposed characters---those consisting of base character + one or more diacritics. For greatest accessibility,
these should be entered not as PUA code points, but rather as sequences consisting of base character + one or more
diacritics. For example, instead of MUFI \unic{U+E498} \textUName{latin small letter e with dot below and acute}, use
\textLetterExample{e} + \unic{U+0323} \textUName{combining dot below} + \unic{U+0301} \textUName{combining acute accent}:
\textLetterExample{ẹ́} (when applying combining marks, start with any marks below the character and work
downwards, then continue with any marks above the character and work upwards. For example, to make
\textLetterExample{ǭ̣́}, place characters in this order: \textLetterExample{o},
\textUName{combining ogonek} \unic{U+0328}, \textUName{combining dot below} \unic{U+0323}, \textUName{combining
macron} \unic{U+0304}, \textUName{combining acute} \unic{U+0301}). Some MUFI characters have marks in unconventional locations,
e.g. \textLetterExample{ȯ́} \textUName{latin small letter o with dot above and acute}, where the
acute appears beside the dot instead of above. This and other characters like it should still be entered as a sequence
of base character + marks (here \textLetterExample{o}, \textUName{combining dot above} \unic{U+0307},
\textUName{combining acute} \unic{U+0301}). Junicode will position the marks correctly.
\item Characters for which a base character (a Unicode character to which it can be linked) cannot be identified, or for
which there may be an inconsistency in the MUFI recommendation. These include:

\begin{itemize}
\item \textLetterExample{} \unic{U+E8AF}. This is a ligature of long \textLetterExample{s} and \textLetterExample{l} with stroke,
but there are no base characters with this style of stroke.
\item \textLetterExample{} \unic{U+EFD8} and \unic{U+EFD9}. MUFI lists these as ligatures (corresponding to the
historic ligatures \textLetterExample{\hlig{uuUU}}, but they cannot be treated as ligatures in the
font because a single diacritic is positioned over the glyphs as if they were digraphs like
\textLetterExample{ꜳꜲ}.
\item \textLetterExample{} \unic{U+EBE7} and \unic{U+EBE6}, for the same reason.
\item \textLetterExample{} \unic{U+F159} \textUName{latin abbreviation sign small de}. Neither a variant of
\textLetterExample{d} nor an eth (\textLetterExample{ð}), this character may be a candidate for Unicode
encoding.
\end{itemize}
\item Characters for which OpenType programming is not yet available. These will be added as they are located and
studied. [Check: \unic{U+EBF1}, and smcp version.]
\end{itemize}
These characters should be avoided, even if you are otherwise using MUFI’s PUA characters:

\liststyleLii
\begin{itemize}
\item \unic{U+F1C5} \textUName{combining curl high position}. Use \unic{U+1DCE} \textUName{combining ogonek above}. The
positioning problem mentioned in the MUFI recommendation is solved in Junicode (and, to be fair, many other fonts with
OpenType programming).
\item \unic{U+F1CA} \textUName{combining dot above high position}. Use \unic{U+0307} \textUName{combining dot above}. It
will be positioned correctly on any character.
\end{itemize}
\pagebreak
\hypertarget{SectionA}{}\section{A. Case-Related Features}
\subsection[1. smcp {}-- Small Capitals]{\stepcounter{Feature}{\theFeature}.
\textSourceText{smcp} -- Small Capitals}
Converts lowercase letters to small caps; also several symbols and combining marks. All lower- and uppercase pairs (with
exceptions noted below) have a small cap equivalent. Lowercase letters without matching caps may lack matching small
caps. fghij $\rightarrow $ \textsc{fghij}.

Note: Precomposed characters defined by MUFI in the Private Use Area have no small cap equivalents. Instead, compose
characters using combining diacritics, as outlined in the introduction. For example, \textSourceText{smcp} applied
to the sequence \textLetterExample{t} + \textUName{combining ogonek} (\unic{U+0328}) + \textUName{combining
acute} (\unic{U+0301}) will change \textLetterExample{t̨́} to \textLetterExample{\textsc{t̨́}}.

\subsection[2. c2sc {}-- Small Capitals from Capitals]{\stepcounter{Feature}{\theFeature}. \textSourceText{c2sc} --
Small Capitals from Capitals}
Use with \textSourceText{smcp} for all-small-cap text. ABCDE $\rightarrow $ {\addfontfeature{Letters = UppercaseSmallCaps}ABCDE}.

\subsection[3. pcap {}-- Petite Capitals]{\stepcounter{Feature}{\theFeature}. \textSourceText{pcap} -- Petite
Capitals}
Produces small caps in a smaller size than \textSourceText{smcp}. Use these when small caps have to be mixed with
lowercase letters. The whole of the basic Latin alphabet is covered, plus several other letters. klmno{\th}
$\rightarrow $ {\addfontfeature{Letters = PetiteCaps}klmno\th}.

\subsection[4. case {}-- Case{}-Sensitive Forms]{\stepcounter{Feature}{\theFeature}. \textSourceText{case} --
Case-Sensitive Forms}
Produces combining marks that harmonize with capital letters: {\addfontfeature{Letters=Uppercase}\v{R}, X̉}, etc. Use of this feature reduces the
likelihood that a combining mark will collide with a glyph in the line above. Some applications turn this
feature on automatically for runs of capitals, and precomposed characters
(e.g. \textLetterExample{É} \unic{U+00C9}, \textLetterExample{Ū} \unic{U+016A})
already use case-appropriate combining marks.
\pagebreak
\hypertarget{SectionB}{}\section{B. Numbers and Sequencing}
\subsection[4a. frac {}-- Fractions]{\stepcounter{Feature}{\theFeature}.
\textSourceText{frac} -- Fractions}
Applied to a slash and surrounding numbers, produces fractions with diagonal
slashes. 6/9 becomes {\addfontfeature{Fractions=On}6/9}, 16/91 becomes {\addfontfeature{Fractions=On}16/91}.

\subsection[4b. frac {}-- Numerators]{\stepcounter{Feature}{\theFeature}.
\textSourceText{numr} -- Numerators}
Changes numbers to those suitable for use on the left/upper side of fractions
with diagonal stroke (\unic{U+2044}). This can be used, with \textSourceText{dnom}, to manually construct
fractions, but for most users \textSourceText{frac} will be a better solution.

\subsection[4c. frac {}-- Denominators]{\stepcounter{Feature}{\theFeature}.
\textSourceText{dnom} -- Denominators}
Changes numbers to those suitable for use on the right/lower side of fractions
with diagonal stroke (\unic{U+2044}). This can be used, with \textSourceText{numr}, to manually construct
fractions, but for most users \textSourceText{frac} will be a better solution.

\subsection[5. nalt {}-- Alternate Annotation Forms]{\stepcounter{Feature}{\theFeature}.
\textSourceText{nalt} -- Alternate Annotation Forms}
Produces letters and numbers circled, in parenthesis, or followed by periods, as follows:

\textSourceText{nalt[1]}, circled letters or numbers: {\addfontfeature{Annotation=0}a b .~.~. z; 0 1 2 .~.~. 20}.

\textSourceText{nalt[2]}, letter or numbers in parentheses: {\addfontfeature{Annotation=1}a .~.~. z; 0 1 .~.~. 20}.

\textSourceText{nalt[3]}, double-circled numbers: {\addfontfeature{Annotation=2}0 1 .~.~. 10}.

\textSourceText{nalt[4]}, white numbers in black circles: {\addfontfeature{Annotation=3}0 1 2 3 . . . 20}.

\textSourceText{nalt[5]}, numbers followed by period: {\addfontfeature{Annotation=0}0 1 . . . 20}.

\noindent For enclosed figures 10 and higher, \textSourceText{rlig} (Required Ligatures) must also be enabled (as it should
be by default: see \hyperlink{req}{Required Features} below).

\subsection[6. tnum {}-- Tabular Figures]{\stepcounter{Feature}{\theFeature}. \textSourceText{tnum} -- Tabular
Figures}
Fixed-width figures: \ltab{0123456789} (default or with \textSourceText{lnum}), \otab{0123456789} (with
\textSourceText{onum}).

\subsection[7. onum {}-- Oldstyle Figures]{\stepcounter{Feature}{\theFeature}. \textSourceText{onum} -- Oldstyle
Figures}
Figures that harmonize with lowercase characters: \otab{0123456789} (default or with
\textSourceText{tnum}), \oprop{0123456789}
(with \textSourceText{pnum}). When combined with \textSourceText{pnum}, this feature also affects subscripts
and superscripts.

\subsection[8. pnum {}-- Proportional Figures]{\stepcounter{Feature}{\theFeature}. \textSourceText{pnum} --
Proportional Figures}
Proportionally spaced figures: \lprop{0123456789} (default or with \textSourceText{lnum}),
\oprop{0123456789} (with
\textSourceText{onum}). When combined with \textSourceText{onum}, this feature also affects subscripts and
superscripts. Most applications (including MS Word) with any support of OpenType features will support this feature and
\textSourceText{lnum} in such a way that you don't have to enter them manually.

\subsection[9. lnum {}-- Lining Figures]{\stepcounter{Feature}{\theFeature}. \textSourceText{lnum} -- Lining
Figures}
Figures in a uniform height, harmonizing with uppercase letters: \ltab{0123456789} (default or with
\textSourceText{tnum}), \lprop{0123456789} (with \textSourceText{pnum}).

\subsection[10. zero {}-- Slashed Zero]{\stepcounter{Feature}{\theFeature}. \textSourceText{zero} -- Slashed Zero}
Produces slashed zero in all number styles:
{\addfontfeature{Numbers=SlashedZero}\ltab{0} \otab{0} \lprop{0} \oprop{0}. Includes superscripts, subscripts, and
fractions formed with \textSourceText{frac}:
\sups{\ltab{0}~\oprop{0}}~\subs{\ltab{0}~\oprop{0}} \addfontfeature{Fractions=On} 10/30}.

\hypertarget{SectionC}{}\section{C. Superscripts and Subscripts}
\subsection[11. sups {}-- Superscripts]{\stepcounter{Feature}{\theFeature}.
\textSourceText{sups} -- Superscripts}
Produces superscript numbers and letters. Only affects lining tabular and oldstyle proportional figures. All lowercase
letters of the basic Latin alphabet are covered, and most uppercase letters: \sups{\ltab{0123} \oprop{4567} abcde ABDEG}. Wherever
superscripts are needed (e.g. for footnote numbers), use \textSourceText{sups} instead of the raised and scaled
characters generated by some programs. With sups: \sups{4567}. Scaled: \textsuperscript{4567}.

\subsection[12. subs {}-- Subscripts]{\stepcounter{Feature}{\theFeature}. \textSourceText{subs} -- Subscripts}
Produces subscript numbers. Only affects lining tabular (the default numbers) and oldstyle proportional figures (use
\textSourceText{pnum} and \textSourceText{onum}): \subs{\ltab{8901} \oprop{2345}}.

\hypertarget{SectionD}{}\section{D. Ornaments}
\subsection[13. ornm {}-- Ornaments]{\stepcounter{Feature}{\theFeature}.
\textSourceText{ornm} -- Ornaments}
Produces ornaments (fleurons) in either of two ways: as an indexed variant of the bullet character (\unic{U+2022}) or as
variants of a-z, A-C (all fleurons are available by either method):

As a variant of {\textbullet}: 1=\ornm{\textbullet}, 2=\ornm[1]{\textbullet}, 3=\ornm[2]{\textbullet}, 4=\ornm[3]{\textbullet}, etc., up to 29.

As a variant of a-z, A-C: e=\ornm{e}, f=\ornm{f}, g=\ornm{g}, h=\ornm{h}, etc.

\noindent The method with letters of the alphabet is easier, but the method with bullets will produce a more satisfactory result
when text is displayed in an environment where Junicode is not available or \textSourceText{ornm} is not
implemented.

\hypertarget{SectionE}{}\section{E. Alphabetic Variants}
\subsection[14. cv01{}--cv52 {}-- Basic Latin Variants]{\stepcounter{Feature}{\theFeature}.
\textSourceText{cv01-cv52} -- Basic Latin Variants}
These features also affect small cap (\textSourceText{smcp}) and underdotted (\textSourceText{ss07}) forms,
where available. Variants in \cvc{magenta} are also available via \textSourceText{ss06} “Enlarged Minuscules.”
Use the \textSourceText{cvNN} features instead of \textSourceText{ss06} when you want to substitute an
enlarged minuscule for a capital (or, less likely, a lowercase) letter everywhere in a text.

%\definecolor{LightBlueGray}{RGB}{212,218,218}
\begin{center}
%\tablefirsthead{}
\tablefirsthead{\hline
%\rowcolor{LightBlueGray}
\centering{\bfseries Variant of} &\
\centering{\bfseries cvNN} &
\centering\arraybslash{\bfseries Variants}\\}
%\tablehead{}
\tablehead{\hline
%\rowcolor{LightBlueGray}
\centering{\bfseries Variant of} &
\centering{\bfseries cvNN} &
\centering\arraybslash{\bfseries Variants}\\}
\tabletail{\hline}
\tablelasttail{}
\begin{supertabular}{|m{0.79135984in}|m{0.79135984in}|m{2.9212599in}|}
\hline
%\centering{\bfseries Variant of} &
%\centering{\bfseries cvNN} &
%\centering\arraybslash{\bfseries Variants}\\\hline
\color{black}\centering{A} &
\centering{cv01} &
1=\cvd{1}{A}, 2=\cvd[1]{1}{A}, 3=\cvd[2]{1}{A}, \cvc{4=\cvd[3]{1}{A}}\\\hline
\centering{a} &
\centering{cv02} &
{1=\cvd{2}{a}, 2=\cvd[1]{2}{a}, 3=\cvd[2]{2}{a}, 4=\cvd[3]{2}{a},
5=\cvd[4]{2}{a}, \cvc{6=\cvd[5]{2}{a}}}\\\hline
\centering{B} &
\centering{cv03} &
{\cvc{1=\cvd{3}{B}}}\\\hline
\centering{b} &
\centering{cv04} &
{\cvc{1=\cvd{4}{b}}}\\\hline
\centering{C} &
\centering{cv05} &
{1=\cvd{5}{C}, \cvc{2=\cvd[1]{5}{C}}}\\\hline
\centering{c} &
\centering{cv06} &
{1=\cvd{6}{c}, 2=\cvd[1]{6}{c}}\\\hline
\centering{D} &
\centering{cv07} &
{1=\cvd{7}{D}, \cvc{2=\cvd[1]{7}{D}}, \cvc{3=\cvd[2]{7}{D}}}\\\hline
\centering{d} &
\centering{cv08} &
{1=\cvd{8}{d}, 2=\cvd[1]{8}{d}, 3=\cvd[2]{8}{d}, \cvc{4=\cvd[3]{8}{d},
5=\cvd[4]{8}{d}} (for 1, see also ss02)}\\\hline
\centering{E} &
\centering{cv09} &
{1=\cvd{9}{E}, 2=\cvd[1]{9}{E}, \cvc{3=\cvd[2]{9}{E}}}\\\hline
\centering{e} &
\centering{cv10} &
{1=\cvd{10}{e}, 2=\cvd[1]{10}{e}, 3=\cvd[2]{10}{e}, \cvc{4=\cvd[3]{10}{e}}}\\\hline
\centering{F} &
\centering{cv11} &
{1=\cvd{11}{F},  \cvc{2=\cvd[1]{11}{F}, 3=\cvd[2]{11}{F}}}\\\hline
\centering{f} &
\centering{cv12} &
{1=\cvd{12}{f}, 2=\cvd[1]{12}{f}, 3=\cvd[2]{12}{f}, 4=\cvd[3]{12}{f}, 5=\cvd[4]{12}{f},
6=\cvd[5]{12}{f}, \cvc{7=\cvd[6]{12}{f}, 8=\cvd[7]{12}{f}}}\\\hline
\centering{G} &
\centering{cv13} &
{1=\cvd{13}{G}, 2=\cvd[1]{13}{G}, 3=\cvd[2]{13}{G}, \cvc{4=\cvd[3]{13}{G}}}\\\hline
\centering{g} &
\centering{cv14} &
{1=\cvd{14}{g}, 2=\cvd[1]{14}{g}, 3=\cvd[2]{14}{g}, 4=\cvd[3]{14}{g}, 5=\cvd[4]{14}{g},
6=\cvd[5]{14}{g}, 7=\cvd[6]{14}{g}, \cvc{8=\cvd[7]{14}{g}}}\\\hline
\centering{H} &
\centering{cv15} &
{1=\cvd{15}{H}, \cvc{2=\cvd[1]{15}{H}}}\\\hline
\centering{h} &
\centering{cv16} &
{1=\cvd{16}{h}, 2=\cvd[1]{16}{h}, \cvc{3=\cvd[2]{16}{h}}}\\\hline
\centering{I} &
\centering{cv17} &
{1=\cvd{17}{I}, 2=\cvd[1]{17}{I}, \cvc{3=\cvd[2]{17}{I}}}\\\hline
\centering{i} &
\centering{cv18} &
{1=\cvd{18}{i}, 2=\cvd[1]{18}{i}, 3=\cvd[2]{18}{i}, 4=\cvd[3]{18}{ii}, 5=\cvd[4]{18}{i}, \cvc{6=\cvd[5]{18}{i}}*}\\\hline
\centering{J} &
\centering{cv19} &
{1=\cvd{19}{J}, \cvc{2=\cvd[1]{19}{J}}}\\\hline
\centering{j} &
\centering{cv20} &
{1=\cvd{20}{j}, 2=\cvd[1]{20}{j}, 3=\cvd[2]{20}{j}, \cvc{4=\cvd[3]{20}{j}}}\\\hline
\centering{K} &
\centering{cv21} &
{\cvc{1=\cvd{21}{K}}}\\\hline
\centering{k} &
\centering{cv22} &
{1=\cvd{22}{k}, 2=\cvd[1]{22}{k}, 3=\cvd[2]{22}{k}, 4=\cvd[3]{22}{k}, \cvc{5=\cvd[4]{22}{k}}}\\\hline
\centering{L} &
\centering{cv23} &
{\cvc{1=\cvd{23}{L}}}\\\hline
\centering{l} &
\centering{cv24} &
{1=\cvd{24}{l}, \cvc{2=\cvd[1]{24}{l}}}\\\hline
\centering{M} &
\centering{cv25} &
{1=\cvd{25}{M}, 2=\cvd[1]{25}{M}, 3=\cvd[2]{25}{M}, \cvc{4=\cvd[3]{25}{M}}}\\\hline
\centering{m} &
\centering{cv26} &
{1=\cvd{26}{m}, 2=\cvd[1]{26}{m}, 3=\cvd[2]{26}{m}, \cvc{4=\cvd[3]{26}{m}}}\\\hline
\centering{N} &
\centering{cv27} &
{1=\cvd{27}{N}, \cvc{2=\cvd[1]{27}{N}}}\\\hline
\centering{n} &
\centering{cv28} &
{1=\cvd{28}{n}, 2=\cvd[1]{28}{n}, 3=\cvd[2]{28}{n}, 4=\cvd[3]{28}{n}, \cvc{5=\cvd[4]{28}{n}}}\\\hline
\centering{O} &
\centering{cv29} &
{1=\cvd{29}{O}, \cvc{2=\cvd[1]{29}{O}}}\\\hline
\centering{o} &
\centering{cv30} &
{1=\cvd{30}{o}, \cvc{2=\cvd[1]{30}{o}}}\\\hline
\centering{P} &
\centering{cv31} &
{1=\cvd{31}{P}, \cvc{2=\cvd[1]{31}{P}}}\\\hline
\centering{p} &
\centering{cv32} &
{\cvc{1=\cvd{32}{p}}}\\\hline
\centering{Q} &
\centering{cv33} &
{1=\cvd{33}{Q}, \cvc{2=\cvd[1]{33}{Q}}, 3=\cvd[2]{33}{Q}◌,
4=\cvd[3]{33}{Q}◌◌}\\\hline
\centering{q} &
\centering{cv34} &
{1=\cvd{34}{q}, \cvc{2=\cvd[1]{34}{q}}}\\\hline
\centering{R} &
\centering{cv35} &
{1=\cvd{35}{R}, 2=\cvd[1]{35}{R}, \cvc{3=\cvd[2]{35}{R}}}\\\hline
\centering{r} &
\centering{cv36} &
{1=\cvd{36}{r}, 2=\cvd[1]{36}{r}, 3=\cvd[2]{36}{r}, \cvc{4=\cvd[3]{36}{r}}}\\\hline
\centering{S} &
\centering{cv37} &
{1=\cvd{37}{S}, 2=\cvd[1]{37}{S}, \cvc{3=\cvd[2]{37}{S}}}\\\hline
\centering{s} &
\centering{cv38} &
{1=\cvd{38}{s}, 2=\cvd[1]{38}{s}, 3=\cvd[2]{38}{s}, 4=\cvd[3]{38}{s},
            5=\cvd[4]{38}{s}, 6=\cvd[5]{38}{s}, \cvc{7=\cvd[6]{38}{s}}}\\\hline
\centering{T} &
\centering{cv39} &
{1=\cvd{39}{T}, \cvc{2=\cvd[1]{39}{T}}}\\\hline
\centering{t} &
\centering{cv40} &
{1=\cvd{40}{t}, 2=\cvd[1]{40}{t}, \cvc{3=\cvd[2]{40}{t}}}\\\hline
\centering{U} &
\centering{cv41} &
{\cvc{1=\cvd{41}{U}}}\\\hline
\centering{u} &
\centering{cv42} &
{\cvc{1=\cvd{42}{u}}}\\\hline
\centering{V} &
\centering{cv43} &
{\cvc{1=\cvd{43}{V}}}\\\hline
\centering{v} &
\centering{cv44} &
{1=\cvd{44}{v}, 2=\cvd[1]{44}{v}, 3=\cvd[2]{44}{v}, 4=\cvd[3]{44}{v}, \cvc{5=\cvd[4]{44}{v}}}\\\hline
\centering{W} &
\centering{cv45} &
{\cvc{1=\cvd{45}{W}}}\\\hline
\centering{w} &
\centering{cv46} &
{\cvc{1=\cvd{46}{w}}}\\\hline
\centering{X} &
\centering{cv47} &
{\cvc{1=\cvd{47}{X}}}\\\hline
\centering{x} &
\centering{cv48} &
{1=\cvd{48}{x}, 2=\cvd[1]{48}{x}, 3=\cvd[2]{48}{x}, 4=\cvd[3]{48}{x}, \cvc{5=\cvd[4]{48}{x}}}\\\hline
\centering{Y} &
\centering{cv49} &
{1=\cvd{49}{Y}, \cvc{2=\cvd[1]{49}{Y}}}\\\hline
\centering{y} &
\centering{cv50} &
{1=\cvd{50}{y}, 2=\cvd[1]{50}{y}, \cvc{3=\cvd[2]{50}{y}}}\\\hline
\centering{Z} &
\centering{cv51} &
{1=\cvd{51}{Z}, \cvc{2=\cvd[1]{51}{Z}}}\\\hline
\centering{z} &
\centering{cv52} &
{1=\cvd{52}{z}, 2=\cvd[1]{52}{z}, \cvc{3=\cvd[2]{52}{z}}}\\\hline
\end{supertabular}
\end{center}

\noindent * \textSourceText{cv18[4]} changes ii to ij at the end of a word;
\textSourceText{cv18[5]} changess i to j at the end of a word whether another
i precedes or not. These variants are chiefly useful for roman numbers, but
also for Latin words ending in -ii. The j produced by this feature is
searchable as i.

\hypertarget{OtherLatin}{}\subsection[15. cv53{}-cv66 {}-- Other Latin Letters]{\stepcounter{Feature}{\theFeature}.
\textSourceText{cv53-cv66} -- Other Latin Letters}
Some features affect both upper- and lowercase forms. \textSourceText{cv62} also affects
combining \textLetterExample{e} with ogonek, accessible via \textSourceText{\hyperlink{ss10}{ss10}} with the
entity reference \textSourceText{\&\_eogo;}.
features.
\pagebreak
\begin{center}
\tablefirsthead{\hline
%\rowcolor{LightBlueGray}
\centering{\bfseries Variant of} &
\centering{\bfseries cvNN} &
\centering\arraybslash{\bfseries Variants}\\}
\tablehead{\hline
%\rowcolor{LightBlueGray}
\centering{\bfseries Variant of} &
\centering{\bfseries cvNN} &
\centering\arraybslash{\bfseries Variants}\\}
\tabletail{\hline}
\tablelasttail{}
\begin{supertabular}{|m{1.8913599in}|m{0.79135984in}|m{1.8212599in}|}
\hline
\centering \k{A} (U+0104) &
\centering cv53 &
{1=\cvd{53}{Ą}, 2=\cvd[1]{53}{Ą}, 3=\cvd[2]{53}{Ą}}\\\hline
\centering \k{a} (U+0105) &
\centering cv54 &
{1=\cvd{54}{ą}, 2=\cvd[1]{54}{ą}}\\\hline
\centering ꜳ (U+A733) &
\centering cv55 &
{1=\cvd{55}{ꜳ}, 2=\cvd[1]{55}{ꜳ}, \cvc{3=\cvd[2]{55}{ꜳ}}}\\\hline
\centering  {\AE} (U+00C6) &
\centering cv56 &
{1=\cvd{56}{\AE}, \cvc{2=\cvd[1]{56}{\AE}}}\\\hline
\centering {\ae} (U+00E6) &
\centering cv57 &
{1=\cvd{57}{\ae}, \cvc{2=\cvd[1]{57}{\ae}}, 3=\cvd[2]{57}{\ae}, 4=\cvd[3]{57}{\ae}}\\\hline
\centering Ꜵ (U+A734) &
\centering cv58 &
{1=\cvd{58}{Ꜵ}, \cvc{2=\cvd[1]{58}{Ꜵ}, 3=\cvd[2]{58}{Ꜵ}}}\\\hline
\centering ꜵ (U+A735) &
\centering cv59 &
{1=\cvd{59}{ꜵ}, 2=\cvd[1]{59}{ꜵ}, \cvc{3=\cvd[2]{59}{ꜵ}}}\\\hline
\centering ꜹ (U+A739) &
\centering cv60 &
{1=\cvd{60}{ꜹ}}\\\hline
\centering {\dj} (U+0111) &
\centering cv61 &
{1=\cvd{61}{\dj}}\\\hline
\centering {\narrow Ę, ę ... (U+0118, U+0119)} &
\centering cv62 &
{1=\cvd{62}{Ę, ę ...}; 2=\cvd[1]{62}{Ę, ę ...}}\\\hline
\centering {\narrow Ȝ, ȝ (U+021C, U+021D)} &
\centering cv63 &
{1=\cvd{63}{Ȝ, ȝ}}\\\hline
\centering ꝉ (U+A749) &
\centering cv64 &
{1=\cvd{64}{ꝉ}}\\\hline
\centering {\o} (U+00F8) &
\centering cv65 &
{1=\cvd{65}{\o}, 2=\cvd[1]{65}{\o}, 3=\cvd[2]{65}{\o}, 4=\cvd[3]{65}{\o}}\\\hline
\centering ꝥ, \revthorn{ꝥ} (U+A765) &
\centering cv66 &
{1=\cvd{66}{ꝥ, \revthorn{ꝥ}}}\\\hline
\end{supertabular}
\end{center}
\subsection[16. ss01 {}-- Alternate thorn and eth]{\stepcounter{Feature}{\theFeature}. \textSourceText{ss01} --
Alternate thorn and eth}
Produces Nordic thorn and eth (\revthorn{{\th}{\dh}{\TH}}) when the language is English, and English thorn and eth
({\icel\revthorn{{\th}{\dh}{\TH}}}) with any other language, reversing the font’s ordinary usage.
This also affects small caps, crossed
thorn (ꝥ \revthorn{ꝥ}---see also
\hyperlink{OtherLatin}{\textSourceText{cv66}}), combining mark eth
(\unic{U+1DD9}, ◌ᷙ \revthorn{◌ᷙ}), and enlarged thorn and eth (see \textSourceText{\hyperlink{ss06}{ss06}}).
This feature depends on \textSourceText{\hyperlink{req}{loca}} (Localized Forms), which in most applications will
always be enabled.

\subsection[17. ss02 {}-- Insular Letter{}-Forms]{\stepcounter{Feature}{\theFeature}. \textSourceText{ss02} --
Insular Letter-Forms}
Produces insular letter-forms, e.g. {\addfontfeature{StylisticSet=2}dfgrsw}. Does not affect capitals (except W), as these do not not commonly have
insular shapes in early manuscripts. For these, enter the Unicode code points or use the Character Variant
(\textSourceText{cvNN}) features.

\subsection[18. ss04 {}-- High Overline]{\stepcounter{Feature}{\theFeature}. \textSourceText{ss04} -- High
Overline}
Produces a high overline over letters used as roman numbers: {\addfontfeature{StylisticSet=4}cdijlmvx CDIJLMVXↃ}.

\subsection[19. ss05 {}-- Medium{}-High Overline]{\stepcounter{Feature}{\theFeature}. \textSourceText{ss05} --
Medium-High Overline}
Produces a medium-high overline over (or through the ascenders of) letters used as roman numbers, and some others as
well: {\addfontfeature{StylisticSet=5,Style=Historic}bcdhijklmſvx{\th}}.

\hypertarget{ss06}{}\subsection[20. ss06 {}-- Enlarged Minuscules]{\stepcounter{Feature}{\theFeature}. \textSourceText{ss06} --
Enlarged Minuscules}
Letters that are lowercase in form but uppercase in function, and between upper- and
lowercase in size. They are often used to begin sentences in medieval manuscripts:
{\addfontfeature{StylisticSet=6}abcdefg}. The feature
covers the whole of the basic Latin alphabet and a number of other letters that
occur at the beginnings of sentences: consult the MUFI recommendation for details.
Uppercase letters are also covered by this feature so that enlarged minuscules
can, if you like, be searched as capitals.

If you are using the variable version of the font (JunicodeVF), consider using the
\href{https://psb1558.github.io/Junicode-New/EnlargedAxis.html}{Enlarged axis} instead, for reasons of flexibility and
accessibility.

\subsection[21. ss07 {}-- Underdotted Text]{\stepcounter{Feature}{\theFeature}. \textSourceText{ss07} --
Underdotted Text}
Produces underdotted text (indicating deletion in medieval manuscripts) for many
letters (including
the whole of the basic Latin alphabet and a number of other letters), e.g.
{\addfontfeature{StylisticSet=7}abcdefg HIJKLM}. This also affects small
caps, e.g. \textsc{abcdef} $\rightarrow $ {\addfontfeature{StylisticSet=7}\textsc{abcdef}}.
For letters without corresponding underdotted forms (e.g. \unic{U+A751}, ꝑ),
use \unic{U+0323}, combining dot below (\hspace{0.2em}ꝑ̣).

\subsection[22. ss08 {}-- Contextual Long s]{\stepcounter{Feature}{\theFeature}. \textSourceText{ss08} --
Contextual Long s}
In English and French text only, varies \textLetterExample{s} and \textLetterExample{ſ} according to rules
followed by many early printers: {\colongs sports, essence, stormy, disheveled, transfusions, slyness, cliffside}. For this
feature to work properly, \textSourceText{calt} ``Contextual Alternates'' must also be enabled (as it should be by
default: see \hyperlink{req}{Required Features} below). This feature does not work in {\ltech}, except in harf mode.

\hypertarget{ss16}{}\subsection[23. ss16 {}-- Contextual r Rotunda]{\stepcounter{Feature}{\theFeature}. \textSourceText{ss16} --
Contextual r Rotunda}
Converts \textLetterExample{r} to \textLetterExample{ꝛ} (lowercase only) following the
most common rules of medieval manuscripts: {\addfontfeature{StylisticSet=16,Contextuals=Alternate}priest, firmer, frost, ornament}. For this feature to work properly,
\textSourceText{calt} ``Contextual Alternates'' must also be enabled (as it should be by default: see
\hyperlink{req}{Required Features} below).  This feature does not work in {\ltech}, except in harf mode.

\subsection[24. cv68 {}-- Variant of ʔ (U+0294, glottal stop)]{\stepcounter{Feature}{\theFeature}.
\textSourceText{cv68} -- Variant of ʔ (\unic{U+0294}, glottal stop)}
1=\cvd{68}{ʔ}.

\hypertarget{SectionF}{}\section{F. Greek}
Junicode has two distinct styles of Greek. In the roman face, it is upright and
modern, especially designed to harmonize with Junicode's Latin letters. In the
italic, it is slanted and old-style, being based on the eighteenth-century
Greek type designed by Alexander Wilson and used by the Foulis Press in
Glasgow. Both Greek styles include the full polytonic and monotonic character
sets: αβγδεζ \textit{αβγδεζ}.

To set Greek properly (especially polytonic text) requires that both \textSourceText{locl}
and \textSourceText{ccmp} be active, as they should be by default in most
text processing applications (but in MS Word they must be explicitly enabled
by checking the ``kerning'' box on the ``Advanced'' tab of the Font Dialog).

Modern monotonic Greek should be set using only characters from the Unicode “Greek
and Coptic” range (\unic{U+0370–U+03FF}). When monotonic text is set in all caps, Junicode
suppresses accents automatically (except in single-letter words, for which
you must substitute unaccented forms manually). This substitution is not
performed on text containing visually identical letters from the ``Greek Extended''
range (\unic{U+0F00–U+1FFF}).
Thus when setting polytonic Greek, one should use (for example) \textLetterExample{Ά}
(\unic{U+1FBB}), not \textLetterExample{Ά} (\unic{U+0386}),
though they look the same.

You can set polytonic Greek either by entering code points from the Greek
Extended range or by entering sequences of base characters and diacritics.
When using the latter method, you must first make sure the language for the
text in question (whether a single word, a short passage, or a complete
document) is set to Greek, and then enter characters in canonical order
(that is, the sequence defined by Unicode as equivalent to the composite
character). The order is as follows: 1. base character; 2. diacritics
positioned either above or in front of the base character, working from left
to right or bottom to top; 3. the \textit{ypogegrammeni} (\unic{U+0345}), or for
capitals, if you prefer, the \textit{prosgegrammeni} (\unic{U+1FBE}).\footnote{\ Some
applications will automatically reorder sequences of letters and accents,
sparing you the trouble of remembering the canonical order.}

For example, the sequence ω (\unic{U+03C9}) ◌̓ (\unic{U+0313}) ◌́ (\unic{U+0301}) ◌ͅ (\unic{U+0345})
produces \textLetterExample{\grk ᾤ}.
Substitute capital Ω (\unic{U+03A9}) and the result is
\textLetterExample{\grk ᾬ}. Note that in a number of applications the layout
engine will perform these substitutions before Junicode’s own programming is
invoked. If either the layout engine or Junicode fails to produce your
preferred result, try placing \unic{U+034F} \textUName{combining grapheme joiner} somewhere
in the sequence of combining marks---for example, before the \textit{ypogegrammeni}
to make \textLetterExample{\grk Ὤ͏ͅ}.

\subsection[26. ss03 {}-- Alternate Greek]{\stepcounter{Feature}{\theFeature}.
\textSourceText{ss03} -- Alternate Greek}
Provides alternate shapes of β γ θ π φ χ ω: {\addfontfeature{StylisticSet=3}β γ θ π φ χ ω}.
These are chiefly useful in linguistics, as they harmonize with IPA characters.

\hypertarget{SectionG}{}\section{G. Punctuation}
MUFI encodes nearly twenty marks of punctuation in the PUA. In Junicode these can be accessed in
either of two ways: all are indexed variants of \textLetterExample{.} (period), and all are associated with the Unicode marks of
punctuation they most resemble (but it should not be inferred that the medieval marks are semantically identical with
the Unicode marks, or that there is an etymological relationship between them). The first method will be easier for
most to use, but the second is more likely to yield acceptable fallbacks in environments where Junicode is not
available.

Marks with Unicode encoding are not included here, as they can safely be entered directly. In MUFI 4.0 several marks
have PUA encodings, but have since that release been assigned Unicode code points: \textit{paragraphus} (⹍
\unic{U+2E4D}), medieval comma (⹌~\unic{U+2E4C}), \textit{punctus elevatus} (⹎ \unic{U+2E4E}), \textit{virgula suspensiva}
(⹊ \unic{U+2E4A}), triple dagger (⹋ \unic{U+2E4B}).

\subsection[27. ss18 {}-- Old{}-Style Punctuation Spacing]{\stepcounter{Feature}{\theFeature}.
\textSourceText{ss18} -- Old-Style Punctuation Spacing}
Colons, semicolons, parentheses, quotation marks and several other glyphs are spaced as in early printed books.

\subsection[28. cv69 {}-- Variants of ⁊⹒ (U+204A / U+2E52, Tironian
nota)]{\stepcounter{Feature}{\theFeature}. \textSourceText{cv69} -- Variants of ⁊⹒
(\unic{U+204A / U+2E52}, Tironian nota)}
1=\cvd{69}{⁊⹒}, 2=\cvd[1]{69}{⁊⹒}.

\subsection[29. cv70 {}-- Variants of . (period)]{\stepcounter{Feature}{\theFeature}. \textSourceText{cv70} --
  Variants of . (period)}
1=\cvd{70}{.}, 2=\cvd[1]{70}{.}, 3=\cvd[2]{70}{.}, 4=\cvd[3]{70}{.}, 5=\cvd[4]{70}{.}, 6=\cvd[5]{70}{.},
7=\cvd[6]{70}{.}, 8=\cvd[7]{70}{.}, 9=\cvd[8]{70}{.}, 10=\cvd[9]{70}{.}, 11=\cvd[10]{70}{.}, 12=\cvd[11]{70}{.},
13=\cvd[12]{70}{.}, 14=\cvd[13]{70}{.}, 15=\cvd[14]{70}{.}, 16=\cvd[15]{70}{.}, 17=\cvd[16]{70}{.},
18=\cvd[17]{70}{.}, 19=\cvd[18]{70}{.}, 20=\cvd[19]{70}{.}. This
feature provides access to all non-Unicode MUFI punctuation marks. Some of them are available via other features (see
below).

\subsection[30. cv71 {}-- Variant of {\textperiodcentered} (U+00B7, middle dot)]{\stepcounter{Feature}{\theFeature}.
\textSourceText{cv71} -- Variant of {\textperiodcentered} (\unic{U+00B7}, middle dot)}
1=\cvd{71}{\textperiodcentered} (\textit{distinctio}).

\subsection[31. cv72 {}-- Variants of , (comma)]{\stepcounter{Feature}{\theFeature}. \textSourceText{cv72} --
Variants of , (comma)}
1=\cvd{72}{,}, 2=\cvd[1]{72}{,}.

\subsection[32. cv73 {}-- Variants of ; (semicolon)]{\stepcounter{Feature}{\theFeature}. \textSourceText{cv73} --
Variants of ; (semicolon)}
1=\cvd{73}{;} (\textit{punctus versus}), 2=\cvd[1]{73}{;}, 3=\cvd[2]{73}{;}, 4=\cvd[3]{73}{;}, 5=\cvd[4]{73}{;}.

\subsection[33. cv74 {}-- Variants of ⹎ (U+2E4E, punctus elevatus)]{\stepcounter{Feature}{\theFeature}.
\textSourceText{cv74} -- Variants of ⹎ (\unic{U+2E4E}, \textit{punctus elevatus})}
1=\cvd{74}{⹎}, 2=\cvd[1]{74}{⹎}, 3=\cvd[2]{74}{⹎}, 4=\cvd[3]{74}{⹎} (\textit{punctus flexus}).

\subsection[34. cv75 {}-- Variant of ! (exclamation mark)]{\stepcounter{Feature}{\theFeature}.
\textSourceText{cv7}\textSourceText{5} -- Variant of ! (exclamation mark)}
1=\cvd{75}{!} (\textit{punctus exclamativus}).

\subsection[35. cv76 {}-- Variants of ? (question mark)]{\stepcounter{Feature}{\theFeature}. \textSourceText{cv76}
-- Variants of ? (question mark)}
1=\cvd{76}{?}, 2=\cvd[1]{76}{?}, 3=\cvd[2]{76}{?}. Shapes of the \textit{punctus interrogativus}.

\subsection[36. cv77 {}-- Variant of \~{} (ASCII tilde)]{\stepcounter{Feature}{\theFeature}. \textSourceText{cv77}
-- Variant of \~{} (ASCII tilde)}
1=\cvd{77}{\~{}} (same as MUFI \unic{U+F1F9}, ``wavy line'').

\subsection[37. cv78 {}-- Variant of * (asterisk)]{\stepcounter{Feature}{\theFeature}. \textSourceText{cv78} --
Variant of * (asterisk)}
1=\cvd{78}{*}. MUFI defines \unic{U+F1EC} as a \textit{signe de renvoi}. Manuscripts employ a number of shapes (of which this is one) for
this purpose. Junicode defines it as a variant of the asterisk---the most common modern \textit{signe de renvoi}.

\subsection[38. cv79 {}-- Variants of / (slash)]{\stepcounter{Feature}{\theFeature}.
\textSourceText{cv7}\textSourceText{9} -- Variants of / (slash)}
1=\cvd{79}{/}, 2=\cvd[1]{79}{/}. The first of these is Unicode, \unic{U+2E4E}.

\hypertarget{SectionH}{}\section{H. Abbreviations}
\subsection[39. cv80 {}-- Variant of ꝝ (U+A75D, rum
abbreviation)]{\stepcounter{Feature}{\theFeature}. \textSourceText{cv80} -- Variant of ꝝ (\unic{U+A75D}, rum
abbreviation)}
1=\cvd{80}{ꝝ}.

\subsection[40. cv81 {}-- Variants of ◌͛ (U+035B, combining zigzag
above)]{\stepcounter{Feature}{\theFeature}. \textSourceText{cv81} -- Variants of ◌͛ (\unic{U+035B}, combining
zigzag above)}
1=\cvd{81}{◌͛}, 2=\cvd[1]{81}{◌͛}, 3=\cvd[2]{81}{◌͛}. Positioning of the zigzag can differ from that of other combining
marks, e.g. b͛, f͛, d͛. If \textSourceText{calt} ``Contextual Alternates'' is enabled (as it is by
default in most apps), variant forms of \textSourceText{cv81[2]} will be used with several letters, e.g. \cvd[1]{81}{d͛,
  f͛, k͛}. Enable \textSourceText{case} for forms that harmonize with capitals
({\addfontfeature{Letters=Uppercase}\cvd[1]{81}{A͛ B͛ C͛ D͛}}), \textSourceText{smcp} for forms that harmonize with small caps
  (\textsc{\cvd[1]{81}{e͛ f͛ g͛ h͛}}).

\subsection[41. cv82 {}-- Variants of spacing ꝰ (U+A770)]{\stepcounter{Feature}{\theFeature}.
\textSourceText{cv82} -- Variants of spacing ꝰ (\unic{U+A770})}
1=\cvd{82}{ꝰ}, 2=\cvd[1]{82}{ꝰ}. \textSourceText{cv82[1]} produces the baseline \textit{{}-us} abbreviation (same as MUFI
\unic{U+F1A6}). MUFI also has an uppercase baseline \textit{{}-us} abbreviation (\unic{U+F1A5}), but as there is no uppercase version
of \unic{U+A770} to pair it with, it is indexed separately here.

\subsection[42. cv83 {}-- Variants of ꝫ (U+A76B, {}``et{}'' abbreviation)]{\stepcounter{Feature}{\theFeature}.
\textSourceText{cv83} -- Variants of ꝫ (\unic{U+A76B}, ``et'' abbreviation)}
1=\cvd{83}{ꝫ}, 2=\cvd[1]{83}{ꝫ}. \textSourceText{[1]} is identical in shape to a
semicolon, but as it is semantically the same as \unic{U+A76B}, it is preferable to use that
character with this feature. \textSourceText{[2]} produces a subscript version of
the character, a common variant in printed books.

\hypertarget{SectionI}{}\section{I. Combining Marks}
  \hypertarget{cv84}{}\subsection[43. cv84 {}-- MUFI combining marks (variants of ◌̄
(U+0304)]{\stepcounter{Feature}{\theFeature}. \textSourceText{cv84} -- MUFI combining marks (variants of \unic{U+0304})}
MUFI encodes a number of combining marks in the PUA (with code points between \unic{E000} and \unic{F8FF}), but when these characters
are entered directly, they can interfere with searching and accessibility, and some important applications fail to
position them correctly over their base characters. To avoid these problems, enter \unic{U+0304} (◌̄, \textUName{combining
macron}) and apply \textSourceText{cv84}, with the appropriate index, to both mark and base character. This
collection of marks does not include any Unicode-encoded marks (from the ``Combining Diacritical Marks'' ranges), as
these can safely be entered directly. It does include three marks (\textSourceText{cv84[30]},
\textSourceText{[31]} and \textSourceText{[32]}) that lack MUFI code points but are used to form MUFI
characters, and three more (\textSourceText{[2]}, \textSourceText{[33]},
and \textSourceText{[34]}) that have no code points in Unicode or MUFI.

This feature may often appear to have no effect. When this happens it is because
an application replaced a sequence like \textLetterExample{a \unic{U+0304}} with a precomposed character
like \textLetterExample{ā} (\unic{U+0101}) before Junicode's OpenType programming had a chance to work.
This process is called normalization, and it usually has the effect of simplifying
text processing tasks, but can sometimes prevent the proper functioning of OpenType
features. To disable it, place the character \unic{U+034F} \textUName{combining
grapheme joiner} (don't waste time puzzling over the name) between the base
character and the combining mark (or the first combining mark). For example, to produce
the combination \textLetterExample{\cvd[1]{84}{u͏̄}}, enter \textLetterExample{u \unic{U+034F U+0304}}.
(Without \unic{U+034F}, you would get \textLetterExample{\cvd[1]{84}{ū}}).

These marks can sometimes be produced by other \textSourceText{cvNN} features, which may be preferable to
\textSourceText{cv84} as providing more suitable fallbacks for applications that do not support Character Variant
(\textSourceText{cvNN}) features.

For some marks with PUA code points, users may find it easier to use \hyperlink{ss10}{entities} than this feature.

These marks are not affected by most other features. This is to preserve flexibility, given the rule that the feature
that produces them must be applied to both the mark and the base character. For example, if you had to
apply \textSourceText{smcp} ``Small Caps'' to \textSourceText{U+1DE8} ◌͏ᷨ to get
\textSourceText{cv84[11]} \cvd[10]{84}{◌͏̄}, it would be impossible to produce the sequence
\textLetterExample{\cvd[10]{84}{na͏{\char"34F\char"304}a}}
(or the reverse case \textLetterExample{\textsc{na{\char"1DE8}a}})
with the diacritic properly positioned.

\begin{multicols}{4}
\color{BrickRed}1=\cvd{84}{◌͏̄}

2=\cvd[1]{84}{◌͏̄}

3=\cvd[2]{84}{◌͏̄}

4=\cvd[3]{84}{◌͏̄}

5=\cvd[4]{84}{◌͏̄}

6=\cvd[5]{84}{◌͏̄}

7=\cvd[6]{84}{◌͏̄}

8=\cvd[7]{84}{◌͏̄}

9=\cvd[8]{84}{◌͏̄}

10=\cvd[9]{84}{◌͏̄}

11=\cvd[10]{84}{◌͏̄}

12=\cvd[11]{84}{◌͏̄}

13=\cvd[12]{84}{◌͏̄}

14=\cvd[13]{84}{◌͏̄}

15=\cvd[14]{84}{◌͏̄}

16=\cvd[15]{84}{◌͏̄}

17=\cvd[16]{84}{◌͏̄}

18=\cvd[17]{84}{◌͏̄}

19=\cvd[18]{84}{◌͏̄}

20=\cvd[19]{84}{◌͏̄}

21=\cvd[20]{84}{◌͏̄}

22=\cvd[21]{84}{◌͏̄}

23=\cvd[22]{84}{◌͏̄}

24=\cvd[23]{84}{◌͏̄}

25=\cvd[24]{84}{◌͏̄}

26=\cvd[25]{84}{◌͏̄}

27=\cvd[26]{84}{◌͏̄}

28=\cvd[27]{84}{◌͏̄}

29=\cvd[28]{84}{◌͏̄}

30=\cvd[29]{84}{◌͏̄}

31=\cvd[30]{84}{◌͏̄}

32=\cvd[31]{84}{◌͏̄}

33=\cvd[32]{84}{◌͏̄}

34=\cvd[33]{84}{◌͏̄}
\end{multicols}

\hypertarget{cv67}{}\subsection[44. cv67 {}-- Spacing zigzag]{\stepcounter{Feature}{\theFeature}.
\textSourceText{cv67} -- Spacing zigzag (variant of \unic{U+00AF}, spacing macron)}
A spacing version of ◌͛ (\unic{U+035B}, combining zigzag) appears in John Hutchins,
\textit{The History and Antiquities of the County of Dorset} (London, 1774). It
is not in Unicode or MUFI. In the future this feature may be used, as necessary,
for other spacing marks of abbreviation.

\hypertarget{ss10}{}\subsection[44. ss10 {}-- Entity References for Combining Marks]{\stepcounter{Feature}{\theFeature}.
\textSourceText{ss10} -- Entity References for Combining Marks}
Instead of \textSourceText{\hyperlink{cv84}{cv84}} for representing non-Unicode combining marks, some users may
wish to use XML/HTML-style entities. When \textSourceText{ss10} is turned on (preferably for the entire
text), these entities appear as combining marks and are correctly positioned over base characters.
For example, the letter \textLetterExample{e} followed by
\textLetterExample{\&{\textcompwordmark}\_eogo;} will yield \textLetterExample{e\&\_eogo;}. An advantage of entities is that
they are (unlike the PUA code points or the indexes of \textSourceText{cv84}) mnemonic and thus easy to use.
A disadvantage is that
searches cannot ignore combining marks entered by this method as they can using the \textSourceText{cv84} method.
(Every method of entering non-Unicode combining marks has disadvantages: users should weigh these, choose a method,
and stick with it.)

If you use any of these entities in a work intended for print publication, you should call your publisher’s
attention to them, since they will likely have their own method of representing them.

\begin{multicols}{3}
\color{RViolet}
\&{\textcompwordmark}\_ansc; $\rightarrow $
\textstyleEntityRef{◌\&\_ansc;}

\&{\textcompwordmark}\_an; $\rightarrow $
\textstyleEntityRef{◌\&\_an;}

\&{\textcompwordmark}\_ar; $\rightarrow $
\textstyleEntityRef{◌\&\_ar;}

\&{\textcompwordmark}\_arsc; $\rightarrow $
\textstyleEntityRef{◌\&\_arsc;}

\&{\textcompwordmark}\_bsc; $\rightarrow $
\textstyleEntityRef{◌\&\_bsc;}

\&{\textcompwordmark}\_dsc; $\rightarrow $
\textstyleEntityRef{◌\&\_dsc;}

\&{\textcompwordmark}\_eogo; $\rightarrow $
\textstyleEntityRef{◌\&\_eogo;}

\&{\textcompwordmark}\_emac; $\rightarrow $
\textstyleEntityRef{◌\&\_emac;}

\&\_{\textcompwordmark}idotl; $\rightarrow $
\textstyleEntityRef{◌\&\_idotl;}

\&\_{\textcompwordmark}j; $\rightarrow $
\textstyleEntityRef{◌\&\_j;}

\&\_{\textcompwordmark}jdotl; $\rightarrow $
\textstyleEntityRef{◌\&\_jdotl;}

\&\_{\textcompwordmark}ksc; $\rightarrow $
\textstyleEntityRef{◌\&\_ksc;}

{\narrow\&\_{\textcompwordmark}munc; $\rightarrow $
\textstyleEntityRef{◌\&\_munc;}}

\&\_{\textcompwordmark}oogo; $\rightarrow $
\textstyleEntityRef{◌\&\_oogo;}

{\narrow\&\_{\textcompwordmark}oslash; $\rightarrow $
\textstyleEntityRef{◌\&\_oslash;}}

\&\_{\textcompwordmark}omac; $\rightarrow $
\textstyleEntityRef{◌\&\_omac;}

\&\_{\textcompwordmark}orr; $\rightarrow $
\textstyleEntityRef{◌\&\_orr;}

\&\_{\textcompwordmark}oru; $\rightarrow $
\textstyleEntityRef{◌\&\_oru;}

\&\_{\textcompwordmark}q; $\rightarrow $
\textstyleEntityRef{◌\&\_q;}

\&\_{\textcompwordmark}ru; $\rightarrow $
\textstyleEntityRef{◌\&\_ru;}

\&\_{\textcompwordmark}tsc; $\rightarrow $
\textstyleEntityRef{◌\&\_tsc;}

\&\_{\textcompwordmark}y; $\rightarrow $
\textstyleEntityRef{◌\&\_y;}

{\narrow\&\_{\textcompwordmark}thorn; $\rightarrow $
\textstyleEntityRef{◌\&\_thorn;}}
\end{multicols}

\subsection[45. ss20 {}-- Low Diacritics]{\stepcounter{Feature}{\theFeature}. ss20 -- Low Diacritics}
The MUFI recommendation includes a number of precomposed characters with base letters b, h, k, {\th}, ꝺ and {\dh}
and combining marks ◌ͣ (\unic{U+0363}), ◌ͤ (\unic{U+0364}), \cvd[14]{84}{◌͏̄}
(\unic{U+0304}\slash\textSourceText{cv84[15]}), ◌ͦ (\unic{U+0366}), ◌ͬ (\unic{U+036C}), ◌ᷢ (\unic{U+1DE2}),
◌ͭ (\unic{U+036D}), ◌ͮ (\unic{U+036E}), ◌ᷦ (\unic{U+1DE6}) and \cvd[18]{84}{◌͏̄}
(\unic{U+0304}/\textSourceText{cv84[19]}). Instead of being positioned above ascender height as usual (e.g.
\textLetterExample{hͣ}), the MUFI glyphs have the marks positioned above the x-height
(e.g. \textLetterExample{\addfontfeature{StylisticSet=20}hͣ}).
Using the MUFI code points for these precomposed glyphs can interfere with searching
and drastically reduce accessibility. Users of Junicode should instead use a sequence of base character + combining
mark, and apply \textSourceText{ss20} to the two glyphs. A variant shape of eth (\textLetterExample{{\dh}})
that accommodates the combining mark will be substituted for the normal base character (but this is not necessary for
the other base characters). Examples:
{\addfontfeature{StylisticSet=20}bͦ, ꝺᷦ, \cvd[14]{84}{h̄}, kͤ, {\th}ͭ, ðᷢ}.

\textSourceText{ss20} affects only the diacritics and base characters listed here; other combinations (e.g.
\textLetterExample{mͤ}, \textLetterExample{\'{h}}) are not affected. It will therefore probably be safe
to apply this feature to the whole text if it is needed anywhere.

\subsection[46. cv85 {}-- Variant of ◌ᷓ (U+1DD3, combining open a)]{\stepcounter{Feature}{\theFeature}.
\textSourceText{cv85} -- Variant of ◌ᷓ (U+1DD3, combining open a)}
1=\cvd{85}{◌ᷓ}.

\subsection[47. cv86 {}-- Variant of ◌ᷘ (U+1DD8, combining insular d)]{%
\stepcounter{Feature}{\theFeature}. \textSourceText{cv86} -- Variant of ◌ᷘ (\unic{U+1DD8}, combining insular
d)}
1=\cvd{86}{◌ᷘ}.

\subsection[48. cv87 {}-- Variant of ◌ᷣ (U+1DE3, combining r rotunda)]{\stepcounter{Feature}{\theFeature}.
\textSourceText{cv87} -- Variant of ◌ᷣ (\unic{U+1DE3}, combining r rotunda)}
1=\cvd{87}{◌ᷣ}.

\subsection[49. cv88 {}-- Variant of combining dieresis (U+0308)]{\stepcounter{Feature}{\theFeature}.
\textSourceText{cv8}\textSourceText{8} -- Variant of combining dieresis (\unic{U+0308})}
1=\cvd{88}{◌̈}. This also affects precomposed characters on which this variant dieresis may occur, e.g.
\textLetterExample{\cvd{88}{\"a}}.

\subsection[50. cv89 {}-- Variant of ◌̅ (U+0305, combining
overline)]{\stepcounter{Feature}{\theFeature}. \textSourceText{cv89} -- Variant of ◌̅ (\unic{U+0305},
combining overline)}
1=\cvd{89}{◌̅}.

\subsection[51. cv90 {}-- Variants of ◌͞◌ (\unic{U+035E}, combining double macron)]{\stepcounter{Feature}{\theFeature}.
\textSourceText{cv}\textSourceText{90} -- Variants of ◌͞◌ (\unic{U+035E}, combining double macron)}
1=\cvd{90}{◌͞◌}, 2=\cvd[1]{90}{◌͞◌}.

\subsection[53. cv92 {}-- Variant of breve below (U+032E)]{\stepcounter{Feature}{\theFeature}. cv92 -- Variant of breve
below (\unic{U+032E})}
1=\cvd{92}{◌◌̮◌}. Position the mark after the middle of three glyphs, and apply \textSourceText{cv92}
to both the mark and (at least) the middle glyph. This mark is not available via \textSourceText{cv84}.

\hypertarget{SectionJ}{}\section{J. Currency and Weights}
\subsection[54. cv93 {}-- Variants of {\textcurrency} (U+0044, generic currency
sign)]{\stepcounter{Feature}{\theFeature}. \textSourceText{cv93} -- Variants of {\textcurrency} (\unic{U+0044}, generic
currency sign)}

\begin{multicols}{4}
\color{RViolet}1=\cvd{93}{\textcurrency}

2=\cvd[1]{93}{\textcurrency}

3=\cvd[2]{93}{\textcurrency}

4=\cvd[3]{93}{\textcurrency}

5=\cvd[4]{93}{\textcurrency}

6=\cvd[5]{93}{\textcurrency}

7=\cvd[6]{93}{\textcurrency}

8=\cvd[7]{93}{\textcurrency}

9=\cvd[8]{93}{\textcurrency}

10=\cvd[9]{93}{\textcurrency}

11=\cvd[10]{93}{\textcurrency}

12=\cvd[11]{93}{\textcurrency}

13=\cvd[12]{93}{\textcurrency}

14=\cvd[13]{93}{\textcurrency}

15=\cvd[14]{93}{\textcurrency}

16=\cvd[15]{93}{\textcurrency}

17=\cvd[16]{93}{\textcurrency}

18=\cvd[17]{93}{\textcurrency}

19=\cvd[18]{93}{\textcurrency}

20=\cvd[19]{93}{\textcurrency}

21=\cvd[20]{93}{\textcurrency}

22=\cvd[21]{93}{\textcurrency}

23=\cvd[22]{93}{\textcurrency}

24=\cvd[23]{93}{\textcurrency}

25=\cvd[24]{93}{\textcurrency}

26=\cvd[25]{93}{\textcurrency}

27=\cvd[26]{93}{\textcurrency}
\end{multicols}

\noindent All of MUFI’s currency and weight symbols (those that do
not have Unicode code points) are gathered here, but some are also variants of other currency signs (see below).

\subsection[55. cv94 {}-- Variant of ℔ (U+2114)]{\stepcounter{Feature}{\theFeature}.
\textSourceText{cv9}\textSourceText{4} -- Variant of ℔ (\unic{U+2114})}
1=\cvd{94}{℔}. Same as MUFI \unic{U+F2EB} (French Libra sign).

\subsection[56. cv95 {}-- Variants of {\pounds} (U+00A3, British pound sign)]{\stepcounter{Feature}{\theFeature}.
\textSourceText{cv95} -- Variants of {\pounds} (\unic{U+00A3}, British pound sign)}
1=\cvd{95}{\pounds}, 2=\cvd[1]{95}{\pounds}, 3=\cvd[2]{95}{\pounds}, 4=\cvd[3]{95}{\pounds},
5=\cvd[4]{95}{\pounds}, 6=\cvd[5]{95}{\pounds}. Same as MUFI \unic{U+F2EA, F2EB, F2EC, F2ED,
F2EE, F2EF}, pound signs from various locales.

\subsection[57. cv96 {}-- Variant of ₰ (U+20B0, German penny sign)]{\stepcounter{Feature}{\theFeature}.
\textSourceText{cv96} -- Variant of ₰ (\unic{U+20B0}, German penny sign)}
1=\cvd{96}{₰}. Same as MUFI \unic{U+F2F5}.

\subsection[58. cv97 {}-- Variant of ƒ (U+0192, florin)]{\stepcounter{Feature}{\theFeature}.
\textSourceText{cv97} -- Variant of ƒ (\unic{U+0192}, florin)}
1=\cvd{97}{ƒ}. Same as MUFI \unic{U+F2E8}.

\subsection[59. cv98 {}-- Variant of ℥ (U+2125, Ounce sign)]{\stepcounter{Feature}{\theFeature}.
\textSourceText{cv98} -- Variant of ℥ (\unic{U+2125}, Ounce sign)}
1=\cvd{98}{℥}. Same as MUFI \unic{U+F2FD}, Script ounce sign.

\hypertarget{SectionK}{}\section{K. Gothic}
\subsection[60. ss19 {}-- Latin to Gothic Transliteration]{\stepcounter{Feature}{\theFeature}.
\textSourceText{ss19} -- Latin to Gothic Transliteration}
Produces Gothic letters from Latin: \revthorn{Warþ þan in dagans jainans} $\rightarrow $
{\addfontfeature{StylisticSet=19}Warþ þan in dagans
jainans}. In web pages, the letters will be searchable as their Latin equivalents.

\hypertarget{SectionL}{}\section[L. Runic]{K. Runic}
\subsection[61. ss12 {}-- Early English Futhorc]{\stepcounter{Feature}{\theFeature}.
\textSourceText{ss12} -- Early English Futhorc}
Changes Latin letters to their equivalents in the early English futhorc. Because of the variability of the runic
alphabet, this method of transliteration may not produce the result you want. In that case, it may be necessary to
manually edit the result. fisc flodu ahof $\rightarrow $ {\addfontfeature{StylisticSet=12}fisc flodu ahof}.

\subsection[62. ss13 {}-- Elder Futhark]{\stepcounter{Feature}{\theFeature}. \textSourceText{ss13} -- Elder
Futhark}
Changes Latin letters to their equivalents in the Elder Futhark. Because of the variability of the runic alphabet, this
method of transliteration may not produce the result you want. In that case, it may be necessary to manually edit the
result. ABCDEFG $\rightarrow $ {\addfontfeature{StylisticSet=13}ABCDEFG}.

\subsection[63. ss14 {}-- Younger Futhark]{\stepcounter{Feature}{\theFeature}. \textSourceText{ss14} -- Younger
Futhark}
Changes Latin letters to their equivalents in the Younger Futhark. Because of the variability of the runic alphabet,
this method of transliteration may not produce the result you want. In that case, it may be necessary to manually edit
the result. ABCDEFG $\rightarrow $ {\addfontfeature{StylisticSet=14}ABCDEFG}.

\subsection[64. ss15 {}-- Long Branch to Short Twig]{\stepcounter{Feature}{\theFeature}. \textSourceText{ss15} --
Long Branch to Short Twig}
In combination with \textSourceText{ss14}, converts long branch (the default for the Younger Futhark) to short twig runes:
{\addfontfeature{StylisticSet=14}{ABCDEFG $\rightarrow $
\addfontfeature{StylisticSet=15}ABCDEFG}}.

\subsection[65. rtlm {}-- Right to Left Mirrored Forms]{\stepcounter{Feature}{\theFeature}. \textSourceText{rtlm}
-- Right to Left Mirrored Forms}
Produces mirrored runes, e.g. {\addfontfeature{StylisticSet=12}ABCDEFG $\rightarrow $ \addfontfeature{MyStyle=mirrored}GFEDCBA}.
This feature cannot change the direction of text.

\hypertarget{SectionM}{}\section[M. Ligatures and Digraphs]{L. Ligatures and Digraphs}
\subsection[66. hlig {}-- Historic Ligatures]{\stepcounter{Feature}{\theFeature}.
\textSourceText{hlig} -- Historic Ligatures}

Produces ligatures for combinations that should not ordinarily be rendered as
ligatures in modern text.\footnote{\ Some
fonts define \textSourceText{hlig} differently, as including all ligatures in which at least one
element is an archaic character, e.g.
those involving long s (\textrm{ſ\hspace{0.2em}}). In Junicode, however, a
historic ligature is defined not by the characters it is composed of, but
rather by the join between them. If two characters (though modern) should not be joined except
in certain historic contexts, they form a historic ligature. If they should be
joined in all contexts (even if archaic), the ligature is not historic
and should be defined in \textSourceText{liga}.} Most of these are from the MUFI recommendation,
ranges B.1.1(b) and B.1.4. This feature does
not produce digraphs (which have a phonetic value), for which see
\textSourceText{\hyperlink{ss17}{ss17}}. The ligatures:
\addfontfeatures{Ligatures=Historic}

\begin{multicols}{5}
{\color[rgb]{0.38039216,0.09019608,0.16078432}
a{\textcompwordmark}f$\rightarrow $af}

{\color[rgb]{0.38039216,0.09019608,0.16078432}
a{\textcompwordmark}ꝼ$\rightarrow $aꝼ}

{\color[rgb]{0.38039216,0.09019608,0.16078432}
a{\textcompwordmark}g$\rightarrow $ag}

{\color[rgb]{0.38039216,0.09019608,0.16078432}
a{\textcompwordmark}l$\rightarrow $al}

{\color[rgb]{0.38039216,0.09019608,0.16078432}
a{\textcompwordmark}n$\rightarrow $an}

{\color[rgb]{0.38039216,0.09019608,0.16078432}
a{\textcompwordmark}N$\rightarrow $aN}

{\color[rgb]{0.38039216,0.09019608,0.16078432}
a{\textcompwordmark}p$\rightarrow $ap}

{\color[rgb]{0.38039216,0.09019608,0.16078432}
a{\textcompwordmark}r$\rightarrow $ar}

{\color[rgb]{0.38039216,0.09019608,0.16078432}
a{\textcompwordmark}R$\rightarrow $aR}

{\color[rgb]{0.38039216,0.09019608,0.16078432}
a{\textcompwordmark}{\th}$\rightarrow $a{\th}}

{\color[rgb]{0.38039216,0.09019608,0.16078432}
b{\textcompwordmark}b$\rightarrow $bb}

{\color[rgb]{0.38039216,0.09019608,0.16078432}
b{\textcompwordmark}g$\rightarrow $bg}

{\color[rgb]{0.38039216,0.09019608,0.16078432}
c{\textcompwordmark}h$\rightarrow $ch}

{\color[rgb]{0.38039216,0.09019608,0.16078432}
c{\textcompwordmark}k$\rightarrow $ck}

{\color[rgb]{0.38039216,0.09019608,0.16078432}
ꝺ{\textcompwordmark}ꝺ$\rightarrow $ꝺꝺ}

{\color[rgb]{0.38039216,0.09019608,0.16078432}
e{\textcompwordmark}y$\rightarrow $ey}

{\color[rgb]{0.38039216,0.09019608,0.16078432}
f{\textcompwordmark}ä$\rightarrow $fä}

{\color[rgb]{0.38039216,0.09019608,0.16078432}
g{\textcompwordmark}d$\rightarrow $gd}

{\color[rgb]{0.38039216,0.09019608,0.16078432}
g{\textcompwordmark}\revthorn{ð}$\rightarrow $\revthorn{gð}}

{\color[rgb]{0.38039216,0.09019608,0.16078432}
g{\textcompwordmark}ꝺ$\rightarrow $gꝺ}

{\color[rgb]{0.38039216,0.09019608,0.16078432}
g{\textcompwordmark}g$\rightarrow $gg}

{\color[rgb]{0.38039216,0.09019608,0.16078432}
\cvd[2]{14}{ɡ{\textcompwordmark}ɡ}$\rightarrow $ɡɡ}

{\color[rgb]{0.38039216,0.09019608,0.16078432}
g{\textcompwordmark}o$\rightarrow $go}

{\color[rgb]{0.38039216,0.09019608,0.16078432}
g{\textcompwordmark}p$\rightarrow $gp}

{\color[rgb]{0.38039216,0.09019608,0.16078432}
g{\textcompwordmark}r$\rightarrow $gr}

{\color[rgb]{0.38039216,0.09019608,0.16078432}
H{\textcompwordmark}r$\rightarrow $Hr}

{\color[rgb]{0.38039216,0.09019608,0.16078432}
h{\textcompwordmark}r$\rightarrow $hr}

{\color[rgb]{0.38039216,0.09019608,0.16078432}
h{\textcompwordmark}ſ$\rightarrow $hſ}

{\color[rgb]{0.38039216,0.09019608,0.16078432}
h{\textcompwordmark}ẝ$\rightarrow $hẝ}

{\color[rgb]{0.38039216,0.09019608,0.16078432}
k{\textcompwordmark}r$\rightarrow $kr}

{\color[rgb]{0.38039216,0.09019608,0.16078432}
k{\textcompwordmark}ſ$\rightarrow $kſ}

{\color[rgb]{0.38039216,0.09019608,0.16078432}
k{\textcompwordmark}ẝ$\rightarrow $kẝ}

{\color[rgb]{0.38039216,0.09019608,0.16078432}
l{\textcompwordmark}l$\rightarrow $ll}

{\color[rgb]{0.38039216,0.09019608,0.16078432}
\textsc{n}{\textcompwordmark}ſ$\rightarrow $\textsc{nſ}}

{\color[rgb]{0.38039216,0.09019608,0.16078432}
o{\textcompwordmark}c$\rightarrow $oc}

{\color[rgb]{0.38039216,0.09019608,0.16078432}
O{\textcompwordmark}Ꝛ$\rightarrow $OꝚ}

{\color[rgb]{0.38039216,0.09019608,0.16078432}
o{\textcompwordmark}ꝛ$\rightarrow $oꝛ}

{\color[rgb]{0.38039216,0.09019608,0.16078432}
O{\textcompwordmark}Ꝝ$\rightarrow $OꝜ}

{\color[rgb]{0.38039216,0.09019608,0.16078432}
o{\textcompwordmark}ꝝ$\rightarrow $oꝝ}

{\color[rgb]{0.38039216,0.09019608,0.16078432}
P{\textcompwordmark}P$\rightarrow $PP}

{\color[rgb]{0.38039216,0.09019608,0.16078432}
p{\textcompwordmark}p$\rightarrow $pp}

{\color[rgb]{0.38039216,0.09019608,0.16078432}
ꝓ{\textcompwordmark}p$\rightarrow $ꝓp}

{\color[rgb]{0.38039216,0.09019608,0.16078432}
P{\textcompwordmark}s$\rightarrow $Ps}

{\color[rgb]{0.38039216,0.09019608,0.16078432}
p{\textcompwordmark}s$\rightarrow $ps}

{\color[rgb]{0.38039216,0.09019608,0.16078432}
P{\textcompwordmark}si$\rightarrow $Psi}

{\color[rgb]{0.38039216,0.09019608,0.16078432}
p{\textcompwordmark}si$\rightarrow $psi}

{\color[rgb]{0.38039216,0.09019608,0.16078432}
q{\textcompwordmark}ꝩ$\rightarrow $qꝩ}

{\color[rgb]{0.38039216,0.09019608,0.16078432}
q{\textcompwordmark}ꝫ/q\cvd[1]{83}{ꝫ}$\rightarrow $qꝫ/\cvd[1]{83}{qꝫ}}

{\color[rgb]{0.38039216,0.09019608,0.16078432}
ꝗ{\textcompwordmark}ꝗ$\rightarrow $ꝗꝗ}

{\color[rgb]{0.38039216,0.09019608,0.16078432}
Q{\textcompwordmark}Ꝛ$\rightarrow $QꝚ}

{\color[rgb]{0.38039216,0.09019608,0.16078432}
q{\textcompwordmark}ꝛ$\rightarrow $qꝛ}

{\color[rgb]{0.38039216,0.09019608,0.16078432}
ſ{\textcompwordmark}\"a$\rightarrow $ſ\"a}

{\color[rgb]{0.38039216,0.09019608,0.16078432}
ſ{\textcompwordmark}c{\textcompwordmark}h$\rightarrow $ſch}

{\color[rgb]{0.38039216,0.09019608,0.16078432}
ſ{\textcompwordmark}t{\textcompwordmark}r$\rightarrow $ſtr}

{\color[rgb]{0.38039216,0.09019608,0.16078432}
ſ{\textcompwordmark}ꝩ$\rightarrow $ſꝩ}

{\color[rgb]{0.38039216,0.09019608,0.16078432}
ꞇ{\textcompwordmark}ꞇ$\rightarrow $ꞇꞇ}

{\color[rgb]{0.38039216,0.09019608,0.16078432}
U{\textcompwordmark}E$\rightarrow $UE}

{\color[rgb]{0.38039216,0.09019608,0.16078432}
u{\textcompwordmark}e$\rightarrow $ue}

{\color[rgb]{0.38039216,0.09019608,0.16078432}
U{\textcompwordmark}U$\rightarrow $UU}

{\color[rgb]{0.38039216,0.09019608,0.16078432}
u{\textcompwordmark}u$\rightarrow $uu}

{\color[rgb]{0.38039216,0.09019608,0.16078432}
ƿ{\textcompwordmark}ƿ$\rightarrow $ƿƿ}

{\color[rgb]{0.38039216,0.09019608,0.16078432}
\revthorn{{\th}{\textcompwordmark}r$\rightarrow ${\th}r}}

{\color[rgb]{0.38039216,0.09019608,0.16078432}
\revthorn{{\th}{\textcompwordmark}s$\rightarrow ${\th}s}}

{\color[rgb]{0.38039216,0.09019608,0.16078432}
\revthorn{{\th}{\textcompwordmark}ẝ$\rightarrow ${\th}ẝ}}

{\color[rgb]{0.38039216,0.09019608,0.16078432}
{\th}\textcompwordmark{\th}$\rightarrow ${\th}{\th}}
\end{multicols}

\noindent\addfontfeatures{Ligatures=histoff}
Note: For the ligature \textLetterExample{\textsc{\hlig{nſ}}} to
work properly, \unic{U+017F} \textLetterExample{ſ} must be entered directly, not by applying an OpenType feature to
\textLetterExample{s}.

\subsection[67. dlig {}-- Discretionary Ligatures]{\stepcounter{Feature}{\theFeature}. \textSourceText{dlig} --
Discretionary Ligatures}
Produces lesser-used ligatures, but also roman numbers, e.g.
{\addfontfeature{Ligatures=Rare}ii, II, xi, XI}. The lesser-used ligatures:
\textLetterExample{\textcolor[rgb]{0.5529412,0.15686275,0.11764706}{\addfontfeature{Ligatures=Rare}ct, ſp, str, st, tr, tt, ty}}.

\hypertarget{ss17}{}\subsection[68. ss17 {}-- Rare Digraphs]{\stepcounter{Feature}{\theFeature}. \textSourceText{ss17} -- Rare
Digraphs}
By ``digraph'' we mean conjoined letters that represent a phonetic value: the most common examples
for western languages are \textLetterExample{{\ae}} and \textLetterExample{{\oe}} (though these, because they
are so common, are not included in this feature). Use of this feature in web pages enables easier searches: for
example, producing \textLetterExample{\addfontfeature{StylisticSet=17}{\th}au} from
\textLetterExample{{\th}au} allows the word to be
searched as ``{\th}au.'' The digraphs covered by this feature are \textcolor[rgb]{0.5529412,0.15686275,0.11764706}{%
\addfontfeature{StylisticSet=17}aa, ao, au, av, ay, oo, vy,} plus capital and small cap equivalents and digraph + diacritic combinations anticipated in the
MUFI recommendation. To produce such a digraph + diacritic combination, either type a letter + diacritic combination as
the second element of the digraph or type the diacritic after the second element. For example,
\textLetterExample{a} + \textLetterExample{\'u} yields \textLetterExample{\addfontfeature{StylisticSet=17}a\'u}, and so does
\textLetterExample{a} + \textLetterExample{u} + \unic{U+0301} (combining acute accent). To produce a digraph +
diacritic combination not covered by MUFI (e.g. \textLetterExample{ꜵ̀}), you may have to place \unic{U+034F}
\textUName{combining grapheme joiner} (see \hyperlink{cv84}{cv84} above) between the second element of the digraph and the combining mark.
Without \unic{U+034F}: \textLetterExample{\addfontfeature{StylisticSet=17}aō}.
With \unic{U+034F}: \textLetterExample{\addfontfeature{StylisticSet=17}ao͏̄}.

\hypertarget{req}{}\section{N. Required Features}
Required features, which provide some of the font’s most basic functionality---ligatures, support for
other features, kerning, and more---include \textSourceText{ccmp} (Glyph Composition/Decomposition),
\textSourceText{calt} (Contextual Alternates), \textSourceText{liga} (Standard Ligatures),
\textSourceText{loca} (Localized Forms), \textSourceText{rlig} (Required Ligatures),
\textSourceText{kern} (Horizontal Kerning), and \textSourceText{mark}/\textSourceText{mkmk} (Mark
Positioning). In MS Word these features have to be explicitly enabled on the Advanced tab of the Font dialog (Ctrl-D or
Cmd-D: enable Kerning, Standard Ligatures, and Contextual Alternates, and the others will be enabled automatically),
but in most other applications they are enabled by default.


\hypertarget{nonmufi}{}\section{O. Non-MUFI Code Points}
Characters in Junicode that do not have Unicode code points should be accessed via OpenType
features whenever possible. MUFI/PUA code points should be used only in applications that do not support OpenType, or
that support it only partially (for example, MS Word). For certain characters that lack either Unicode or MUFI code
points, code points in the Supplementary Private Use Area-A (plane 15) are available.

\begin{multicols}{4}
{\color[rgb]{0.13333334,0.29411766,0.07058824}
U+F0000 󰀀}

{\color[rgb]{0.13333334,0.29411766,0.07058824}
U+F0001 󰀁}

{\color[rgb]{0.13333334,0.29411766,0.07058824}
U+F0002 󰀂}

{\color[rgb]{0.13333334,0.29411766,0.07058824}
U+F0003 󰀃}

{\color[rgb]{0.13333334,0.29411766,0.07058824}
U+F0004 󰀄}

{\color[rgb]{0.13333334,0.29411766,0.07058824}
U+F0005 󰀅}

{\color[rgb]{0.13333334,0.29411766,0.07058824}
U+F0006 󰀆}

{\color[rgb]{0.13333334,0.29411766,0.07058824}
U+F0007 󰀇}

{\color[rgb]{0.13333334,0.29411766,0.07058824}
U+F0008 󰀈}

{\color[rgb]{0.13333334,0.29411766,0.07058824}
U+F0009 󰀉}

{\color[rgb]{0.13333334,0.29411766,0.07058824}
U+F000A 󰀊}

{\color[rgb]{0.13333334,0.29411766,0.07058824}
U+F000B 󰀋}

{\color[rgb]{0.13333334,0.29411766,0.07058824}
U+F000C 󰀌}

{\color[rgb]{0.13333334,0.29411766,0.07058824}
U+F000D 󰀍}

{\color[rgb]{0.13333334,0.29411766,0.07058824}
U+F000E 󰀎}

{\color[rgb]{0.13333334,0.29411766,0.07058824}
U+F000F 󰀏}

{\color[rgb]{0.13333334,0.29411766,0.07058824}
U+F0010 󰀐}

{\color[rgb]{0.13333334,0.29411766,0.07058824}
U+F0011 󰀑}

{\color[rgb]{0.13333334,0.29411766,0.07058824}
U+F0012 󰀒}

{\color[rgb]{0.13333334,0.29411766,0.07058824}
U+F0013 󰀓}

{\color[rgb]{0.13333334,0.29411766,0.07058824}
U+F0014 󰀔}

{\color[rgb]{0.13333334,0.29411766,0.07058824}
U+F0015 󰀕}

{\color[rgb]{0.13333334,0.29411766,0.07058824}
U+F0016 󰀖}

{\color[rgb]{0.13333334,0.29411766,0.07058824}
U+F0017 󰀗}

{\color[rgb]{0.13333334,0.29411766,0.07058824}
U+F0018 󰀘}

{\color[rgb]{0.13333334,0.29411766,0.07058824}
U+F0019 󰀙}

{\color[rgb]{0.13333334,0.29411766,0.07058824}
U+F001A 󰀚}

{\color[rgb]{0.13333334,0.29411766,0.07058824}
U+F001B 󰀛}

{\color[rgb]{0.13333334,0.29411766,0.07058824}
U+F001C 󰀜}

{\color[rgb]{0.13333334,0.29411766,0.07058824}
U+F001D 󰀝}

{\color[rgb]{0.13333334,0.29411766,0.07058824}
U+F001E 󰀞}

{\color[rgb]{0.13333334,0.29411766,0.07058824}
U+F001F 󰀟}
\end{multicols}

\chapter*{Junicode on the Web}

If you are using Junicode on a web page, you should prefer the variable fonts
(those with “VF” in the family name and filename) to the static fonts. One
variable font file can do the work of many traditional font files. For example,
the \href{https://psb1558.github.io/Junicode-font/Junicode-2-feature-test.html}%
{Test of High-Level CSS Properties} web page displays Junicode in regular,
bold and semicondensed styles. It used
to be that your user would have to download three font files, one for each
style, but one variable font will display all three.

But you may be thinking, \textit{That font is big}! It's true: even the
compressed webfont (.woff2) is nearly a megabyte in size—enough to seriously
slow down the loading of a web page.

To solve this problem, you'll need to subset the font—that is, produce a copy
that contains only what you need. The subsetted font that downloads
with the property test web page
is approximately 275k in size—almost thirty percent of the
size of the full webfont. It's still a pretty big download, but that's because the page
displays a lot of the font's features. If I were displaying, say, a diplomatic
transcript of a Latin text, the font would be much smaller.

So the first section of this chapter will talk about how to subset the Junicode
font.

\section{Subsetting Junicode}

First the legalities. It is perfectly all right to create a modified version
of Junicode via subsetting, compress it into a webfont (almost certainly in
woff2 format), and host it on your web server. This is because “Junicode” is
not a “Reserved Font Name” (which complicates web use of
many fonts licensed under the Open Font License). If you are nervous about the
legal requirements of the Open Font License, you can change the font name to
something arbitrary with the \texttt{-‌-obfuscate-names} option
of the pyftsubset program, and you can embed the Open Font License, or a
link to it, in your CSS. These steps should settle any ambiguity about whether
you are in compliance with the license.

Generating a subsetted version of Junicode should be one of the last tasks you
perform before deploying your web page(s). Until then, it is recommended that
you work with the unmodified font. After subsetting, review your
pages thoroughly to make sure everything is displayed properly. If you have
forgotten to include a glyph in your subsetted font, you will see little boxes
where characters should be or, perhaps, the correct characters
displayed in the wrong typeface. If you have omitted features, you will see
default instead of transformed characters.

There are many subsetting programs, some online and very easy to use. But for
maximum control (and thus the smallest fonts), you should choose pyftsubset, a
part of the \href{https://github.com/fonttools/fonttools}{fontTools} library,
which runs under Python 3.7 or higher. This is a command-line tool which
takes a long list of arguments; you should create a shell script to run it.

Here is the script used to create the subsetted font for the property test
web page mentioned above:

\small\begin{verbatim}
#!/bin/zsh
pyftsubset JunicodeTwoBetaVF-Roman.ttf \
--flavor=woff2 --output-file=JunicodeVFsubset.woff2 \
--recommended-glyphs \
--text="jq" --text-file=Junicode-2-feature-test.html \
--layout-features+=liga,ss01,ss02,ss03,ss04,ss05,ss06,ss07,ss08,\
ss10,ss12,ss13,ss14,ss15,ss16,ss17,ss18,ss19,ss20,cv01,cv02,cv05,\
cv06,cv07,cv08,cv09,cv10,dlig,hlig,onum,pnum,pcap,smcp,c2sc,subs,\
sups,zero \
--layout-features-=rlig
\end{verbatim}

\normalsize\noindent For those unfamiliar with shell scripts, the first line specifies the shell
the script is to run under (in this case the default shell for Mac OS, but
\texttt{bash} is another possible choice), and the backslashes mean
that the command continues on the next line. The rest of the file is a list
of arguments passed to pyftsubset. Let's walk through them.

First after the program name comes the name of the unsubsetted, uncompressed
font file. After that,
the \texttt{--flavor} argument tells the program that you want a webfont in
woff2 format, and \texttt{--output} is the name of the font file you want the
program to save.

Having taken care of this preliminary business, we tell pyftsubset what we
want the font to contain.

\texttt{--recommended-glyphs} includes a few characters that
every font should have, according to the OpenType specification---though in
fact modern browsers don't care. It's best, however, to conform to the specification,
since it's impossible to say with absolute certainty that no program will
ever reject the font
because of the absence of these few characters.

\texttt{--text-file} is the name of a file to treat as a list of characters
that \textit{must} be included in the font. In this case I have simply used
the html file for the web page for this purpose. If your site contains
multiple web pages, your job will be more complicated. You must make sure the
text file contains all the characters used on the site---either that or
supplement the text file with a \texttt{--text} argument (which here adds
two lowercase letters that don't appear in the web page---just in case).
The text file will
contain only encoded characters—you don't have to worry here about unencoded
characters produced by OpenType features.

\texttt{--layout-features+} tells the program which OpenType features you
want to retain in the font. All others, except for the
\hyperlink{req}{Required Features}, are discarded. All of the characters
referenced in these features will also be included in the output file, as long
as those characters are variants of characters in your text file. For example,
the \textSourceText{smcp} (Small Caps) feature has many more small caps than there are letters
of the alphabet, but most of them are not included in the subsetted font. The
program's parsimony with characters keeps the font file as small as possible.
Note that some features are included automatically: \textSourceText{ccmp},
\textSourceText{locl}, \textSourceText{calt}, \textSourceText{liga}, \textSourceText{rlig},
\textSourceText{kern}, \textSourceText{mark}, and \textSourceText{mkmk}.


\texttt{--layout-features-} tells the program which OpenType features to omit.
Normally, \textSourceText{rlig} (Required Ligtures) is automatically
included in fonts by pyftsubset, but as it has no relevance to this web page,
it can be omitted.

These are the most useful arguments, but there are many more. Type
\texttt{pyftsubset --help} for a complete list. Once you have written your
script, run it (in Mac OS or Linux you need to make the file executable
by typing \texttt{chmod +x mysubsetscript} on the command line).

Before you put your subsetted font to work, check it carefully in a program
like \href{https://github.com/justvanrossum/fontgoggles}{FontGoggles},
which lets you preview the font and test all its OpenType features. If you
find errors, revise your script and run it again.

\section{Junicode and CSS/HTML}

This section assumes a basic knowledge of HTML (Hypertext Markup Language, used
to construct web pages) and CSS (Cascading Style Sheets, used to format them).
If you want to learn about these subjects, the number of good books and online
tutorials is so great that it makes no sense to try to list them. Just make sure
that the instructional materials you choose are of recent vintage, because the
relevant standards are always changing.

In the CSS for your web page, the \texttt{@font-face} at-rule for a variable
font is a little different
from the one for a static font in that the range of possible values for each
axis can be declared:

\small\begin{verbatim}
@font-face {
    font-family: "Junicode Two Beta VF";
    src: url("./webfiles/JunicodeVFsubset.woff2");
    font-weight: 300 700;
    font-stretch: 75% 125%;
    font-style: normal;
}
\end{verbatim}

\normalsize\noindent These ranges are not strictly necessary, but they will prevent your
supplying invalid values for \texttt{font-weight} and \texttt{font-stretch}
(that is, \texttt{width}) in other CSS rules.

Once you have declared the font, you can invoke it in setting up classes.
For example:

\small\begin{verbatim}
body {
  font-family: "Junicode Two Beta VF";
  font-size: 28px;
  font-weight: normal; /* that is, 400 */
  font-stretch: 112.5%; /* that is, semiexpanded */
}
h1 {
  font-family: "Junicode Two Beta VF";
  font-size: 125%;
  font-weight: 600; /* that is, semibold */
  font-stretch: 112.5%; /* that is, semiexpanded */
}
.annotation {
  font-size: 90%;
  font-weight: 300; /* that is, light */
  font-stretch: 87.5%; /* that is, semicondensed */
}
\end{verbatim}

\normalsize\noindent These classes should be tested in all browsers. If any fail to
display text properly, you can use \texttt{font-variation-settings}
instead of the high-level \texttt{font-weight} and \texttt{font-stretch}:

\small\begin{verbatim}
body {
  font-family: "Junicode Two Beta VF";
  font-size: 28px;
  font-variation-settings: "wght" 400, "wdth" 112.5;
}
h1 {
  font-family: "Junicode Two Beta VF";
  font-size: 125%;
  font-variation-settings: "wght" 600, "wdth" 112.5;
}
.annotation {
  font-size: 90%;
  font-variation-settings: "wght" 300, "wdth" 87.5;
}
\end{verbatim}

\normalsize\noindent To accommodate older browsers, you should make a selection of
Junicode static fonts, subset them, and include them in your CSS. For example,
if you need normal and bold weights of Junicode roman, your
\texttt{@font-face} at-rule may look like this:

\small\begin{verbatim}
@font-face {
    font-family: "Junicode Two Beta VF";
    src: url("./webfiles/JunicodeVFsubset.woff2");
    font-weight: 300 700;
    font-stretch: 75% 125%;
    font-style: normal;
}
@font-face {
    font-family: "Junicode Two Beta";
    src: url("./webfiles/JunicodeTwoBeta-Regular.woff2");
    font-weight: 400;
    font-style: normal;
}
@font-face {
    font-family: "Junicode Two Beta";
    src: url("./webfiles/JunicodeTwoBeta-Bold.woff2");
    font-weight: 700;
    font-style: normal;
}
\end{verbatim}

\normalsize\noindent Now use \texttt{@supports} in your CSS rules to determine which
font gets downloaded:

\small\begin{verbatim}
body {
  font-family: "Junicode Two Beta", serif;
}
@supports (font-variation-settings: normal) {
  body {
    font-family: "Junicode Two Beta VF", serif;
  }
}
b {
  font-weight: 700;
}
@supports (font-variation-settings: "wght" 700) {
  b {
    font-variation-settings: "wght" 700;
  }
}
\end{verbatim}

\normalsize\noindent The variable version of Junicode will be downloaded only if
the browser supports it, and the static version will be downloaded only if
needed.


\vspace*{\fill}
\begin{center}
{\stditalic{This document was set in 12pt Junicode SemiExpanded\\
using the {\XeLaTeX} typesetting system with fontspec for font management.\\
The source for the document, Feature\_Reference.tex, is available at}}\\
{\color{BrickRed}https://github.com/psb1558/Junicode-font.}
\end{center}
%\thispagestyle{plain}
\end{document}
